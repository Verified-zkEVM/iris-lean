\section{Soundness}
\label{sec:appendix:soundness}

\subsection{Soundness of Primitive Rules}
\label{sec:appendix:primitive-rules}


\subsubsection{Soundness of Distribution Ownership Rules}
\begin{lemma}
\label{proof:and-to-star}
  \Cref{rule:and-to-star} is sound.
\end{lemma}

\begin{proof}
  Assume a valid $a\in\Model_I$ is such that
  it satisfies $ P \land Q $.
  This means that for some $(\m{\salg},\m{\prob},\m{\permap}) \raLeq a$, both
    $P(\m{\salg},\m{\prob},\m{\permap})$ and
    $Q(\m{\salg},\m{\prob},\m{\permap})$
  hold.
  We want to prove $(P \ast Q)(a)$ holds.
  To this end, let
  $ (\m{\salg}_1, \m{\prob}_1, \m{\permap}_1) $ and
  $ (\m{\salg}_2, \m{\prob}_2, \m{\permap}_2) $
  be such that
  for every $i \in \idx(P)$:
  \begin{align*}
    \m{\salg}_1(i) &= \m{\salg}(i)
    &
    \m{\salg}_2(i) &= \set{\emptyset, \Outcomes}
    \\
    \m{\prob}_1(i) &= \m{\prob}(i)
    &
    \m{\prob}_2(i) &= \fun \event. \ITE{\event=\Outcomes}{1}{0}
    \\
    \m{\permap}_1(i) &= \m{\permap}(i)
    &
    \m{\permap}_2(i) &= \fun \wtv. 0
  \intertext{
    and for all $i\in I \setminus \idx(P)$:
  }
    \m{\salg}_2(i) &= \m{\salg}(i)
    &
    \m{\salg}_1(i) &= \set{\emptyset, \Outcomes}
    \\
    \m{\prob}_2(i) &= \m{\prob}(i)
    &
    \m{\prob}_1(i) &= \fun \event. \ITE{\event=\Outcomes}{1}{0}
    \\
    \m{\permap}_2(i) &= \m{\permap}(i)
    &
    \m{\permap}_1(i) &= \fun \wtv. 0
  \end{align*}
Clearly, by construction,
  $
    (\m{\salg}_1, \m{\prob}_1, \m{\permap}_1)
    \iprod
    (\m{\salg}_2, \m{\prob}_2, \m{\permap}_2)
    =
    (\m{\salg},\m{\prob},\m{\permap}).
  $
  and
  $P(\m{\salg}_1, \m{\prob}_1, \m{\permap}_1)$.
  Since $\idx(P) \inters \idx(Q) = \emptyset$,
  we also have
  $Q(\m{\salg}_2, \m{\prob}_2, \m{\permap}_2)$.
  Therefore,
  $(P \ast Q)(\m{\salg}, \m{\prob}, \m{\permap})$,
  and so $(P \ast Q)(a)$ by upward closure.
\end{proof} \begin{lemma}
\label{proof:dist-inj}
  \Cref{rule:dist-inj} is sound.
\end{lemma}

\begin{proof}
  Assume a valid $a\in\Model_I$ is such that both
  $ \distAs{E\at{i}}{\prob }(a) $ and
  $ \distAs{E\at{i}}{\prob'}(a) $
  hold.
  Let $ a = (\m{\salg}, \m{\prob}_0,\m{\permap}) $,
  then we know
  $ \prob = \m{\prob}_0 \circ \inv{E\at{i}} = \prob'$,
  which proves the claim.
\end{proof} \begin{lemma}
\label{proof:sure-merge}
  \Cref{rule:sure-merge} is sound.
\end{lemma}

\begin{proof}
  The proof for the forward direction is very similar to
  the one for~\cref{rule:sure-eq-inj}.
  For $a \in \Model_I$,
  if $(\sure{E_1\at{i}} \ast \sure{E_2 \at{i}})(a)$.
  Then there exists
  $a_1, a_2$ such that $a_1 \raOp a_2 \raLeq a$ and
  $\sure{E_1 \at{i}}(a_1)$,
  $\sure{E_2 \at{i}}(a_2)$.
  Say $a = (\m{\sigmaF}, \m{\mu}, \m{\permap})$,
  $a_1 = (\m{\sigmaF}_1, \m{\mu}_1, \m{\permap}_1)$
  and $a_2 = (\m{\sigmaF}_2, \m{\mu}_2, \m{\permap}_2)$.
  Then $\sure{E_1\at{i}}(a_1)$ implies that
  \begin{align*}
    \m{\mu}_1 (\inv{E_1\at{i}}(\True)) = 1
  \end{align*}
  And similarly,
  \begin{align*}
    \m{\mu}_2 (\inv{E_2\at{i} }(\True)) = 1
  \end{align*}
  Thus,
  \begin{align*}
    \m{\mu} (\inv{E_1\at{i}}(\True) \cap \inv{E_2\at{i}}(\True) )
    &= \m{\mu}_1 (\inv{E_1\at{i}}(\True))  \cdot \m{\mu}_2 (\inv{E_2\at{i}}(\True))
     = 1.
  \end{align*}
  Hence,
  \begin{align*}
    \m{\mu} (\inv{E_1\at{i} \land E_2\at{i}}(\True) )
    &= \m{\mu} (\inv{E_1\at{i}}(\True) \cap \inv{E_2\at{i}}(\True)) = 1
  \end{align*}
  Thus, $\sure{E_1\at{i} \land E_2\at{j}} (a)$.

  Now we prove the backwards direction:
  Say $a = (\m{\sigmaF}, \m{\mu}, \m{\permap})$.
  if  $\sure{E_1\at{i} \land E_2\at{j}} (a)$,
  then $\m{\mu} (\inv{E_1\at{i} \land E_2\at{i}}(\True)) = 1$,
  and then
  \begin{align*}
    \m{\mu} (\inv{E_1\at{i}}(\True))  &\geq \m{\mu} (\inv{E_1\at{i} \land E_2\at{i}}(\True)) = 1 \\
    \m{\mu} (\inv{E_2\at{i}}(\True))  &\geq \m{\mu} (\inv{E_1\at{i} \land E_2\at{i}}(\True)) = 1
  \end{align*}

  Let $\m{\sigmaF_1} = \closure{\inv{E_1\at{i}}(\True)}$
  and $\m{\sigmaF_2} = \closure{\inv{E_2\at{i}}(\True)}$.
  Then,
  \begin{gather*}
    \sure{E_1\at{i}} (\m{\sigmaF_1}, \constrain{\m{\mu}}{\m{\sigmaF_1}}, \fun \wtv. 0) \\
    \sure{E_2\at{i}} (\m{\sigmaF_2}, \constrain{\m{\mu}}{\m{\sigmaF_2}}, \fun \wtv. 0) \\
    (\m{\sigmaF_1}, \constrain{\m{\mu}}{\m{\sigmaF_1}}, \fun \wtv. 0) \ast  (\m{\sigmaF_2}, \constrain{\m{\mu}}{\m{\sigmaF_2}}, \fun \wtv. 0) \raLeq a
  \end{gather*}
  Thus, $\sure{E_1\at{i}} \ast \sure{E_2\at{i}}$ holds on $a$.
\end{proof} \begin{lemma}
\label{proof:sure-and-star}
  \Cref{rule:sure-and-star} is sound.
\end{lemma}

\begin{proof}
  Assume $a = (\m{\salg}, \m{\prob}, \m{\permap}) \in \Model_I$ and
  $(\sure{E\at{i}} \land P)(a)$ holds.
  We want to show that
  $(\sure{E\at{i}} \ast P)(a)$ holds.
  First note that:
  \begin{align*}
    (\sure{E\at{i}} \land P)(a)
    & \implies \sure{E\at{i}}(a) \land P(a) \\
    & \implies \almostM{E}{(\m{\salg}(i), \m{\prob}(i))}
    \land \m{\prob} \circ \inv{E\at{i}}(\true) = \dirac{\True}
    \land P(a)
  \end{align*}

  Define $\m{\salg}', \m{\permap}_{\aexpr}, \m{\permap}_{P}$ such that,
  for any $j \in I$:
  \begin{align*}
    \m{\salg}'(j) &=
    \begin{cases}
      \set{\emptyset, \Store} \CASE j \neq i\\
      \set{\emptyset, \Store, \inv{E\at{i}}(\True), \Store \setminus
      \inv{E\at{i}}(\True)} \OTHERWISE
    \end{cases}
    \\
    \m{\permap}_{\aexpr}(j) &=
    \begin{cases}
      \fun \wtv.0 \CASE j \ne i \\
      \fun \ip{x}{i}.
        \ITE{\p{x} \in \pvar(\aexpr)}{\m{\permap}(i)(\ip{x}{i})/2}{0}
    \CASE j = i
    \end{cases}
    \\
    \m{\permap}_{P}(j) &=
    \begin{cases}
      \m{\permap}(j) \CASE j \ne i \\
      \fun \ip{x}{i}.
        \ITE{\p{x} \in \pvar(\aexpr)}{\m{\permap}(i)(\ip{x}{i})/2}{\m{\permap}(i)(\ip{x}{i})}
    \CASE j = i
    \end{cases}
  \end{align*}
  By construction, we have
  $
    \m{\permap} = \m{\permap}_{\aexpr} \raOp \m{\permap}_{P}.
  $
  Now let:
  \begin{align*}
    b &= (\m{\salg}', \restr{\m{\prob}}{\m{\salg}'}, \m{\permap}_{\aexpr})
    &
    a' &= (\m{\salg}, \m{\prob}, \m{\permap}_{P})
  \end{align*}
  note that $\raValid(b)$ holds because $\m{\salg}'(i)$ can at best be non-trivial on $\pvar(\aexpr)$.
  The resource $a'$ is also valid, since $\m{\permap}_{P}$ has the same non-zero components as $\m{\permap}$.
  Then
  $\sure{E\at{i}}(b)$ holds because
  $\almostM{E}{( \m{\salg}'(i), \restr{\m{\prob}}{\m{\salg}'}(i) )}$
  and $\restr{\m{\prob}}{\m{\salg}'} \circ \inv{E\at{i}}
  = \m{\prob} \circ \inv{E\at{i}} = \dirac{\True}$.
  By applying~\cref{lemma:indep-prod-exists}, it is easy to show that
  $(\m{\salg}', \restr{\m{\prob}}{\m{\salg}'}) \iprod (\m{\salg}, \m{\prob})$
  is defined and is equal to $(\m{\salg}, \m{\prob})$.
  Therefore,
  $\raValid(b \raOp a)$ and $b \raOp a = a$.
  By the side condition $\psinv(P, \pvar(E\at{i}))$ and the fact that
  $\m{\permap}_{P}$ is a scaled down version of $\m{\permap}$,
  we obtain from $P(a)$ that $P(a')$ holds too.
  This proves that
  $(\sure{E\at{i}} \ast P)(a)$ holds, as desired.
\end{proof} \begin{lemma}
\label{proof:prod-split}
  \Cref{rule:prod-split} is sound.
\end{lemma}

\begin{proof}
  For any $(\m{\salg},\m{\prob}, \m{\permap})$ such that
  $(\distAs{(\aexpr_1\at{i}, \aexpr_2\at{i})}{\mu_1 \pprod \mu_2}) (\m{\salg},\m{\prob}, \m{\permap})$,
  by definition, it must
  \begin{align*}
      \E \m{\salg'},\m{\prob'}.
      (\Own{\m{\salg'},\m{\prob''}})(\m{\salg},\m{\prob}, \m{\permap}) *
    \almostM{(\aexpr_1, \aexpr_2)}{(\m{\salg'}(i),\m{\prob'}(i))}
    \land
    \mu_1 \pprod \mu_2 = \m{\prob'}(i) \circ \inv{(\aexpr_1, \aexpr_2)}
    .
  \end{align*}
  We can derive from it that
  \begin{align*}
    \E \m{\salg'},\m{\prob'}, \m{\permap'}.
      & (\m{\salg'},\m{\prob'}) \raLeq (\m{\salg},\m{\prob}, \m{\permap}) * \\
      & \Big( \forall a, b  \in A. \exists L_{a, b},U_{a,b} \in \salg'(i)  \st
        L_{a, b} \subs \inv{(\aexpr_1, \aexpr_2)}(a, b) \subs U_{a, b}
       \land
        \prob'(L_{a, b})=\prob'(U_{a, b}) \land  \\
      &
       \mu_1 \pprod \mu_2 (a, b)
= \m{\prob'}(i) (L_{a, b})
       = \m{\prob'}(i) (U_{a, b})
    \Big)
  \end{align*}
  Also, for any $a, b, a', b' \in A$ such that $a \neq a'$
  or $b \neq b'$, we have
  $L_{a,b}$ disjoint from $L_{a',b'}$ because on $L_{a,b} \inters L_{a',b'}$,
  the random variable $(\aexpr_1, \aexpr_2)$ maps to both
  $(a, b)$ and $(a',b')$.

  Define
  \[
    \m{\salg}_1(i) = \closure{\set{(\Union_{b \in A} L_{a, b} ) \mid a \in A} \union \set{(\Union_{b \in A} U_{a, b})  \mid a \in A}},
  \]
  and similarly define
    \[
      \m{\salg}_2(i) = \closure{\set{(\Union_{a \in A} L_{a, b} ) \mid b \in A} \union \set{(\Union_{a \in A} U_{a, b} )  \mid b \in A}}.
  \]
  Denote $\m{\prob'}$ restricted to $\m{\salg}_1$ as $\m{\prob'}_1$
  and $\m{\prob'}$ restricted to $\m{\salg}_2$ as $\m{\prob'}_2$.


  We want to show that
  $(\m{\salg}_1(i), \m{\prob'}_1(i)) \indepcomb (\m{\salg}_2(i), \m{\prob'}_2(i)) \extTo (\m{\salg'}(i), \m{\prob'}(i))$,
  which boils down to show that for any $\event_1 \in \m{\salg}_1(i)$, any
  $\event_2 \in \m{\salg}_2(i)$,
  \begin{align*}
    \m{\prob'}(\event_1 \inters \event_2) = \m{\prob'}_1(\event_1) \cdot  \m{\prob'}_2(\event_2)
  \end{align*}

      For convenience, we will
      denote $\union_{b \in A} L_{a, b}$ as $L_a$,
      denote $\union_{a \in A} L_{a, b}$ as $L_b$,
      denote $\union_{b \in A} U_{a, b}$ as $U_a$,
      and denote $\union_{a \in A} U_{a, b}$ as $U_b$.




      First, using a standard construction in measure theory proofs,
      we rewrite $\salg_1$ and $\salg_2$ as sigma algebra generated
      by sets of partitions.
      Specifically, $\salg_1$ is equivalent to
      \[
        \closure{\set{\Inters_{a \in S_1} L_a \inters \Inters_{a \in S_2} U_a \setminus (\Union_{a \in A \setminus S_1} L_a \union \Union_{a \in A \setminus S_2} U_a)  \mid S_1, S_2 \subseteq A}}
      \]
      and similarly, $\salg_2$ is equivalent to
      \[
        \closure{\set{\Inters_{b \in T_1} L_b \inters \Inters_{b \in T_2} U_b \setminus (\Union_{b \in A \setminus T_1} L_b \union \Union_{b \in A \setminus T_2} U_b)  \mid T_1, T_2 \subseteq A}}.
      \]
      Thus, by~\cref{lemma:sigma-alg-representation}, any event $\event_1$ in
      $\salg_1$ can be represented by
      \[
        \Dunion_{S_1 \in I_1, S_2 \in I_2}
        \Inters_{a \in S_1} L_a \inters \Inters_{a \in S_2} U_a \setminus (\Union_{a \in A \setminus S_1} L_a \union \Union_{a \in A \setminus S_2} U_a)
      \]
      for some $I_1, I_2 \subseteq \mathcal{P}(A)$, where
      $\mathcal{P}$ is the powerset over $A$.
      Similarly, any event $\event_2$ in $\salg_2$ can be represented by
      \[
        \Dunion_{S_3 \in I_3, S_4 \in I_4}
        \Inters_{b \in S_3} L_b \inters \Inters_{b \in S_4} U_b \setminus (\Union_{b \in A \setminus S_3} L_b \union \Union_{b \in A \setminus S_2} U_b)
      \]
      for some  $I_3, I_4 \subseteq \mathcal{P}(A)$.
      Thus, $\event_1 \inters \event_2$ can be represented as
      \begin{align*}
        \event_1 \inters \event_2
        &=(\Dunion_{S_1 \in I_1, S_2 \in I_2}
        \Inters_{a \in S_1} L_a \inters \Inters_{a \in S_2} U_a \setminus (\Union_{a \in A \setminus S_1} L_a \union \Union_{a \in A \setminus S_2} U_a) ) \\
        &\Inters
        (\Dunion_{S_3 \in I_3, S_4 \in I_4}
        \Inters_{b \in S_3} L_b \inters \Inters_{b \in S_4} U_b \setminus (\Union_{b \in A \setminus S_3} L_b \union \Union_{b \in A \setminus S_2} U_b) )\\
        = & \Dunion_{S_1 \in I_1, S_2 \in I_2, S_3 \in I_3, S_4 \in I_4} (\Inters_{a \in S_1} L_a \inters \Inters_{a \in S_2} U_a \setminus (\Union_{a \in A \setminus S_1} L_a \union \Union_{a \in A \setminus S_2} U_a) ) \\
          &\inters ( \Inters_{b \in S_3} L_b \inters \Inters_{b \in S_4} U_b \setminus (\Union_{b \in A \setminus S_3} L_b \union \Union_{b \in A \setminus S_2} U_b) )
      \end{align*}

      Because $L_{a,b}$ and $L_{a',b'}$ are disjoint as long as not
      $a = a'$ and $b = b'$,
      we have $L_a$ disjoint from $L_{a'}$ if $a \neq a'$.
      Thus,
      $\Inters_{a \in S_1} L_a \inters \Inters_{a \in S_2} U_a \setminus (\Union_{a \in A \setminus S_1} L_a \union \Union_{a \in A \setminus S_2} U_a)$
      is not empty only when $S_1$ is singleton and empty.
      \begin{itemize}
        \item If $S_1$ is empty,
      then
      \[
        \Inters_{a \in S_1} L_a \inters \Inters_{a \in S_2} U_a \setminus (\Union_{a \in A \setminus S_1} L_a \union \Union_{a \in A \setminus S_2} U_a)
        =  \Inters_{a \in S_2} U_a \setminus (\Union_{a \in A} L_a \union \Union_{a \in A \setminus S_2} U_a)
      \]
      has measure 0 because $\Union_{a \in A} L_a$ has measure 1.
        \item Otherwise, if $S_1$ is singleton, say $S_1 = \{a'\}$,
      then
      \begin{align*}
        \Inters_{a \in S_1} L_a \inters \Inters_{a \in S_2} U_a \setminus (\Union_{a \in A \setminus S_1} L_a \union \Union_{a \in A \setminus S_2} U_a)
        &= L_{a'} \inters \Inters_{a \in S_2} U_a \setminus \Union_{a \in A \setminus S_2} U_a).
      \end{align*}
Furthermore,
      \begin{align*}
        \m{\prob'}(\Inters_{a \in S_2} U_a)
        &= \m{\prob'}(\Inters_{a \in S_2} L_a \disjunion (U_a \setminus L_a)) \\
        &= \m{\prob'}(\Inters_{a \in S_2} L_a) + 0
      \end{align*}
      And $\Inters_{a \in S_2} L_a$ is non-empty only if
      $S_2$ is a singleton set or empty set.
      Thus, $L_{a'} \inters \Inters_{a \in S_2} U_a \setminus \Union_{a \in A \setminus S_2} U_a) \subseteq \Inters_{a \in S_2} U_a$ has non-zero measure only if
      $S_2$ is empty or a singleton set.
\begin{itemize}
        \item When $S_2$ is empty,
      \begin{align*}
        L_{a'} \inters \Inters_{a \in S_2} U_a \setminus \Union_{a \in A \setminus S_2} U_a
        &= L_{a'} \setminus \Union_{a \in A} U_a
        \subseteq L_{a'} \setminus  U_{a'}
        =\emptyset
      \end{align*}
        \item When $S_2 = \{a'\}$,
      \begin{align*}
        L_{a'} \inters \Inters_{a \in S_2} U_a \setminus \Union_{a \in A \setminus S_2} U_a
        &= L_{a'} \setminus \Union_{a \in A, a \neq a'} U_a .
      \end{align*}
        \item
      When $S_2 = \{a''\}$ for some $a'' \neq a'$
      \begin{align*}
        L_{a'} \inters \Inters_{a \in S_2} U_a \setminus \Union_{a \in A \setminus S_2} U_a
        &= L_{a'} \inters U_{a''} \setminus \Union_{a \in A, a \neq a''} U_a \\
        &= \emptyset
      \end{align*}
      \end{itemize}
      \end{itemize}

Thus,
      \begin{align*}
         \m{\prob'}(\event_1)
        = & \m{\prob'}\Big(\Union_{S_1 \in I_1, S_2 \in I_2} \Inters_{a \in S_1} L_a \inters \Inters_{a \in S_2} U_a \setminus (\Union_{a \in A \setminus S_1} L_a \union \Union_{a \in A \setminus S_2} U_a) \inters) \\
          = & \m{\prob'}\Big(\Union_{\{a'\} \in I_1, S_2 \in I_2} ( L_{a'} \inters \Inters_{a \in S_2} U_a \setminus \Union_{a \in A \setminus S_2} U_a) \Big) \\
          = & \m{\prob'}\Big(\Union_{\{a'\} \in I_1 \inters I_2} L_{a'} \inters U_{a'} \setminus \Union_{a \in A, a \neq a'} U_a \Big)  \\
          = & \m{\prob'}\Big(\Union_{\{a'\} \in I_1 \inters I_2} ( L_{a'} \setminus \Union_{a \in A, a \neq a'} U_a )  \Big) \\
          = & \m{\prob'}\Big(\Union_{\{a'\} \in I_1 \inters I_2} ( L_{a'} \setminus \Union_{a \in A, a \neq a'} (L_a \Union (U_a \setminus L_a)) )  \Big) \\
          = & \m{\prob'}\Big(\Union_{\{a'\} \in I_1 \inters I_2} ( L_{a'} \setminus \Union_{a \in A, a \neq a'} (L_a ) ) \Big) \\
          = & \m{\prob'}\Big(\Union_{\{a'\} \in I_1 \inters I_2} L_{a'} \Big)
      \end{align*}
      Denote $\Union_{\{a'\} \in I_1 \inters I_2} L_{a'}$
      as $\event'_1$.
      And $\event_1 \setminus \event'_1$ and $\event'_1 \setminus \event_1$ both have measure 0.

      Similar results hold for $\event_2$ as well, and we can show that
      \begin{align*}
        \m{\prob'}(\event_2)
         = & \m{\prob'}\Big(\Union_{\{b'\} \in I_3 \inters I_4} L_{b'} \Big)
      \end{align*}
      Denote $\Union_{\{b'\} \in I_3 \inters I_4} L_{b'}$
      as $\event'_2$.
      And $\event_2 \setminus \event'_2$ and $\event'_2 \setminus \event_2$ both have measure 0.


      Thus,
      \begin{align*}
         \m{\prob'}(\event_1 \inters \event_2)
        =& \m{\prob'}(\event_1 \inters \event_2 \inters \event'_1)
        + \m{\prob'}((\event_1 \inters \event_2) \setminus \event'_1)\\
        =& \m{\prob'}(\event_1 \inters \event_2 \inters \event'_1)
        + 0 \\
        =& \m{\prob'}(\event_1 \inters \event_2 \inters \event'_1 \inters \event'_2) + \m{\prob'}((\event_1 \inters \event_2 \inters \event'_1) \setminus \event'_2) + 0 \\
        =& \m{\prob'}(\event_1 \inters \event_2 \inters \event'_1 \inters \event'_2) + 0 + 0 \\
        =&  \m{\prob'}(\event_1 \inters \event_2 \inters \event'_1 \inters \event'_2) +  \m{\prob'}((\event_2 \inters \event'_1 \inters \event'_2 ) \setminus \event_1) \\
        =&  \m{\prob'}(\event_2 \inters \event'_1 \inters \event'_2)  \\
        =&  \m{\prob'}(\event_2 \inters \event'_1 \inters \event'_2) + \m{\prob'}((\event'_1 \inters \event'_2 ) \setminus \event_2) \\
        =&  \m{\prob'}(\event'_1 \inters \event'_2) \\
        =&  \m{\prob'}\left((\Union_{\{a'\} \in I_1 \inters I_2} L_{a'}) \inters (\Union_{\{b'\} \in I_3 \inters I_4} L_{b'})\right) \\
         =&  \m{\prob'}\left(\Union_{\{a'\} \in I_1 \inters I_2, \{b'\} \in I_3 \inters I_4} L_{a', b'}\right) \\
         =&  \sum_{\substack{\{a'\} \in I_1 \inters I_2 \\ \{b'\} \in I_3 \inters I_4}} \m{\prob'}(L_{a', b'})
       \end{align*}

Next we show that
       $         \m{\prob'}(i) (L_{a, b})
         = \m{\prob'}(i) (\event_1) \cdot \m{\prob'}(i) (\event_2)$.
         Note that
      $\m{\prob'}(L_a)  = \sum_{b} \m{\prob'}(L_{a,b}) = \m{\prob'}(\inv{\aexpr_1}(a))$,
      and
      $\m{\prob'}(L_b)  = \sum_{a} \m{\prob'}(L_{a,b}) = \m{\prob'}(\inv{\aexpr_2}(b))$.
      And $\mu_1 \pprod \mu_2 = \m{\prob'}(i) \circ \inv{(\aexpr_1, \aexpr_2)}$
      implies that
      \begin{align*}
        \m{\prob'}(i) (L_{a, b})
        &= \mu_1 \pprod \mu_2 (a, b)\\
        &=\mu_1(a) \cdot \mu_2(b)
      \end{align*}
      Then
      \begin{align*}
        \mu_1(a)
        &= \mu_1(a) \cdot \sum_{b \in A} \mu_2(b)\\
        &= \sum_{b \in A} \mu_1(a) \cdot \mu_2(b) \\
        &= \sum_{b \in A} \m{\prob'}(i) (L_{a, b}) \\
        &= \m{\prob'}(i) \left(\sum_{b \in A} L_{a, b}\right)  \\
        &= \m{\prob'}(i) (L_a),
      \end{align*}
      and similarly,
      \begin{align*}
        \mu_2(b)
        &= \left(\sum_{a \in A} \mu_1(a)\right) \cdot \mu_2(b)\\
        &= \sum_{a \in A} (\mu_1(a) \cdot \mu_2(b))\\
        &= \sum_{a \in A} \m{\prob'}(i) (L_{a, b}) \\
        &= \m{\prob'}(i) \left(\sum_{a \in A} L_{a, b}\right)  \\
        &= \m{\prob'}(i) (L_b).
      \end{align*}
      Thus,
      \begin{align*}
         \m{\prob'}(i) (L_{a, b})
         &=\mu_1(a) \cdot \mu_2(b)
         = \m{\prob'}(i) (L_a) \cdot \m{\prob'}(i) (L_b)
      \end{align*}



Therefore,
       \begin{align*}
         \m{\prob'}(\event_1 \inters \event_2)
         =&  \sum_{\substack{\{a'\} \in I_1 \inters I_2 \\ \{b'\} \in I_3 \inters I_4}} \m{\prob'}(L_{a', b'}) \\
         =&  \sum_{\substack{\{a'\} \in I_1 \inters I_2 \\ \{b'\} \in I_3 \inters I_4}} \m{\prob'}(L_{a'}) \cdot \m{\prob'}(L_{b'}) \\
       =&  \sum_{\mathclap{\{a'\} \in I_1 \inters I_2}}  \m{\prob'}(L_{a'}) \cdot \sum_{\mathclap{\{b'\} \in I_3 \inters I_4}} \m{\prob'}(L_{b'}) \\
       =& \m{\prob'}(\event_1) \cdot \m{\prob'}(\event_2)\\
       =& \m{\prob_1'}(\event_1) \cdot \m{\prob_2'}(\event_2)
      \end{align*}

  Thus we have
  $(\m{\salg}_1, \m{\prob'}_1) \iprod (\m{\salg}_2, \m{\prob'}_2) \extTo (\m{\salg'}, \m{\prob'})$.
  Let $\m{\permap_1} = \m{\permap_2} = \fun x. \m{\permap'}(x)/2$.

  Next we show that $\distAs{\aexpr_1}{\mu_1} (\m{\salg}_1, \m{\prob'}_1, \m{\permap_1}) $ and $\distAs{\aexpr_2}{\mu_2} (\m{\salg}_2, \m{\prob'}_2, \m{\permap_2})$.
  By definition,
  $\distAs{\aexpr_1}{\mu_1} (\m{\salg}_1, \m{\prob'}_1, \m{\permap_1})$
  is equivalent to
\begin{align*}
      \E \m{\salg''},\m{\prob''}.
      (\Own{\m{\salg''},\m{\prob''}})(\m{\salg}_1, \m{\prob'}_1, \m{\permap_1}) *
    \almostM{\aexpr_1}{(\m{\salg''}(i), \m{\prob''}(i))}
    \land
    \mu_1 = \m{\prob''}(i) \circ \inv{\aexpr_1}
    ,
  \end{align*}
which is equivalent to
\begin{multline*}
      \E \m{\salg''},\m{\prob''}.
      (\m{\salg''},\m{\prob''}) \raLeq (\m{\salg}_1, \m{\prob'}_1) *
      \bigl(\forall a \in A. \exists S_a, T_a \in \ \m{\salg''}(i).\\
      S_a \subseteq \inv{\aexpr_1}(a) \subseteq T_a  \land
      \m{\prob''}(i)(S_a) =  \m{\prob''}(i)(S_a) \land
      \mu_1(a) = \m{\prob''}(i)(S_a) = \m{\prob''}(i)(T_{a})
      \bigr)
  \end{multline*}
We can pick the existential witness to be
  $\m{\salg}_1, \m{\prob'}_1$.
  For any $a \in A$,
  $ \inv{\aexpr_1}(a) = \Union_{b \in A}\inv{(\aexpr_1, \aexpr_2)} (a, b)$.
  Because we have $L_{a, b} \subseteq \inv{(\aexpr_1, \aexpr_2)} (a, b) \subseteq U_{a,b}$,
  then
  \[
   \Union_{b \in A} L_{a, b} \subseteq
   \inv{\aexpr_1}(a) = \Union_{b \in A}\inv{(\aexpr_1, \aexpr_2)} (a, b)
   \subseteq \Union_{b \in A} U_{a, b} .
 \]
  By definition, for each $a$,
  $\Union_{b \in A} L_{a, b} \in \m{\salg}_1(i)$ and
  $\Union_{b \in A} U_{a, b} \in \m{\salg}_1(i)$,
  and we also have
  \begin{align*}
     \m{\prob'}_1(i) (\Union_{b \in A} L_{a, b})
     &= \sum_{b \in A} \m{\prob'}_1(i) (L_{a,b})\\
     &= \sum_{b \in A} \m{\prob'}_1(i) (U_{a,b})\\
     &= \m{\prob'}_1(i) \bigl(\Union_{b \in A} U_{a, b}\bigr)\\
     &= \mu_1(a)
  \end{align*}
  Thus, $S_a = \Union_{b \in A} L_{a, b}$ and
  $T_a = \Union_{b \in A} U_{a, b}$ witnesses the conditions needed
  for
  $\distAs{\aexpr_1}{\mu_1} (\m{\salg}_1, \m{\prob'}_1, \m{\permap_1}) $.
  And similarly, we have $\distAs{\aexpr_2}{\mu_2} (\m{\salg}_2, \m{\prob'}_2, \m{\permap_2}) $.
\end{proof}
 
