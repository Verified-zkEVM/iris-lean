\subsection{Soundness of Primitive Rules}
\label{sec:appendix:primitive-rules}


\subsubsection{Soundness of Distribution Ownership Rules}
\begin{lemma}
\label{proof:and-to-star}
  \Cref{rule:and-to-star} is sound.
\end{lemma}

\begin{proof}
  Assume a valid $a\in\Model_I$ is such that
  it satisfies $ P \land Q $.
  This means that for some $(\m{\salg},\m{\prob},\m{\permap}) \raLeq a$, both
    $P(\m{\salg},\m{\prob},\m{\permap})$ and
    $Q(\m{\salg},\m{\prob},\m{\permap})$
  hold.
  We want to prove $(P \ast Q)(a)$ holds.
  To this end, let
  $ (\m{\salg}_1, \m{\prob}_1, \m{\permap}_1) $ and
  $ (\m{\salg}_2, \m{\prob}_2, \m{\permap}_2) $
  be such that
  for every $i \in \idx(P)$:
  \begin{align*}
    \m{\salg}_1(i) &= \m{\salg}(i)
    &
    \m{\salg}_2(i) &= \set{\emptyset, \Outcomes}
    \\
    \m{\prob}_1(i) &= \m{\prob}(i)
    &
    \m{\prob}_2(i) &= \fun \event. \ITE{\event=\Outcomes}{1}{0}
    \\
    \m{\permap}_1(i) &= \m{\permap}(i)
    &
    \m{\permap}_2(i) &= \fun \wtv. 0
  \intertext{
    and for all $i\in I \setminus \idx(P)$:
  }
    \m{\salg}_2(i) &= \m{\salg}(i)
    &
    \m{\salg}_1(i) &= \set{\emptyset, \Outcomes}
    \\
    \m{\prob}_2(i) &= \m{\prob}(i)
    &
    \m{\prob}_1(i) &= \fun \event. \ITE{\event=\Outcomes}{1}{0}
    \\
    \m{\permap}_2(i) &= \m{\permap}(i)
    &
    \m{\permap}_1(i) &= \fun \wtv. 0
  \end{align*}
Clearly, by construction,
  $
    (\m{\salg}_1, \m{\prob}_1, \m{\permap}_1)
    \iprod
    (\m{\salg}_2, \m{\prob}_2, \m{\permap}_2)
    =
    (\m{\salg},\m{\prob},\m{\permap}).
  $
  and
  $P(\m{\salg}_1, \m{\prob}_1, \m{\permap}_1)$.
  Since $\idx(P) \inters \idx(Q) = \emptyset$,
  we also have
  $Q(\m{\salg}_2, \m{\prob}_2, \m{\permap}_2)$.
  Therefore,
  $(P \ast Q)(\m{\salg}, \m{\prob}, \m{\permap})$,
  and so $(P \ast Q)(a)$ by upward closure.
\end{proof} \begin{lemma}
\label{proof:dist-inj}
  \Cref{rule:dist-inj} is sound.
\end{lemma}

\begin{proof}
  Assume a valid $a\in\Model_I$ is such that both
  $ \distAs{E\at{i}}{\prob }(a) $ and
  $ \distAs{E\at{i}}{\prob'}(a) $
  hold.
  Let $ a = (\m{\salg}, \m{\prob}_0,\m{\permap}) $,
  then we know
  $ \prob = \m{\prob}_0 \circ \inv{E\at{i}} = \prob'$,
  which proves the claim.
\end{proof}
 \begin{lemma}
\label{proof:sure-merge}
  \Cref{rule:sure-merge} is sound.
\end{lemma}

\begin{proof}
  The proof for the forward direction is very similar to
  the one for~\cref{rule:sure-eq-inj}.
  For $a \in \Model_I$,
  if $(\sure{E_1\at{i}} \ast \sure{E_2 \at{i}})(a)$.
  Then there exists
  $a_1, a_2$ such that $a_1 \raOp a_2 \raLeq a$ and
  $\sure{E_1 \at{i}}(a_1)$,
  $\sure{E_2 \at{i}}(a_2)$.
  Say $a = (\m{\sigmaF}, \m{\mu}, \m{\permap})$,
  $a_1 = (\m{\sigmaF}_1, \m{\mu}_1, \m{\permap}_1)$
  and $a_2 = (\m{\sigmaF}_2, \m{\mu}_2, \m{\permap}_2)$.
  Then $\sure{E_1\at{i}}(a_1)$ implies that
  \begin{align*}
    \m{\mu}_1 (\inv{E_1\at{i}}(\True)) = 1
  \end{align*}
  And similarly,
  \begin{align*}
    \m{\mu}_2 (\inv{E_2\at{i} }(\True)) = 1
  \end{align*}
  Thus,
  \begin{align*}
    \m{\mu} (\inv{E_1\at{i}}(\True) \cap \inv{E_2\at{i}}(\True) )
    &= \m{\mu}_1 (\inv{E_1\at{i}}(\True))  \cdot \m{\mu}_2 (\inv{E_2\at{i}}(\True))
     = 1.
  \end{align*}
  Hence,
  \begin{align*}
    \m{\mu} (\inv{E_1\at{i} \land E_2\at{i}}(\True) )
    &= \m{\mu} (\inv{E_1\at{i}}(\True) \cap \inv{E_2\at{i}}(\True)) = 1
  \end{align*}
  Thus, $\sure{E_1\at{i} \land E_2\at{j}} (a)$.

  Now we prove the backwards direction:
  Say $a = (\m{\sigmaF}, \m{\mu}, \m{\permap})$.
  if  $\sure{E_1\at{i} \land E_2\at{j}} (a)$,
  then $\m{\mu} (\inv{E_1\at{i} \land E_2\at{i}}(\True)) = 1$,
  and then
  \begin{align*}
    \m{\mu} (\inv{E_1\at{i}}(\True))  &\geq \m{\mu} (\inv{E_1\at{i} \land E_2\at{i}}(\True)) = 1 \\
    \m{\mu} (\inv{E_2\at{i}}(\True))  &\geq \m{\mu} (\inv{E_1\at{i} \land E_2\at{i}}(\True)) = 1
  \end{align*}

  Let $\m{\sigmaF_1} = \closure{\inv{E_1\at{i}}(\True)}$
  and $\m{\sigmaF_2} = \closure{\inv{E_2\at{i}}(\True)}$.
  Then,
  \begin{gather*}
    \sure{E_1\at{i}} (\m{\sigmaF_1}, \constrain{\m{\mu}}{\m{\sigmaF_1}}, \fun \wtv. 0) \\
    \sure{E_2\at{i}} (\m{\sigmaF_2}, \constrain{\m{\mu}}{\m{\sigmaF_2}}, \fun \wtv. 0) \\
    (\m{\sigmaF_1}, \constrain{\m{\mu}}{\m{\sigmaF_1}}, \fun \wtv. 0) \ast  (\m{\sigmaF_2}, \constrain{\m{\mu}}{\m{\sigmaF_2}}, \fun \wtv. 0) \raLeq a
  \end{gather*}
  Thus, $\sure{E_1\at{i}} \ast \sure{E_2\at{i}}$ holds on $a$.
\end{proof} \begin{lemma}
\label{proof:sure-and-star}
  \Cref{rule:sure-and-star} is sound.
\end{lemma}

\begin{proof}
  Assume $a = (\m{\salg}, \m{\prob}, \m{\permap}) \in \Model_I$ and
  $(\sure{E\at{i}} \land P)(a)$ holds.
  We want to show that
  $(\sure{E\at{i}} \ast P)(a)$ holds.
  First note that:
  \begin{align*}
    (\sure{E\at{i}} \land P)(a)
    & \implies \sure{E\at{i}}(a) \land P(a) \\
    & \implies \almostM{E}{(\m{\salg}(i), \m{\prob}(i))}
    \land \m{\prob} \circ \inv{E\at{i}}(\true) = \dirac{\True}
    \land P(a)
  \end{align*}

  Define $\m{\salg}', \m{\permap}_{\aexpr}, \m{\permap}_{P}$ such that,
  for any $j \in I$:
  \begin{align*}
    \m{\salg}'(j) &=
    \begin{cases}
      \set{\emptyset, \Store} \CASE j \neq i\\
      \set{\emptyset, \Store, \inv{E\at{i}}(\True), \Store \setminus
      \inv{E\at{i}}(\True)} \OTHERWISE
    \end{cases}
    \\
    \m{\permap}_{\aexpr}(j) &=
    \begin{cases}
      \fun \wtv.0 \CASE j \ne i \\
      \fun \ip{x}{i}.
        \ITE{\p{x} \in \pvar(\aexpr)}{\m{\permap}(i)(\ip{x}{i})/2}{0}
    \CASE j = i
    \end{cases}
    \\
    \m{\permap}_{P}(j) &=
    \begin{cases}
      \m{\permap}(j) \CASE j \ne i \\
      \fun \ip{x}{i}.
        \ITE{\p{x} \in \pvar(\aexpr)}{\m{\permap}(i)(\ip{x}{i})/2}{\m{\permap}(i)(\ip{x}{i})}
    \CASE j = i
    \end{cases}
  \end{align*}
  By construction, we have
  $
    \m{\permap} = \m{\permap}_{\aexpr} \raOp \m{\permap}_{P}.
  $
  Now let:
  \begin{align*}
    b &= (\m{\salg}', \restr{\m{\prob}}{\m{\salg}'}, \m{\permap}_{\aexpr})
    &
    a' &= (\m{\salg}, \m{\prob}, \m{\permap}_{P})
  \end{align*}
  note that $\raValid(b)$ holds because $\m{\salg}'(i)$ can at best be non-trivial on $\pvar(\aexpr)$.
  The resource $a'$ is also valid, since $\m{\permap}_{P}$ has the same non-zero components as $\m{\permap}$.
  Then
  $\sure{E\at{i}}(b)$ holds because
  $\almostM{E}{( \m{\salg}'(i), \restr{\m{\prob}}{\m{\salg}'}(i) )}$
  and $\restr{\m{\prob}}{\m{\salg}'} \circ \inv{E\at{i}}
  = \m{\prob} \circ \inv{E\at{i}} = \dirac{\True}$.
  By applying~\cref{lemma:indep-prod-exists}, it is easy to show that
  $(\m{\salg}', \restr{\m{\prob}}{\m{\salg}'}) \iprod (\m{\salg}, \m{\prob})$
  is defined and is equal to $(\m{\salg}, \m{\prob})$.
  Therefore,
  $\raValid(b \raOp a)$ and $b \raOp a = a$.
  By the side condition $\psinv(P, \pvar(E\at{i}))$ and the fact that
  $\m{\permap}_{P}$ is a scaled down version of $\m{\permap}$,
  we obtain from $P(a)$ that $P(a')$ holds too.
  This proves that
  $(\sure{E\at{i}} \ast P)(a)$ holds, as desired.
\end{proof}
 \begin{lemma}
\label{proof:prod-split}
  \Cref{rule:prod-split} is sound.
\end{lemma}

\begin{proof}
  For any $(\m{\salg},\m{\prob}, \m{\permap})$ such that
  $(\distAs{(\aexpr_1\at{i}, \aexpr_2\at{i})}{\mu_1 \pprod \mu_2}) (\m{\salg},\m{\prob}, \m{\permap})$,
  by definition, it must
  \begin{align*}
      \E \m{\salg'},\m{\prob'}.
      (\Own{\m{\salg'},\m{\prob'}})(\m{\salg},\m{\prob}, \m{\permap}) *
    \almostM{(\aexpr_1, \aexpr_2)}{(\m{\salg'}(i),\m{\prob'}(i))}
    \land
    \mu_1 \pprod \mu_2 = \m{\prob'}(i) \circ \inv{(\aexpr_1, \aexpr_2)}
    .
  \end{align*}
  We can derive from it that
  \begin{align*}
    \E \m{\salg'},\m{\prob'}, \m{\permap'}.
      & (\m{\salg'},\m{\prob'}) \raLeq (\m{\salg},\m{\prob}, \m{\permap}) * \\
      & \Big( \forall a, b  \in A. \exists L_{a, b},U_{a,b} \in \salg'(i)  \st
        L_{a, b} \subs \inv{(\aexpr_1, \aexpr_2)}(a, b) \subs U_{a, b}
       \land
        \prob'(L_{a, b})=\prob'(U_{a, b}) \land  \\
      &
       \mu_1 \pprod \mu_2 (a, b)
= \m{\prob'}(i) (L_{a, b})
       = \m{\prob'}(i) (U_{a, b})
    \Big)
  \end{align*}
  Also, for any $a, b, a', b' \in A$ such that $a \neq a'$
  or $b \neq b'$, we have
  $L_{a,b}$ disjoint from $L_{a',b'}$ because on $L_{a,b} \inters L_{a',b'}$,
  the random variable $(\aexpr_1, \aexpr_2)$ maps to both
  $(a, b)$ and $(a',b')$.

  Define
  \[
    \m{\salg}_1(i) = \closure{\set{(\Union_{b \in A} L_{a, b} ) \mid a \in A} \union \set{(\Union_{b \in A} U_{a, b})  \mid a \in A}},
  \]
  and similarly define
    \[
      \m{\salg}_2(i) = \closure{\set{(\Union_{a \in A} L_{a, b} ) \mid b \in A} \union \set{(\Union_{a \in A} U_{a, b} )  \mid b \in A}}.
  \]
  Denote $\m{\prob'}$ restricted to $\m{\salg}_1$ as $\m{\prob'}_1$
  and $\m{\prob'}$ restricted to $\m{\salg}_2$ as $\m{\prob'}_2$.


  We want to show that
  $(\m{\salg}_1(i), \m{\prob'}_1(i)) \indepcomb (\m{\salg}_2(i), \m{\prob'}_2(i)) \extTo (\m{\salg'}(i), \m{\prob'}(i))$,
  which boils down to show that for any $\event_1 \in \m{\salg}_1(i)$, any
  $\event_2 \in \m{\salg}_2(i)$,
  \begin{align*}
    \m{\prob'}(\event_1 \inters \event_2) = \m{\prob'}_1(\event_1) \cdot  \m{\prob'}_2(\event_2)
  \end{align*}

      For convenience, we will
      denote $\union_{b \in A} L_{a, b}$ as $L_a$,
      denote $\union_{a \in A} L_{a, b}$ as $L_b$,
      denote $\union_{b \in A} U_{a, b}$ as $U_a$,
      and denote $\union_{a \in A} U_{a, b}$ as $U_b$.




      First, using a standard construction in measure theory proofs,
      we rewrite $\salg_1$ and $\salg_2$ as sigma algebra generated
      by sets of partitions.
      Specifically, $\salg_1$ is equivalent to
      \[
        \closure{\set{\Inters_{a \in S_1} L_a \inters \Inters_{a \in S_2} U_a \setminus (\Union_{a \in A \setminus S_1} L_a \union \Union_{a \in A \setminus S_2} U_a)  \mid S_1, S_2 \subseteq A}}
      \]
      and similarly, $\salg_2$ is equivalent to
      \[
        \closure{\set{\Inters_{b \in T_1} L_b \inters \Inters_{b \in T_2} U_b \setminus (\Union_{b \in A \setminus T_1} L_b \union \Union_{b \in A \setminus T_2} U_b)  \mid T_1, T_2 \subseteq A}}.
      \]
      Thus, by~\cref{lemma:sigma-alg-representation}, any event $\event_1$ in
      $\salg_1$ can be represented by
      \[
        \Dunion_{S_1 \in I_1, S_2 \in I_2}
        \Inters_{a \in S_1} L_a \inters \Inters_{a \in S_2} U_a \setminus (\Union_{a \in A \setminus S_1} L_a \union \Union_{a \in A \setminus S_2} U_a)
      \]
      for some $I_1, I_2 \subseteq \mathcal{P}(A)$, where
      $\mathcal{P}$ is the powerset over $A$.
      Similarly, any event $\event_2$ in $\salg_2$ can be represented by
      \[
        \Dunion_{S_3 \in I_3, S_4 \in I_4}
        \Inters_{b \in S_3} L_b \inters \Inters_{b \in S_4} U_b \setminus (\Union_{b \in A \setminus S_3} L_b \union \Union_{b \in A \setminus S_2} U_b)
      \]
      for some  $I_3, I_4 \subseteq \mathcal{P}(A)$.
      Thus, $\event_1 \inters \event_2$ can be represented as
      \begin{align*}
        \event_1 \inters \event_2
        &=(\Dunion_{S_1 \in I_1, S_2 \in I_2}
        \Inters_{a \in S_1} L_a \inters \Inters_{a \in S_2} U_a \setminus (\Union_{a \in A \setminus S_1} L_a \union \Union_{a \in A \setminus S_2} U_a) ) \\
        &\Inters
        (\Dunion_{S_3 \in I_3, S_4 \in I_4}
        \Inters_{b \in S_3} L_b \inters \Inters_{b \in S_4} U_b \setminus (\Union_{b \in A \setminus S_3} L_b \union \Union_{b \in A \setminus S_2} U_b) )\\
        = & \Dunion_{S_1 \in I_1, S_2 \in I_2, S_3 \in I_3, S_4 \in I_4} (\Inters_{a \in S_1} L_a \inters \Inters_{a \in S_2} U_a \setminus (\Union_{a \in A \setminus S_1} L_a \union \Union_{a \in A \setminus S_2} U_a) ) \\
          &\inters ( \Inters_{b \in S_3} L_b \inters \Inters_{b \in S_4} U_b \setminus (\Union_{b \in A \setminus S_3} L_b \union \Union_{b \in A \setminus S_2} U_b) )
      \end{align*}

      Because $L_{a,b}$ and $L_{a',b'}$ are disjoint as long as not
      $a = a'$ and $b = b'$,
      we have $L_a$ disjoint from $L_{a'}$ if $a \neq a'$.
      Thus,
      $\Inters_{a \in S_1} L_a \inters \Inters_{a \in S_2} U_a \setminus (\Union_{a \in A \setminus S_1} L_a \union \Union_{a \in A \setminus S_2} U_a)$
      is not empty only when $S_1$ is singleton and empty.
      \begin{itemize}
        \item If $S_1$ is empty,
      then
      \[
        \Inters_{a \in S_1} L_a \inters \Inters_{a \in S_2} U_a \setminus (\Union_{a \in A \setminus S_1} L_a \union \Union_{a \in A \setminus S_2} U_a)
        =  \Inters_{a \in S_2} U_a \setminus (\Union_{a \in A} L_a \union \Union_{a \in A \setminus S_2} U_a)
      \]
      has measure 0 because $\Union_{a \in A} L_a$ has measure 1.
        \item Otherwise, if $S_1$ is singleton, say $S_1 = \{a'\}$,
      then
      \begin{align*}
        \Inters_{a \in S_1} L_a \inters \Inters_{a \in S_2} U_a \setminus (\Union_{a \in A \setminus S_1} L_a \union \Union_{a \in A \setminus S_2} U_a)
        &= L_{a'} \inters \Inters_{a \in S_2} U_a \setminus \Union_{a \in A \setminus S_2} U_a).
      \end{align*}
Furthermore,
      \begin{align*}
        \m{\prob'}(\Inters_{a \in S_2} U_a)
        &= \m{\prob'}(\Inters_{a \in S_2} L_a \disjunion (U_a \setminus L_a)) \\
        &= \m{\prob'}(\Inters_{a \in S_2} L_a) + 0
      \end{align*}
      And $\Inters_{a \in S_2} L_a$ is non-empty only if
      $S_2$ is a singleton set or empty set.
      Thus, $L_{a'} \inters \Inters_{a \in S_2} U_a \setminus \Union_{a \in A \setminus S_2} U_a) \subseteq \Inters_{a \in S_2} U_a$ has non-zero measure only if
      $S_2$ is empty or a singleton set.
\begin{itemize}
        \item When $S_2$ is empty,
      \begin{align*}
        L_{a'} \inters \Inters_{a \in S_2} U_a \setminus \Union_{a \in A \setminus S_2} U_a
        &= L_{a'} \setminus \Union_{a \in A} U_a
        \subseteq L_{a'} \setminus  U_{a'}
        =\emptyset
      \end{align*}
        \item When $S_2 = \{a'\}$,
      \begin{align*}
        L_{a'} \inters \Inters_{a \in S_2} U_a \setminus \Union_{a \in A \setminus S_2} U_a
        &= L_{a'} \setminus \Union_{a \in A, a \neq a'} U_a .
      \end{align*}
        \item
      When $S_2 = \{a''\}$ for some $a'' \neq a'$
      \begin{align*}
        L_{a'} \inters \Inters_{a \in S_2} U_a \setminus \Union_{a \in A \setminus S_2} U_a
        &= L_{a'} \inters U_{a''} \setminus \Union_{a \in A, a \neq a''} U_a \\
        &= \emptyset
      \end{align*}
      \end{itemize}
      \end{itemize}

Thus,
      \begin{align*}
         \m{\prob'}(\event_1)
        = & \m{\prob'}\Big(\Union_{S_1 \in I_1, S_2 \in I_2} \Inters_{a \in S_1} L_a \inters \Inters_{a \in S_2} U_a \setminus (\Union_{a \in A \setminus S_1} L_a \union \Union_{a \in A \setminus S_2} U_a) \inters) \\
          = & \m{\prob'}\Big(\Union_{\{a'\} \in I_1, S_2 \in I_2} ( L_{a'} \inters \Inters_{a \in S_2} U_a \setminus \Union_{a \in A \setminus S_2} U_a) \Big) \\
          = & \m{\prob'}\Big(\Union_{\{a'\} \in I_1 \inters I_2} L_{a'} \inters U_{a'} \setminus \Union_{a \in A, a \neq a'} U_a \Big)  \\
          = & \m{\prob'}\Big(\Union_{\{a'\} \in I_1 \inters I_2} ( L_{a'} \setminus \Union_{a \in A, a \neq a'} U_a )  \Big) \\
          = & \m{\prob'}\Big(\Union_{\{a'\} \in I_1 \inters I_2} ( L_{a'} \setminus \Union_{a \in A, a \neq a'} (L_a \Union (U_a \setminus L_a)) )  \Big) \\
          = & \m{\prob'}\Big(\Union_{\{a'\} \in I_1 \inters I_2} ( L_{a'} \setminus \Union_{a \in A, a \neq a'} (L_a ) ) \Big) \\
          = & \m{\prob'}\Big(\Union_{\{a'\} \in I_1 \inters I_2} L_{a'} \Big)
      \end{align*}
      Denote $\Union_{\{a'\} \in I_1 \inters I_2} L_{a'}$
      as $\event'_1$.
      And $\event_1 \setminus \event'_1$ and $\event'_1 \setminus \event_1$ both have measure 0.

      Similar results hold for $\event_2$ as well, and we can show that
      \begin{align*}
        \m{\prob'}(\event_2)
         = & \m{\prob'}\Big(\Union_{\{b'\} \in I_3 \inters I_4} L_{b'} \Big)
      \end{align*}
      Denote $\Union_{\{b'\} \in I_3 \inters I_4} L_{b'}$
      as $\event'_2$.
      And $\event_2 \setminus \event'_2$ and $\event'_2 \setminus \event_2$ both have measure 0.


      Thus,
      \begin{align*}
         \m{\prob'}(\event_1 \inters \event_2)
        =& \m{\prob'}(\event_1 \inters \event_2 \inters \event'_1)
        + \m{\prob'}((\event_1 \inters \event_2) \setminus \event'_1)\\
        =& \m{\prob'}(\event_1 \inters \event_2 \inters \event'_1)
        + 0 \\
        =& \m{\prob'}(\event_1 \inters \event_2 \inters \event'_1 \inters \event'_2) + \m{\prob'}((\event_1 \inters \event_2 \inters \event'_1) \setminus \event'_2) + 0 \\
        =& \m{\prob'}(\event_1 \inters \event_2 \inters \event'_1 \inters \event'_2) + 0 + 0 \\
        =&  \m{\prob'}(\event_1 \inters \event_2 \inters \event'_1 \inters \event'_2) +  \m{\prob'}((\event_2 \inters \event'_1 \inters \event'_2 ) \setminus \event_1) \\
        =&  \m{\prob'}(\event_2 \inters \event'_1 \inters \event'_2)  \\
        =&  \m{\prob'}(\event_2 \inters \event'_1 \inters \event'_2) + \m{\prob'}((\event'_1 \inters \event'_2 ) \setminus \event_2) \\
        =&  \m{\prob'}(\event'_1 \inters \event'_2) \\
        =&  \m{\prob'}\left((\Union_{\{a'\} \in I_1 \inters I_2} L_{a'}) \inters (\Union_{\{b'\} \in I_3 \inters I_4} L_{b'})\right) \\
         =&  \m{\prob'}\left(\Union_{\{a'\} \in I_1 \inters I_2, \{b'\} \in I_3 \inters I_4} L_{a', b'}\right) \\
         =&  \sum_{\substack{\{a'\} \in I_1 \inters I_2 \\ \{b'\} \in I_3 \inters I_4}} \m{\prob'}(L_{a', b'})
       \end{align*}

Next we show that
       $         \m{\prob'}(i) (L_{a, b})
         = \m{\prob'}(i) (\event_1) \cdot \m{\prob'}(i) (\event_2)$.
         Note that
      $\m{\prob'}(L_a)  = \sum_{b} \m{\prob'}(L_{a,b}) = \m{\prob'}(\inv{\aexpr_1}(a))$,
      and
      $\m{\prob'}(L_b)  = \sum_{a} \m{\prob'}(L_{a,b}) = \m{\prob'}(\inv{\aexpr_2}(b))$.
      And $\mu_1 \pprod \mu_2 = \m{\prob'}(i) \circ \inv{(\aexpr_1, \aexpr_2)}$
      implies that
      \begin{align*}
        \m{\prob'}(i) (L_{a, b})
        &= \mu_1 \pprod \mu_2 (a, b)\\
        &=\mu_1(a) \cdot \mu_2(b)
      \end{align*}
      Then
      \begin{align*}
        \mu_1(a)
        &= \mu_1(a) \cdot \sum_{b \in A} \mu_2(b)\\
        &= \sum_{b \in A} \mu_1(a) \cdot \mu_2(b) \\
        &= \sum_{b \in A} \m{\prob'}(i) (L_{a, b}) \\
        &= \m{\prob'}(i) \left(\sum_{b \in A} L_{a, b}\right)  \\
        &= \m{\prob'}(i) (L_a),
      \end{align*}
      and similarly,
      \begin{align*}
        \mu_2(b)
        &= \left(\sum_{a \in A} \mu_1(a)\right) \cdot \mu_2(b)\\
        &= \sum_{a \in A} (\mu_1(a) \cdot \mu_2(b))\\
        &= \sum_{a \in A} \m{\prob'}(i) (L_{a, b}) \\
        &= \m{\prob'}(i) \left(\sum_{a \in A} L_{a, b}\right)  \\
        &= \m{\prob'}(i) (L_b).
      \end{align*}
      Thus,
      \begin{align*}
         \m{\prob'}(i) (L_{a, b})
         &=\mu_1(a) \cdot \mu_2(b)
         = \m{\prob'}(i) (L_a) \cdot \m{\prob'}(i) (L_b)
      \end{align*}



Therefore,
       \begin{align*}
         \m{\prob'}(\event_1 \inters \event_2)
         =&  \sum_{\substack{\{a'\} \in I_1 \inters I_2 \\ \{b'\} \in I_3 \inters I_4}} \m{\prob'}(L_{a', b'}) \\
         =&  \sum_{\substack{\{a'\} \in I_1 \inters I_2 \\ \{b'\} \in I_3 \inters I_4}} \m{\prob'}(L_{a'}) \cdot \m{\prob'}(L_{b'}) \\
       =&  \sum_{\mathclap{\{a'\} \in I_1 \inters I_2}}  \m{\prob'}(L_{a'}) \cdot \sum_{\mathclap{\{b'\} \in I_3 \inters I_4}} \m{\prob'}(L_{b'}) \\
       =& \m{\prob'}(\event_1) \cdot \m{\prob'}(\event_2)\\
       =& \m{\prob_1'}(\event_1) \cdot \m{\prob_2'}(\event_2)
      \end{align*}

  Thus we have
  $(\m{\salg}_1, \m{\prob'}_1) \iprod (\m{\salg}_2, \m{\prob'}_2) \extTo (\m{\salg'}, \m{\prob'})$.
  Let $\m{\permap_1} = \m{\permap_2} = \fun x. \m{\permap'}(x)/2$.

  Next we show that $\distAs{\aexpr_1}{\mu_1} (\m{\salg}_1, \m{\prob'}_1, \m{\permap_1}) $ and $\distAs{\aexpr_2}{\mu_2} (\m{\salg}_2, \m{\prob'}_2, \m{\permap_2})$.
  By definition,
  $\distAs{\aexpr_1}{\mu_1} (\m{\salg}_1, \m{\prob'}_1, \m{\permap_1})$
  is equivalent to
\begin{align*}
      \E \m{\salg''},\m{\prob''}.
      (\Own{\m{\salg''},\m{\prob''}})(\m{\salg}_1, \m{\prob'}_1, \m{\permap_1}) *
    \almostM{\aexpr_1}{(\m{\salg''}(i), \m{\prob''}(i))}
    \land
    \mu_1 = \m{\prob''}(i) \circ \inv{\aexpr_1}
    ,
  \end{align*}
which is equivalent to
\begin{multline*}
      \E \m{\salg''},\m{\prob''}.
      (\m{\salg''},\m{\prob''}) \raLeq (\m{\salg}_1, \m{\prob'}_1) *
      \bigl(\forall a \in A. \exists S_a, T_a \in \ \m{\salg''}(i).\\
      S_a \subseteq \inv{\aexpr_1}(a) \subseteq T_a  \land
      \m{\prob''}(i)(S_a) =  \m{\prob''}(i)(S_a) \land
      \mu_1(a) = \m{\prob''}(i)(S_a) = \m{\prob''}(i)(T_{a})
      \bigr)
  \end{multline*}
We can pick the existential witness to be
  $\m{\salg}_1, \m{\prob'}_1$.
  For any $a \in A$,
  $ \inv{\aexpr_1}(a) = \Union_{b \in A}\inv{(\aexpr_1, \aexpr_2)} (a, b)$.
  Because we have $L_{a, b} \subseteq \inv{(\aexpr_1, \aexpr_2)} (a, b) \subseteq U_{a,b}$,
  then
  \[
   \Union_{b \in A} L_{a, b} \subseteq
   \inv{\aexpr_1}(a) = \Union_{b \in A}\inv{(\aexpr_1, \aexpr_2)} (a, b)
   \subseteq \Union_{b \in A} U_{a, b} .
 \]
  By definition, for each $a$,
  $\Union_{b \in A} L_{a, b} \in \m{\salg}_1(i)$ and
  $\Union_{b \in A} U_{a, b} \in \m{\salg}_1(i)$,
  and we also have
  \begin{align*}
     \m{\prob'}_1(i) (\Union_{b \in A} L_{a, b})
     &= \sum_{b \in A} \m{\prob'}_1(i) (L_{a,b})\\
     &= \sum_{b \in A} \m{\prob'}_1(i) (U_{a,b})\\
     &= \m{\prob'}_1(i) \bigl(\Union_{b \in A} U_{a, b}\bigr)\\
     &= \mu_1(a)
  \end{align*}
  Thus, $S_a = \Union_{b \in A} L_{a, b}$ and
  $T_a = \Union_{b \in A} U_{a, b}$ witnesses the conditions needed
  for
  $\distAs{\aexpr_1}{\mu_1} (\m{\salg}_1, \m{\prob'}_1, \m{\permap_1}) $.
  And similarly, we have $\distAs{\aexpr_2}{\mu_2} (\m{\salg}_2, \m{\prob'}_2, \m{\permap_2}) $.
\end{proof}
 
\subsubsection{Soundness of Conditioning Rules}
\begin{lemma}
\label{proof:c-true}
  \Cref{rule:c-true} is sound.
\end{lemma}

\begin{proof}
  Let $\m{{\raUnit}} = (\m{\salg}_{\m{{\raUnit}}}, \m{\prob}_{\m{{\raUnit}}}, \m{\permap}_{\m{{\raUnit}}}) \in \Model_I$
  be the unit of $\Model_I$ and
  $\m{\krnl} = \fun v. \m{\prob}_{\m{{\raUnit}}}$.
  Then,
  \begin{eqexplain}
    \True
\whichproves*
    \Own{\m{\salg}_{\m{{\raUnit}}}, \m{\prob}_{\m{{\raUnit}}}}
\whichproves
    \Own{\m{\salg}_{\m{{\raUnit}}}, \m{\prob}_{\m{{\raUnit}}}} *
      \pure{
        \forall i\in I\st
          \m{\prob}_{\m{{\raUnit}}}(i) = \bind(\prob, \m{\krnl}(i))
      }
\whichproves
      \Own{\m{\salg}_{\m{{\raUnit}}}, \m{\prob}_{\m{{\raUnit}}}}
      * \pure{
        \forall i\in I\st
          \m{\prob}_{\m{{\raUnit}}}(i) = \bind(\prob, \m{\krnl}(i))
      }
      * \True
\whichproves
    \E \m{\salg}_{\m{{\raUnit}}}, \m{\prob}_{\m{{\raUnit}}}, \m{\krnl}.
      \Own{\m{\salg}_{\m{{\raUnit}}}, \m{\prob}_{\m{{\raUnit}}}}
      \begin{array}[t]{@{}>{{}}l}
      * \pure{
        \forall i\in I\st
          \m{\prob}_{\m{{\raUnit}}}(i) = \bind(\prob, \m{\krnl}(i))
      } \\
      * (
        \forall v \in \psupp(\prob).
         \Own{\m{\salg}_{\m{{\raUnit}}}, \m{\krnl}(I)(v), \m{\permap}_{\m{{\raUnit}}}} \wand \True
      )
      \end{array}
\whichproves
    \CMod{\prob} \wtv. \True
  \qedhere
  \end{eqexplain}
\end{proof} \begin{lemma}
\label{proof:c-false}
  \Cref{rule:c-false} is sound.
\end{lemma}

\begin{proof}
  Assume $a \in \Model_I$ is such that
  $\raValid(a)$ and that it satisfies
  $ \CC{\prob} v.\False $.
  By definition, this means that,
  for some $ \m{\salg}_0, \m{\prob}_0,\m{\permap}_0,$ and $ \m{\krnl}_0 $:
  \begin{gather}
    (\m{\salg}_0, \m{\prob}_0,\m{\permap}_0) \raLeq a
    \label{c-false:prob0}
    \\
    \forall i\in I\st
      \m{\prob}_0(i) = \bind(\prob, \m{\krnl}_0(i))
    \label{c-false:prob0-bind}
    \\
    \forall v \in \psupp(\prob) \st
      \False(\m{\salg}_0, \m{\krnl}_0(I)(v), \m{\permap}_0)
    \label{c-false:False-krnl0}
  \end{gather}
  Let $v_0 \in \psupp(\prob)$---we know one exists because $\prob$
  is a (discrete) probability distribution.
  Then by \eqref{c-false:False-krnl0} on~$v_0$
  we get $\False(\m{\salg}_0, \m{\krnl}_0(I)(v_0), \m{\permap}_0)$ holds.
  Since $\False(\wtv)$ is by definition false,
  we get $\False(a)$ holds \emph{ex falso}.
\end{proof} \begin{lemma}
\label{proof:c-cons}
  \Cref{rule:c-cons} is sound.
\end{lemma}

\begin{proof}
  Assume $a \in \Model_I$ is such that
  $\raValid(a)$ and that it satisfies
  $ \CC{\prob} v.K(v) $.
  By definition, this means that,
  for some $ \m{\salg}_0, \m{\prob}_0,\m{\permap}_0,$ and $ \m{\krnl}_0 $:
  \begin{gather}
    (\m{\salg}_0, \m{\prob}_0,\m{\permap}_0) \raLeq a
    \label{c-cons:prob0}
    \\
    \forall i\in I\st
      \m{\prob}_0(i) = \bind(\prob, \m{\krnl}_0(i))
    \label{c-cons:prob0-bind}
    \\
    \forall v \in \psupp(\prob) \st
      K(v)(\m{\salg}_0, \m{\krnl}_0(I)(v), \m{\permap}_0)
    \label{c-cons:K-krnl0}
  \end{gather}
  Then by the premise $\forall v\st K(v) \proves K'(v)$
  and \eqref{c-cons:K-krnl0} we obtain
  \begin{equation}
    \forall v \in \psupp(\prob) \st
      K'(v)(\m{\salg}_0, \m{\krnl}_0(I)(v), \m{\permap}_0)
    \label{c-cons:K'-krnl0}
  \end{equation}
  By
  \eqref{c-cons:prob0}, \eqref{c-cons:prob0-bind}, and \eqref{c-cons:K'-krnl0}
  we get $ \CC{\prob} v.K'(v) $ as desired.
\end{proof} \begin{lemma}
\label{proof:c-frame}
  \Cref{rule:c-frame} is sound.
\end{lemma}

\begin{proof}
  Assume $a \in \Model_I$ is such that
  $\raValid(a)$ and that it satisfies
  $ P * \CC{\prob} v.K(v) $.
  By definition, this means that
  there exist some
  $(\m{\salg}_1, \m{\prob}_1, \m{\permap}_1)$,
  $(\m{\salg}_2, \m{\prob}_2, \m{\permap}_2)$,
  and $\m{\krnl}$
  such that
  \begin{gather}
    (\m{\salg}_1, \m{\prob}_1, \m{\permap}_1)
    \raOp
    (\m{\salg}_2, \m{\prob}_2, \m{\permap}_2)
    \raLeq a
    \\
    P(\m{\salg}_1, \m{\prob}_1, \m{\permap}_1)
    \\
\forall i \in I.
      \m{\prob}_2(i) = \bind(\prob, \m{\krnl}(i))
    \\
    \forall v\in \psupp(\prob) \st
      K(v)(\m{\salg}_2, \m{\krnl}(I)(v), \m{\permap}_2)
  \end{gather}
Now let:
  \begin{align*}
    (\m{\salg}',\m{\prob}',\m{\permap}')
    &=
    (\m{\salg}_1(i), \m{\prob}_1(i)) \iprod (\m{\salg}_2(i), \m{\prob}_2(i))
  &
    \m{\krnl}'(i) &= \fun v. \m{\prob}_1(i) \iprod \m{\krnl}(i)(v)
  \end{align*}
  By~\cref{lemma:fibre-prod-exists}, for each~$i\in I$:
  \begin{align*}
    (\m{\salg}',\m{\prob}',\m{\permap}')
    &= (\m{\salg}_1(i), \m{\prob}_1(i)) \iprod (\m{\salg}_2(i), \m{\prob}_2(i))
    \\
    &= (\m{\salg}_1(i) \punion \m{\salg}_2(i),
       \bind(\prob, \fun v. \m{\prob}_1(i) \iprod \m{\krnl}(i)(v)))
   \tag*{(By \cref{lemma:fibre-prod-exists})}
    \\
    &= (\m{\salg}_1(i) \punion \m{\salg}_2(i),
       \bind(\prob, \m{\krnl}'(i)))
  \end{align*}
  Notice that $ \m{\krnl}'(I)(v) = \m{\prob}_1 \iprod \m{\krnl}(I)(v) $.
  Thus we obtain:
  \begin{gather}
    (\m{\salg}',\m{\prob}',\m{\permap}')
    \raLeq a
    \\
\forall i \in I.
      \m{\prob}'(i) = \bind(\prob, \m{\krnl}'(i))
    \\
    \intertext{and for all $v \in \psupp(\prob)$,}
(\m{\salg}_1, \m{\prob}_1, \m{\permap}_1)
      \iprod
    (\m{\salg}_2, \m{\krnl}(I)(v), \m{\permap}_2)
    =
    (\m{\salg}', \m{\prob}_1 \iprod \m{\krnl}(I)(v), \m{\permap}')
    \raLeq
    (\m{\salg}', \m{\krnl}'(I)(v),\m{\permap}')
    \\
    P(\m{\salg}_1, \m{\prob}_1, \m{\permap}_1)
    \\
    K(v)(\m{\salg}_2, \m{\krnl}(I)(v), \m{\permap}_2)
  \end{gather}
  which gives us that $a$ satisfies
  $ \CC{\prob} v.(P * K(v)) $ as desired.
\end{proof} \begin{lemma}
\label{proof:c-unit-l}
  \Cref{rule:c-unit-l} is sound.
\end{lemma}

\begin{proof}
  Straightforward.
\end{proof} \begin{lemma}
\label{proof:c-unit-r}
  \Cref{rule:c-unit-r} is sound.
\end{lemma}

\begin{proof}
  We prove the two directions separately.

  \begin{casesplit}
  \case*[Forward direction $\distAs{\aexpr\at{i}}{\prob} \proves \CC\prob v.\sure{\aexpr\at{i} = v}$]
    By unfolding the assumption $\distAs{\aexpr\at{i}}{\prob}$ we get
    that there exist $\m{\salg},\m{\prob}$ such that:
    \[
      \Own{\m{\salg},\m{\prob}}
      *
      \pure{
        \almostM{\aexpr}{(\m{\salg}(i),\m{\prob}(i))}
      }
      *
      \pure{
        \prob = \m{\prob}(i) \circ \inv{\aexpr}
      }
    \]
    holds.
    Let
    \begin{align*}
      \m{\krnl} &\is
      \fun j.
        \begin{cases}
          \fun v. \m{\prob}(j) \CASE j \ne i
          \\
          \fun v. \gamma_v     \CASE j=i
        \end{cases}
      &
      \gamma_v &\is
        \fun \event \of \m{\salg}(i).
          \frac{\m{\prob}(i)(\event \inters \inv{(\aexpr=v)})}
               {\m{\prob}(i)(\inv{(\aexpr=v)})}
    \end{align*}
    That is, $\m{\krnl}(j)$ maps every~$v$ to $\m{\prob}(j)$ when $i\ne j$,
    while when $i=j$ it maps~$v$ to the distribution $\m{\prob}(i)$ conditioned on $\aexpr=v$.
    Note that $\m{\krnl}$ is well defined because
    \begin{enumerate*}
      \item
        although the events
        $\event \inters \inv{(\aexpr=v)}$ and
        $\inv{(\aexpr=v)}$
        might not belong to $\m{\salg}(i)$,
        their probability is uniquely determined
        by almost measurability of $\aexpr$;
      \item
        we are only interested in the cases where~$v \in \psupp(\prob)$,
        which implies that the denominator is not zero:
        $\m{\prob}(i)(\inv{(\aexpr=v)}) = \prob(v) > 0$.
    \end{enumerate*}
    By construction we obtain that
    \begin{gather}
      \forall j \in I\st
        \m{\prob}(j) = \bind(\prob, \m{\krnl}(j))
      \label{c-unit-r:bind}
      \\
      \forall v\in \psupp(\prob)\st
        \m{\krnl}(i)(v)(\inv{(E=v)}) = 1
      \label{c-unit-r:prob1}
    \end{gather}
    From \eqref{c-unit-r:prob1} we get that
    $\sure{\aexpr\at{i} = v}$ holds on
    $(\m{\salg}(i), \m{\krnl}(i)(v), \m{\permap}(i))$,
    from which it follows that:
    \[
      \Own{\m{\salg}, \m{\krnl}(I)(v), \m{\permap}}
      \wand \sure{\aexpr\at{i} = v}
    \]
    Therefore we obtain
    \begin{align*}
      & \E \m{\salg},\m{\prob}, \m{\krnl}, \m{\permap}.
          \Own{\m{\salg},\m{\prob}, \m{\permap}} *
          \pure{\forall j \in I. \m{\prob}(j) = \bind(\prob, \m{\krnl}(j))} \\
      & \qquad \qquad  *
          (
            \forall v \in A_{\prob}.
              \Own{\m{\salg}, \m{\krnl}(I)(v), \m{\permap}}
              \wand \sure{\aexpr\at{i} = v}
          )
    \end{align*}
    which gives us $ \CC\prob v.\sure{\aexpr\at{i} = v} $
    by \cref{prop:cond-as-wand}.

\case*[Backward direction $\CC\prob v.\sure{\aexpr\at{i} = v} \proves \distAs{\aexpr\at{i}}{\prob}$]
    First note that
    \begin{align*}
      \sure{\aexpr\at{i} = v} &(\m{\salg}, \m{\krnl}(v), \m{\permap})
      \\
      &\iff
        \bigl(\distAs{((\aexpr = v) \in \true)\at{i}}{\delta_{\True}}\bigr)
          (\m{\salg}, \m{\krnl}(I)(v), \m{\permap})
      \\
      &\iff
        \almostM{((\aexpr = v) \in \true)}{(\m{\salg}(i), \m{\krnl}(i)(v))}
        \land
        \delta_{\True} =
          \m{\krnl}(i)(v) \circ \inv{((\aexpr = v) \in \true)}
      \\
      &\iff
        \almostM{((\aexpr = v) \in \true)}{(\m{\salg}(i), \m{\krnl}(i)(v))}
        \land
        \delta_{v} = \m{\krnl}(i)(v) \circ \inv{\aexpr}
    \end{align*}

for some $\m{\krnl}$.
    This implies
    $\pure{\almostM{E}{\m{\salg}(i), \m{\krnl}(i)(v)}}$.
    Then, for any value $v \in \psupp(\prob)$,
    \begin{align*}
      \m{\prob}(i) \circ \inv{\aexpr}(v)
      &=(\bind(\prob, \m{\krnl}(i) ) \circ \inv{\aexpr})(v)\\
      &=  \bind(\prob, \m{\krnl}(i) )  (\inv{\aexpr}(v)) \\
      &=  \sum_{\mathclap{v'\in\psupp(\prob)}}  \prob(v') \cdot \m{\krnl}(i)(v') (\inv{\aexpr}(v)) \\
      &=  \sum_{\mathclap{v'\in\psupp(\prob)}}  \prob(v') \cdot (\m{\krnl}(i)(v') \circ \inv{\aexpr}) (v) \\
      &=  \sum_{\mathclap{v'\in\psupp(\prob)}}  \prob(v') \cdot \dirac{v'} (v) \\
&=  \prob(v)
    \end{align*}
    This implies the pure facts that
    $ \almostM{\aexpr}{(\m{\salg}(i),\m{\prob}(i))}$ and
    $\prob = \m{\prob}(i) \circ \inv{\aexpr}$.
    Therefore:
    \begin{eqexplain}
      \CC\prob v. \sure{\aexpr\at{i} = v} \notag
      \whichproves*
      \E \m{\salg},\m{\prob}, \m{\krnl}, \m{\permap}.
          \Own{\m{\salg},\m{\prob}, \m{\permap}} *
          \pure{\forall j \in I. \m{\prob}(j) = \bind(\prob, \m{\krnl}(j))} \notag \\
          & \qquad \qquad  *
          (\forall v \in A_{\prob}.
          \Own{\m{\salg}, \m{\krnl}(I)(v), \m{\permap}}
          \wand
          \sure{\aexpr\at{i} = v}
          )
      \whichproves
      \E \m{\salg},\m{\prob}.
        \Own{\m{\salg},\m{\prob}} *
          \pure{
            \almostM{\aexpr}{(\m{\salg}(i),\m{\prob}(i))}}
            \ast
          \pure{
            \prob = \m{\prob}(i) \circ \inv{\aexpr}
        }
       \whichproves
       \distAs{\aexpr\at{i}}{\prob}
       \qedhere
    \end{eqexplain}
  \end{casesplit}
\end{proof}
 \begin{lemma}
\label{proof:c-assoc}
  \Cref{rule:c-assoc} is sound.
\end{lemma}

\begin{proof}
  Define $\krnl' = \fun v . \bind(\krnl(v), \fun w . \return(v, w))$.
  We start by rewriting the assumption $\CC{\prob} v.\CC{\krnl(v)} w.K(v,w)$ so that $k'$ is used and~$K$ depends only on the binding of the innermost modality:
  \begin{eqexplain}
    \CC{\prob} v.\CC{\krnl(v)} w.K(v,w)
    \whichproves*
    \CC{\prob} v.\CC{\krnl'(v)} (v',w).K(v,w)
    \byrules{c-transf,c-cons}
    \whichproves
    \CC{\prob} v.\CC{\krnl'(v)} (v',w).K(v',w)
    \byrules{c-pure,c-cons}
  \end{eqexplain}
  \Cref{rule:c-transf} is applied to the innermost modality
  by using the bijection $f_v(w) = (v,w)$.
  Then, since $(v',w) \in \psupp(k'(v)) \implies v=v'$,
  we can replace~$v'$ for~$v$ in~$K$.

  Our goal is now to prove:
  \[
    \CC{\prob} v.\CC{\krnl'(v)} (v',w).K(v',w)
    \proves
    \CC{\bind(\prob,\krnl')} (v',w).K(v',w)
  \]

  Let $a\in\Model_I$ be such that $ \raValid(a) $ and that it satisfies
  $ \CC{\prob} v.\CC{\krnl'(v)} (v',w).K(v',w). $
  From this assumption we know that,
  for some $ \m{\salg}_0, \m{\prob}_0,\m{\permap}_0,$ and $ \m{\krnl}_0 $:
  \begin{gather}
    (\m{\salg}_0, \m{\prob}_0,\m{\permap}_0) \raLeq a
    \label{c-assoc:prob0}
    \\
    \forall i\in I\st
      \m{\prob}_0(i) = \bind(\prob, \m{\krnl}_0(i))
    \label{c-assoc:prob0-bind}
  \end{gather}
  such that $\forall v \in \psupp(\prob)$,
  there are some
  $ \m{\salg}_1^v, \m{\prob}_1^v,\m{\permap}_1^v,$ and $ \m{\krnl}_1^v $
  satisfying:
  \begin{gather}
    (\m{\salg}_1^v, \m{\prob}_1^v,\m{\permap}_1^v)
    \raLeq
    (\m{\salg}_0, \m{\krnl}_0(I)(v),\m{\permap}_0)
    \label{c-assoc:prob1}
    \\
    \forall i\in I\st
      \m{\prob}_1^v(i) = \bind(\krnl'(v), \m{\krnl}_1^v(i))
    \label{c-assoc:prob1-bind}
    \\
    \forall (v',w) \in \psupp(\krnl'(v)) \st
      K(v',w)(\m{\salg}_1^v, \m{\krnl}_1^v(I)(v',w), \m{\permap}_1^v)
    \label{c-assoc:K-krnl1}
  \end{gather}

  Our goal is to prove
  $ \CC{\bind(\prob,\krnl')} (v',w).K(v',w) $ holds on $a$.
  To this end, we want to show that
  there exists $\m{\krnl}_2'$ such that:
\begin{gather}
    \forall i\in I\st
      \m{\prob}_0(i) = \bind(\bind(\prob,\krnl'), \m{\krnl}_2'(i))
    \label{c-assoc:goal1}
    \\
    \forall (v',w) \in \psupp(\bind(\prob,\krnl')) \st
      K(v', w)(\m{\salg}_0 , \m{\krnl}_2'(I) (v'), \m{\permap}_0)
    \label{c-assoc:goal2}
  \end{gather}

  Now let
  \[
    \m{\krnl}_2(i) = \fun (v', w). \m{\krnl}_1^{v'}(i)(v', w).
  \]
  which by construction and \cref{c-assoc:prob1-bind} gives us
  \[
    \m{\prob}_1^v(i)
    = \bind(\krnl'(v), \m{\krnl}_1^v(i))
    = \bind(\krnl'(v), \m{\krnl}_2(i))
  \]
  Therefore, by \cref{c-assoc:prob1}, we can apply \cref{lemma:bind-extend}
  and obtain that there exists a $\m{\krnl}_2'$ such that
  \begin{gather}
    \m{\krnl}_0(i)(v) = \bind(\krnl'(v), \m{\krnl}_2'(i))
    \label{c-assoc:k0-bind2}
    \\
    \bigl(\m{\salg}_0, \m{\krnl}_2'(i)(v',w)\bigr)
    \extOf
    \bigl(\m{\salg}_1^{v'}, \m{\krnl}_2(i)(v',w)\bigr)
    =
    \bigl(\m{\salg}_1^{v'},\m{\krnl}_1^{v'}(i)(v',w)\bigr)
    \label{c-assoc:kgeq}
  \end{gather}
  By \cref{c-assoc:prob0-bind,c-assoc:k0-bind2}
  we have:
  \begin{align*}
    \m{\prob}_0(i)
    &= \bind(\prob, \m{\krnl}_0(i)) \\
    &= \bind(\prob, \fun v.\bind(\krnl'(v), \m{\krnl}_2'(i)))
      &\text{By \eqref{prop:bind-assoc}}\\
    &= \bind(\bind(\prob, \krnl'), \m{\krnl}_2'(i))
  \end{align*}
  which proves \cref{c-assoc:goal1}.

  Finally, to prove \cref{c-assoc:goal2}, we can observe that
  $(v',w) \in \psupp(\bind(\prob,\krnl'))$ implies $v'\in \psupp(\prob)$;
  therefore, by \eqref{c-assoc:K-krnl1}, upward closure of $K(v',w)$, and
  \eqref{c-assoc:kgeq} and \eqref{c-assoc:prob1},
  we can conclude~$K(v',w)$ holds on
  $(\m{\salg}_0 , \m{\krnl}_2'(I) (v'), \m{\permap}_0)$,
  as desired.
\end{proof} \begin{lemma}
\label{proof:c-unassoc}
  \Cref{rule:c-unassoc} is sound.
\end{lemma}

\begin{proof}
  Assume $a \in \Model_I$ is such that
  $\raValid(a)$ and that it satisfies
  $ \CC{\bind(\prob,\krnl)} w.K(w) $.
  By definition, this means that,
  for some $ \m{\salg}_0, \m{\prob}_0,\m{\permap}_0,$ and $ \m{\krnl}_0 $:
  \begin{gather}
    (\m{\salg}_0, \m{\prob}_0,\m{\permap}_0) \raLeq a
    \label{c-unassoc:prob0}
    \\
    \forall i\in I\st
      \m{\prob}_0(i) = \bind(\bind(\prob, \krnl), \m{\krnl}_0(i))
    \label{c-unassoc:prob0-bind}
    \\
    \forall w \in \psupp(\bind(\prob, \krnl)) \st
      K(w)(\m{\salg}_0, \m{\krnl}_0(I)(w), \m{\permap}_0)
    \label{c-unassoc:K-krnl0}
  \end{gather}
  Our goal is to show that~$a$ satisfies
  $\CC\prob v. \CC{\krnl(v)} w.K(w)$,
  for which it would suffice to show that there is a $\m{\krnl}_1$
  such that:
  \begin{gather}
    \forall i\in I\st
      \m{\prob}_0(i) = \bind(\prob, \m{\krnl}_1(i))
    \label{c-unassoc:prob0-bind1}
    \\
    \intertext{
      and for all $v \in \psupp(\prob)$
      there is a $\m{\krnl}_2^v$ with
    }
    \forall i\in I\st
      \m{\krnl}_1(i)(v) = \bind(\krnl(v), \m{\krnl}_2^v(i))
    \label{c-unassoc:krnl1-bind}
    \\
    \forall w \in \psupp(\krnl(v)) \st
      K(w)(\m{\salg}_0, \m{\krnl}_2^v(I)(w), \m{\permap}_0)
    \label{c-unassoc:K-krnl1}
  \end{gather}

  To prove this we let
  \begin{align*}
    \m{\krnl}_1(i) &= \fun v.\bind(\krnl(v), \m{\krnl}_0(i))
    &
    \m{\krnl}_2^v(i) &= \m{\krnl}_0(i)
  \end{align*}

  By \eqref{prop:bind-assoc} we have
  \[
    \m{\prob}_0(i)
    = \bind(\bind(\prob, \krnl), \m{\krnl}_0(i))
    = \bind(\prob, \fun v.\bind(\krnl(v), \m{\krnl}_0(i)))
    = \bind(\prob, \m{\krnl}_1(i))
  \]
  which proves \eqref{c-unassoc:prob0-bind1}.
  By construction,
  \[
    \m{\krnl}_1(i)(v)
    = \bind(\krnl(v), \m{\krnl}_0(i))
    = \bind(\krnl(v), \m{\krnl}_2^v(i))
  \]
  proving \eqref{c-unassoc:krnl1-bind}.
  Finally,
  $v \in \psupp(\prob)$ and $w \in \psupp(\krnl(v))$
  imply $w \in \psupp(\bind(\prob, \krnl))$,
  so by \eqref{c-unassoc:K-krnl0} we proved
  \eqref{c-unassoc:K-krnl1}, concluding the proof.
\end{proof}
 \begin{lemma}
\label{proof:c-and}
  \Cref{rule:c-and} is sound.
\end{lemma}

\begin{proof}
  Let~$I_1 = \idx(K_1)$ and $I_2 = I \setminus I_1$;
  by $\idx(K_1) \inters \idx(K_2) = \emptyset$
  we have $I_2 \sups \idx(K_2)$.
  Assume $a\in \Model_I$ is such that~$\raValid(a)$ holds and
  that it satisfies $
    \CC{\prob} v. K_1(v)
      \land
    \CC{\prob} v. K_2(v)
  $.
  This means that
  for each $j \in \set{1,2}$,
  for some $ \m{\salg}_j, \m{\prob}_j,\m{\permap}_j,$ and $ \m{\krnl}_j $:
  \begin{gather}
    (\m{\salg}_j, \m{\prob}_j,\m{\permap}_j) \raLeq a
    \label{c-and:prob}
    \\
    \forall i\in I\st
      \m{\prob}_j(i) = \bind(\prob, \m{\krnl}_j(i))
    \label{c-and:prob-bind}
    \\
    \forall v \in \psupp(\prob) \st
      K_j(v)(\m{\salg}_j, \m{\krnl}_j(I)(v), \m{\permap}_j)
    \label{c-and:K-krnl}
  \end{gather}
  Now let
  \begin{align*}
    \hat{\m{\salg}} &=
    \begin{cases}
      \m{\salg}_1(i) \CASE i \in I_1 \\
      \m{\salg}_2(i) \CASE i \in I_2
    \end{cases}
    &
    \hat{\m{\prob}} &=
    \begin{cases}
      \m{\prob}_1(i) \CASE i \in I_1 \\
      \m{\prob}_2(i) \CASE i \in I_2
    \end{cases}
    &
    \hat{\m{\permap}} &=
    \begin{cases}
      \m{\permap}_1(i) \CASE i \in I_1 \\
      \m{\permap}_2(i) \CASE i \in I_2
    \end{cases}
    &
    \hat{\m{\krnl}}(i) &=
    \begin{cases}
      \m{\krnl}_1(i) \CASE i \in I_1 \\
      \m{\krnl}_2(i) \CASE i \in I_2
    \end{cases}
  \end{align*}
  By construction, we have:
  \begin{gather*}
    (\hat{\m{\salg}}, \hat{\m{\prob}},\hat{\m{\permap}}) \raLeq a
    \\
    \forall i\in I\st
      \hat{\m{\prob}}(i) = \bind(\prob, \hat{\m{\krnl}}(i))
  \end{gather*}
  Moreover, for any~$v \in \psupp(\prob)$ and any $j \in \set{1,2}$,
  since $I_j \sups \idx(K_j)$, condition \eqref{c-and:K-krnl} implies
  \[
    K_j(v)(\hat{\m{\salg}}, \hat{\m{\krnl}}(I)(v), \hat{\m{\permap}})
  \]
  This means
  $(\hat{\m{\salg}}, \hat{\m{\krnl}}(I)(v), \hat{\m{\permap}})$
  satisfies
  $(K_1(v) \land K_2(v))$,
  and thus~$a$ satisfies
  $ \CC\prob v. (K_1(v) \land K_2(v)) $,
  as desired.
\end{proof} \begin{lemma}
\label{proof:c-skolem}
  \Cref{rule:c-skolem} is sound.
\end{lemma}

\begin{proof}
  For any resource $r = (\m{\sigmaF}, \m{\mu}, \m{\permap})$,
  \begin{align*}
    &
    \left(\CC\prob v. \E x \of \Var. Q(v, x) \right) (\m{\sigmaF}, \m{\mu}, \m{\permap}) \\
    {}\iff {} &
    \exists \m{\krnl} \st
             \forall i \in I.
             \m{\mu}(i) = \bind(\prob, \m{\krnl}(i))
    {}\land{}
    \forall v\in  \psupp(\prob) \st
     (\E x \of X. Q(v, x))(\m{\sigmaF}, \m{\krnl}(I)(v), \m{\permap})
  \end{align*}

   For all $v\in \psupp(\prob)$,
   $\E x \of X. Q(v, x)$ holds on $(\m{\sigmaF}, \m{\krnl}(I)(v), \m{\permap})$.
   Thus,
   $Q(v, x_v)(\m{\sigmaF}, \m{\krnl}(I)(v), \m{\permap})$
   holds for some $x_v$.
   Then define $f: A \to \Var$ by letting $f(v) = x_v$ for $v \in \psupp(\mu)$.
   Then,
   \begin{align*}
   \exists \m{\krnl} \st
             \forall i \in I.
             \m{\mu}(i) = \bind(\prob, \m{\krnl}(i))
               {}\land{}
              \forall v\in  \psupp(\prob) \st
      Q(v, f(v))(\m{\sigmaF}, \m{\krnl}(I)(v), \m{\permap})
  \end{align*}
  And therefore $\m{\sigmaF}, \m{\mu}, \m{\permap}$
  satisfies ${\E f \of A \to \Var. \CC\prob v. Q(v, x)}$.
\end{proof} \begin{lemma}
\label{proof:c-transf}
  \Cref{rule:c-transf} is sound.
\end{lemma}

\begin{proof}
  For any resource $a = (\m{\sigmaF}, \m{\mu}, \m{\permap})$,
  if $ \model{\CC\prob v.K(v)}{(\m{\sigmaF}, \m{\mu}, \m{\permap})}$,
  then
  \begin{align*}
    \begin{array}[t]{@{}r@{\,}l@{}}
      \E \m{\krnl}.
      & (\m{\sigmaF}, \m{\mu}, \m{\permap}) \raLeq a
      \land
        \forall i\in I\st
        \m{\mu}(i) = \bind(\prob, \m{\krnl}(i))
      \\ & \land \;
        \forall v \in \psupp(\prob).
          \model{K(v)}{ (\m{\sigmaF}, \m{\krnl}(I)(v), \m{\permap}) }
    \end{array}
  \end{align*}

  $\m{\mu} = \bind(\prob, \m{\krnl})$  says that for any
  $E \in \m{\sigmaF}$,
  \begin{align}
    \m{\mu}(E)
    &= \sum_{\mathclap{v \in \psupp(\mu)}} \prob(v) \cdot \m{\krnl}(I)(v)(E) \notag \\
    &= \sum_{\mathclap{v \mid f(v) \in \psupp(\mu)}} \prob(f(v)) \cdot \m{\krnl}(I)(f(v))(E) \tag{Because $f$ is bijective}\\
    &= \sum_{\mathclap{v \in \psupp(\mu')}} \mu'(v) \cdot \m{\krnl}(I)(f(v))(E) \tag{Because $\mu'(v) = \mu(f(v))$} \\
    &= \bind(\prob', \fun v . \m{\krnl}(I)(f(v)))(E) \notag
  \end{align}
  Thus, $\m{\mu} = \bind(\prob', \fun v . \m{\krnl}(I)(f(v)))$.
  Furthermore, $\model{K(f(v))}{ (\m{\sigmaF}, \m{\krnl}(I)(f(v)), \m{\permap}) }$.

  Thus, if we denote $\fun v . \m{\krnl}(I)(f(v))$ as $\m{\krnl'}$, it satisfies
  \begin{align*}
    \begin{array}[t]{@{}r@{\,}l@{}}
      & (\m{\sigmaF}, \m{\mu}, \m{\permap}) \raLeq a
      \land
        \forall i\in I\st
        \m{\mu}(i) = \bind(\prob', \m{\krnl'}(i))
      \\ & \land \;
        \forall v \in \psupp(\prob).
          \model{K(v)}{ (\m{\sigmaF}, \m{\krnl'}(I)(v), \m{\permap})}
    \end{array}
  \end{align*}
  Thus, $ \model{\CC\prob' v.K(f(v))}{(\m{\sigmaF}, \m{\mu}, \m{\permap})}$.
\end{proof}
 \begin{lemma}
\label{proof:sure-str-convex}
  \Cref{rule:sure-str-convex} is sound.
\end{lemma}

\begin{proof}
  Assume $a \in \Model_I$ is a valid resource that
  satisfies $\CMod{\prob} v. (K(v) \ast  \sure{\aexpr\at{i}})$.
  Then, by definition, we know that,
  for some $ (\m{\salg}_0, \m{\prob}_0, \m{\permap}_0) $ and $\m{\krnl}_0$:
  \begin{gather}
    (\m{\salg}_0, \m{\prob}_0, \m{\permap}_0) \raLeq a
    \label{sure-str-convex:a}
    \\
    \forall i\in I\st
       \m{\prob}_0(i) = \bind(\prob, \m{\krnl}_0(i))
    \label{sure-str-convex:bind-a}
    \\
    \intertext{and, for all $v \in \psupp(\prob)$, there are
      $(\m{\salg}^v_1, \m{\prob}^v_1, \m{\permap}^v_1)$,
      $(\m{\salg}^v_2, \m{\prob}^v_2, \m{\permap}^v_2)$
      such that}
(\m{\salg}^v_1, \m{\prob}^v_1, \m{\permap}^v_1)
    \raOp
    (\m{\salg}^v_2, \m{\prob}^v_2, \m{\permap}^v_2)
    \raLeq
      (\m{\salg}_0, \m{\krnl}_0(I)(v), \m{\permap}_0)
    \label{sure-str-convex:star}
    \\
    K(v)(\m{\salg}^v_1, \m{\prob}^v_1, \m{\permap}^v_1)
    \label{sure-str-convex:K}
    \\
    \sure{\aexpr\at{i}}(\m{\salg}^v_2, \m{\prob}^v_2, \m{\permap}^v_2)
    \label{sure-str-convex:sure}
  \end{gather}
  From \eqref{sure-str-convex:sure} we know that for all~$v \in \psupp(\prob)$
  there are $L^v_1,L^v_0,U^v_1,U^v_0 \in \m{\salg}^v_2(i)$ such that:
  \begin{align*}
    L^v_0 &\subs \inv{\aexpr}(\False) \subs U^v_0
    &
    \m{\prob}^v_2(L^v_0) &= \m{\prob}^v_2(U^v_0) = 0
    \\
    L^v_1 &\subs \inv{\aexpr}(\True) \subs U^v_1
    &
    \m{\prob}^v_2(L^v_1) &= \m{\prob}^v_2(U^v_1) = 1
  \end{align*}
  Without loss of generality, all $L^v_0, L^v_1, U^v_0, U^v_1$ can be assumed
  to be only non-trivial on $\pvar(\aexpr)$.
  Consequently, we can also assume that
  $\m{\permap}^v_2(\ip{x}{j})<1$ for every $\ip{x}{j}$,
and in addition
  $\m{\permap}^v_2(\ip{x}{j})>0$
  if and only if $\p{x}\in \pvar{\aexpr}$ and $j=i$.
  From these components we can construct a new resource:
  \begin{align*}
    \m{\salg}_3(j) &\is
      \begin{cases}
        \sigcl*{\set*{
          \Inters_{v \in \psupp(\prob)} L^v_1,
          \Union_{v \in \psupp(\prob)} U^v_1
        }}
        \CASE j  =  i \\
        \set{\Store, \emptyset}
        \CASE j \ne i
      \end{cases}
    \\
    \m{\prob}_3 &\is \restr{\m{\prob}_0}{\m{\salg}_3}
    \\
    \m{\permap}_3 &\is
      \fun \ip{x}{j}.
      \begin{cases}
        \min
          \set{\m{\permap}^v_2(\ip{x}{i}) | v \in \psupp(\prob)}
        \CASE j=i \land \p{x} \in \pvar(\aexpr) \\
        0 \OTHERWISE
      \end{cases}
  \end{align*}
  By construction we obtain that
  $ \forall j\in I\st \m{\salg}_3(j) \subs \m{\salg}_0(j) $,
and that
  $\raValid(\m{\salg}_3, \m{\prob}_3, \m{\permap}_3)$.
Now letting
  $
   \m{\permap}_1' = {\m{\permap}_0-\m{\permap}_3}
  $,
  we obtain a valid resource
  $(\m{\salg}_0, \m{\prob}_0, \m{\permap}_1')$.


  Moreover,
  we have
  $
    \m{\salg}_0 = \m{\salg}_0 \punion \m{\salg}_3
  $
  and
  $
    \forall j\in I\st
    \forall\event\in\m{\salg}_3(j)\st
      \m{\prob}_3(\event)\in\set{0,1}
  $,
  which means that
  for any $X \in \m{\salg}_3$ and $Y \in \m{\salg}_0$,
  $\m{\prob}_3(X) \cdot \m{\prob}_0(Y) = \m{\prob}_0(X \cap Y)$.
  Then, by \eqref{sure-str-convex:bind-a}:
  \[
    (\m{\salg}_0, \bind(\prob, \m{\krnl}_0), \m{\permap}_1')
    \iprod
    (\m{\salg}_3, \m{\prob}_3, \m{\permap}_3)
    \raLeq
    (\m{\salg}_0, \m{\prob}_0, \m{\permap}_0) = a
  \]
  To close the proof it would then suffice to show that
  $ \CMod{\prob} v. K(v) $
  holds on
  $(\m{\salg}_0, \bind(\prob, \m{\krnl}_0), \m{\permap}_1')$
  and that
  $ \sure{\aexpr\at{i}} $
  holds on
  $ (\m{\salg}_3(j),\m{\prob}_3,\m{\permap}_3) $.
  The latter is obvious.
  The former follows from
  the fact that $ \restr{\m{\krnl}_0(j)(v)}{\m{\salg}^v_1} = \m{\prob}^v_1(j) $;
  by upward-closure and \eqref{sure-str-convex:K}
  this means that, for all $v \in \psupp(\prob)$:
  \[
    K(v)(\m{\salg}^v_1, \m{\prob}^v_1, \m{\permap}^v_1)
    \implies
    K(v)(\m{\salg}_0, \m{\krnl}_0(I)(v), \m{\permap}_1')
  \]
  which proves our claim.
\end{proof}
 \begin{lemma}
\label{proof:c-for-all}
  \Cref{rule:c-for-all} is sound.
\end{lemma}

\begin{proof}
  By unfolding the definitions,
\begin{align*}
      &\CMod\prob \m{v}. \forall x: X. Q(\m{v}) \\
    \iff &
    \begin{array}[t]{r@{\,}l}
    \E \m{\sigmaF}, \m{\mu}_0, \m{\krnl}.
      &\Own{(\m{\sigmaF}, \m{\mu}_0)} *
      \pure{
        \forall i\in I\st
          \m{\mu}_0(i) = \bind(\prob, \m{\krnl}(i))
      }\\ & * \; (
        \forall a \in A_{\prob}.
\Own{(\m{\sigmaF}, \m[i: \m{\krnl}(i)(a) | i \in I])} \wand \forall x: X. Q(\m{v})
          )
    \end{array}
    \\
    \implies &
    \begin{array}[t]{r@{\,}l}
    \forall x: X.
    \E \m{\sigmaF}, \m{\mu}_0, \m{\krnl}.
      &\Own{(\m{\sigmaF}, \m{\mu}_0)} *
      \pure{
        \forall i\in I\st
          \m{\mu}_0(i) = \bind(\prob, \m{\krnl}(i))
      }\\ & * \; (
        \forall a \in A_{\prob}.
\Own{(\m{\sigmaF}, \m[i: \m{\krnl}(i)(a) | i \in I])} \wand Q(\m{v})
          )
    \end{array}\\
    \iff & \forall x: X. \CMod\prob \m{v}. Q(\m{v})
  \end{align*}
\end{proof} \begin{lemma}
\label{proof:c-pure}
  \Cref{rule:c-pure} is sound.
\end{lemma}

\begin{proof}
  We first prove the forward direction:
  For any $a \in \Model_I$,
  if $\model{\pure{\mu(X) = 1} \ast \CMod{\mu}. K(v)}{(a)}$,
  then there exists some
  $\m{\salg}_0, \m{\prob}_0,\m{\permap}_0,$ and $ \m{\krnl}_0 $:
  \begin{gather*}
    (\m{\salg}_0, \m{\prob}_0,\m{\permap}_0) \raLeq a
    \\
    \forall i\in I\st
      \m{\prob}_0(i) = \bind(\prob, \m{\krnl}_0(i)) \\
      \forall v \in \psupp(\prob) \st
      \model{K(v)}{(\m{\salg}_0, \m{\krnl}_0(I)(v), \m{\permap}_0)}
  \end{gather*}

   The pure fact $\pure{\mu(X) = 1}$ implies that
   $X \supseteq \psupp(\mu)$ , and thus
   for every $v \in \psupp(\mu)$, $\pure{v \in X}$.
   Therefore, $\model{K(v)}{(\m{\salg}_0, \m{\krnl}_0(I)(v), \m{\permap}_0)}$,
   which witnesses that
   $\model{\CMod{\mu}. \pure{v \in X} \ast K(v)}{(a)}$.

  We then prove the backward direction:
  if $\CMod{\mu}. \pure{v \in X} \ast K(v) $,
  then there exists
    $\m{\salg}_0, \m{\prob}_0,\m{\permap}_0,$ and~$ \m{\krnl}_0 $:
  \begin{gather*}
    (\m{\salg}_0, \m{\prob}_0,\m{\permap}_0) \raLeq a
    \\
    \forall i\in I\st
      \m{\prob}_0(i) = \bind(\prob, \m{\krnl}_0(i)) \\
      \forall v \in \psupp(\prob) \st
      \model{\pure{v \in X} \ast K(v)}{(\m{\salg}_0, \m{\krnl}_0(I)(v), \m{\permap}_0)}
  \end{gather*}
  Then it must $X \supseteq \psupp(\mu)$,
  which implies that $\pure{\mu(X) = 1}$.
  Meanwhile, $\pure{v \in X} \ast K(v)$ holding on ${(\m{\salg}_0, \m{\krnl}_0(I)(v), \m{\permap}_0)}$
  implies that $K(v)$ holds on
  ${(\m{\salg}_0, \m{\krnl}_0(I)(v), \m{\permap}_0)}$
  Therefore, $\pure{\mu(X) = 1} \ast \CMod{\mu}. K(v)$ holds on $a$.
\end{proof}
