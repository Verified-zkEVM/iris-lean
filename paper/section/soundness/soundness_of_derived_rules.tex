\subsection{Soundness of Derived Rules}
\label{sec:appendix:derived-rules}


In this section we provide derivations for the rules we claim
are derivable in \thelogic.


\subsubsection{Ownership and Distributions}
\begin{lemma}
\label{proof:sure-dirac}
  \Cref{rule:sure-dirac} is sound.
\end{lemma}

\begin{proof}
  \begin{eqexplain}
    \distAs{E\at{i}}{\delta_v}
\whichisequiv*
    \E \m{\sigmaF}, \m{\prob}.
      \Own{(\m{\sigmaF}, \m{\prob})}
      *
      \pure{\m{\prob} \circ \inv{E\at{i}} = \dirac{v}}
\whichisequiv
    \E \m{\sigmaF}, \m{\prob}.
      \Own{(\m{\sigmaF}, \m{\prob})}
      *
      \pure{\m{\prob} \circ \inv{(E\at{i} = v)} = \dirac{\True}}
\whichisequiv
    \sure{E\at{i} = v}
\qedhere
  \end{eqexplain}
\end{proof} \begin{lemma}
\label{proof:sure-eq-inj}
  \Cref{rule:sure-eq-inj} is sound.
\end{lemma}

\begin{proof}
\begin{eqexplain}
  \sure{\aexpr\at{i} = v}
  *
  \sure{\aexpr\at{i} = v'}
\whichproves*
  \distAs{\aexpr\at{i}}{\dirac{v}}
  *
  \distAs{\aexpr\at{i}}{\dirac{v'}}
  \byrule{sure-dirac}
\whichproves
  \distAs{\aexpr\at{i}}{\dirac{v}}
  \land
  \distAs{\aexpr\at{i}}{\dirac{v'}}
\whichproves
  \pure{\dirac{v}=\dirac{v'}}
  \byrule{dist-inj}
\whichproves
  \pure{v=v'}
\qedhere
\end{eqexplain}
\end{proof} \begin{lemma}
\label{proof:sure-sub}
  \Cref{rule:sure-sub} is sound.
\end{lemma}

\begin{proof}
  \begin{eqexplain}
\distAs{\aexpr_1\at{i}}{\prob}
    *
    \sure{(\aexpr_2 = f(\aexpr_1))\at{i}}
  \whichproves*
    \CC\prob v.
      \sure{\aexpr_1\at{i} = v}
      *
      \sure{(\aexpr_2 = f(\aexpr_1))\at{i}}
  \byrules{c-unit-r,c-frame}
\whichproves
    \CC\prob v.
      \sure{\aexpr_1\at{i} = v \land \aexpr_2\at{i} = f(\aexpr_1\at{i})}
  \byrule{sure-merge}
\whichproves
    \CC\prob v. \sure{\aexpr_2\at{i} = f(v)}
  \byrule{c-cons}
\whichproves
    \CC\prob v. \CC{\dirac{f(v)}} \pr{v}.\sure{\aexpr_2\at{i} = \pr{v}}
  \byrule{c-unit-l}
\whichproves
    \CC{\pr{\prob}} \pr{v}.\sure{\aexpr_2\at{i} = \pr{v}}
  \byrules{c-assoc,c-sure-proj}
  \end{eqexplain}
  where $\pr{\prob} = \bind(\prob, \fun x. \dirac{f(x)}) = \prob \circ \inv{f}$.
  By \ref{rule:c-unit-r} we thus get
  $\distAs{\aexpr_2\at{i}}{\prob \circ \inv{f}}$.
\end{proof} \begin{lemma}
\label{proof:dist-fun}
  \Cref{rule:dist-fun} is sound.
\end{lemma}

\begin{proof}
  Assume $E\from \Store \to A$ and $ f \from A \to B $, then:
  \begin{eqexplain}
    \distAs{\aexpr\at{i}}{\prob}
\whichproves*
      \CC\prob v. \sure{(\aexpr = v)\at{i}}
    \byrules{c-unit-r}
\whichproves
      \CC\prob v. \sure{(f\circ\aexpr)\at{i} = f(v)}
    \byrules{c-cons}
\whichproves
      \CC\prob v. \CC{\dirac{f(v)}} \pr{v}.\sure{(f\circ\aexpr)\at{i} = \pr{v}}
    \byrule{c-unit-l}
\whichproves
      \CC{\pr{\prob}} \pr{v}.\sure{(f\circ\aexpr)\at{i} = \pr{v}}
    \byrules{c-assoc,c-sure-proj}
  \end{eqexplain}
  where $\pr{\prob} = \bind(\prob, \fun x. \dirac{f(x)}) = \prob \circ \inv{f}$.
  By \ref{rule:c-unit-r} we thus get
  $\distAs{(f\circ\aexpr)\at{i}}{\prob \circ \inv{f}}$.
\end{proof} \begin{lemma}
\label{proof:dirac-dup}
  \Cref{rule:dirac-dup} is sound.
\end{lemma}

\begin{proof}
  \begin{eqexplain}
    \distAs{E\at{i}}{\dirac{v}}
    \whichproves*
    \sure{E\at{i} = v}
    \byrule{sure-dirac}
    \whichproves
    \sure{E\at{i} = v} \ast \sure{E\at{i} = v}
    \byrule{sure-merge}
    \whichproves
    \distAs{E\at{i}}{\dirac{v}} * \distAs{E\at{i}}{\dirac{v}}
    \byrule{sure-dirac}
  \end{eqexplain}
\end{proof} \begin{lemma}
\label{proof:dist-supp}
  \Cref{rule:dist-supp} is sound.
\end{lemma}

\begin{proof}
  \begin{eqexplain}
    \distAs{E\at{i}}{\mu}
    \whichproves*
    \CMod{\mu} v. \sure{E\at{i} = v}
    \byrule{c-unit-r}
    \whichproves
    \pure{\mu(\psupp(\mu)) = 1} \ast
    \CMod{\mu} v. \sure{E\at{i} = v}
    \whichproves
    \CMod{\mu} v. \big(\pure{v \in \psupp(\mu)} \ast \sure{E\at{i} = v}\big)
    \byrule{c-pure}
    \whichproves
    \CMod{\mu} v. \big(\sure{E\at{i} = v} \ast \sure{E\at{i} \in \psupp(\mu)} \big)
\whichproves
    \big(\CMod{\mu} v.\sure{E\at{i} = v}\big)
    \ast \sure{E\at{i} \in \psupp(\mu)}
    \byrule{sure-str-convex}
    \whichproves
    \distAs{E\at{i}}{\mu}
    \ast \sure{E\at{i} \in \psupp(\mu)}
    \byrule{c-unit-r}
  \end{eqexplain}
\end{proof} \begin{lemma}
\label{proof:prod-unsplit}
  \Cref{rule:prod-unsplit} is sound.
\end{lemma}

\begin{proof}
  \begin{eqexplain}
    \distAs{\aexpr_1\at{i}}{\prob_1} *
    \distAs{\aexpr_2\at{i}}{\prob_2}
    \whichproves*
    \CC{\prob_1} v_1.
    \CC{\prob_2} v_2.
    \bigl(
      \sure{\aexpr_1\at{i} = v_1} *
      \sure{\aexpr_2\at{i} = v_2}
    \bigr)
    \byrules{c-unit-r,c-frame}
\whichproves
    \CC{\prob_1} v_1.
    \CC{\prob_2} v_2.
      \sure{(\aexpr_1, \aexpr_2)\at{i} = (v_1, v_2)}
    \byrules{sure-merge}
\whichproves
    \CC{\prob_1 \pprod \prob_2} (v_1,v_2).
      \sure{(\aexpr_1, \aexpr_2)\at{i} = (v_1, v_2)}
    \byrules{c-assoc}
\whichproves
    \distAs{(\aexpr_1\at{i}, \aexpr_2\at{i})}{\prob_1 \otimes \prob_2}
    \byrule{c-unit-r}
  \end{eqexplain}
\end{proof}
 
\subsubsection{\Supercond}
\begin{lemma}
\label{proof:c-fuse}
  \Cref{rule:c-fuse} is sound.
\end{lemma}

\begin{proof}
  Recall that
  $
    \prob \fuse \krnl
    \is
    \fun(v,w). \prob(v)\krnl(v)(w).
  $
  which can be reformulated as
  $
    \prob \fuse \krnl =
    \bind(\prob,\fun v.(\bind(\krnl(v), \fun w.\return(v,w)))).
  $

  The $(\proves)$ direction is an instance of \ref{rule:c-assoc}.

  The $(\provedby)$ direction follows from \ref{rule:c-unassoc}:
  \begin{eqexplain}
    \CC{\prob \fuse \krnl} (v',w'). K(v',w')
\whichproves*
    \CC \prob v.
      \CC{\bind(\krnl(v), \fun w.\dirac{(v,w)})} (v',w'). K(v',w')
    \byrule{c-unassoc}
\whichproves
    \CC \prob v.
      \CC{\krnl(v)} w.
        \CC {\dirac{(v,w)}} (v',w'). K(v',w')
    \byrule{c-unassoc}
\whichproves
    \CC{\prob} v.
    \CC{\krnl(v)} w.
      K(v,w)
    \byrule{c-unit-l}
  \end{eqexplain}
\end{proof}
 \begin{lemma}
\label{proof:c-swap}
  \Cref{rule:c-swap} is sound.
\end{lemma}

\begin{proof}
  \begin{eqexplain}
    \CC{\prob_1} v_1.
      \CC{\prob_2} v_2.
        K(v_1, v_2)
\whichproves*
      \CC{\prob_1 \pprod \prob_2}
        (v_1,v_2).
          K(v_1, v_2)
    \byrule{c-fuse}
\whichproves
    \CC{\prob_2} v_2.
      \CC{\prob_1} v_1.
          K(v_1, v_2)
    \byrule{c-fuse}
  \end{eqexplain}
  Where
  \[
    \prob_1 \pprod \prob_2
    =
    \prob_1 \fuse (\fun \wtv.\prob_2)
=
    \prob_2 \fuse (\fun \wtv.\prob_1)
\]
  justifies the applications of \ref{rule:c-fuse}.
\end{proof}
 \begin{lemma}
\label{proof:sure-convex}
  \Cref{rule:sure-convex} is sound.
\end{lemma}

\begin{proof}
  By \ref{rule:sure-str-convex} with $K = \True$.
\end{proof}
 \begin{lemma}
\label{proof:dist-convex}
  \Cref{rule:dist-convex} is sound.
\end{lemma}

\begin{proof}
  \begin{eqexplain}
    \CMod{\prob} v. \distAs{E\at{i}}{\mu'}
    \whichproves*
    \CMod{\mu} v.  \CMod{\mu'} w. \sure{E\at{i} = w}
    \byrule{c-unit-r}
    \whichproves
    \CMod{\mu'} w. \CMod{\mu} v.  \sure{E\at{i} = w}
    \byrule{c-swap}
    \whichproves
    \CMod{\mu'} w.  \sure{E\at{i} = w}
    \byrule{sure-convex}
    \whichproves
    \distAs{E\at{i}}{\mu'}
    \byrule{c-unit-r}
  \end{eqexplain}
\end{proof}
 \begin{lemma}
\label{proof:c-proj}
  The following rule is sound:
  \begin{proofrule}
  \infer{
    \forall (v,\wtv)\in\psupp(\prob).
    \forall \prob'.
    \CC{\prob'} w. P(v) \proves P(v)
  }{
    \CC \prob (v,w). P(v) \lequiv
    \CC {\prob\circ\inv{\proj}} v. P(v)
  }
  \end{proofrule}
\end{lemma}
\begin{proof}
  Assume that for all $(v,\wtv)\in\psupp(\prob)$,
  $\forall \prob'. \CC{\prob'} w. P(v) \proves P(v)$
  (\ie $P(v)$ is convex).
  By \cref{lm:fuse-split} there is some~$\krnl$ such that
  $ \prob=(\prob\circ\inv{\proj})\fuse\krnl $.
  Then:
  \begin{eqexplain}
    \CC \prob (v,w). P(v)
    \whichisequiv*
    \CC {\prob\circ\inv{\proj}} v.
    \CC {\krnl(v)} w. P(v)
    \byrule{c-fuse}
    \whichisequiv
    \CC {\prob\circ\inv{\proj}} v. P(v)
  \end{eqexplain}
  The last step is justified by the convexity assumption in the~$(\proves)$
  direction,
  and by \ref{rule:c-true} and \ref{rule:c-frame} in the~$(\provedby)$ direction.
\end{proof}



\begin{lemma}
\label{proof:c-sure-proj}
  \Cref{rule:c-sure-proj} is sound.
\end{lemma}
\begin{proof}
  By \cref{proof:c-proj} and \cref{proof:sure-convex}.
\end{proof}
 \begin{lemma}
\label{proof:c-sure-proj-many}
  \Cref{rule:c-sure-proj-many} is sound.
\end{lemma}

\begin{proof}
  Let $
    X_i \is \set{ \p{x} | \ip{x}{i} \in X }
  $ for every $i\in I$.
  Then:
  \begin{eqexplain}
    \CC\prob (\m{v}, w).
      \sure{\ip{x}{i}=\m{v}(\ip{x}{i})}_{\ip{x}{i}\in X}
\whichisequiv*
    \CC{\prob} (\m{v}, w).
      \LAnd_{i\in I}
        \sure{\LAnd_{\p{x}\in X_i} \ip{x}{i}=\m{v}(\ip{x}{i})}
\whichisequiv
    \LAnd_{i\in I}
      \CC{\prob} (\m{v}, w).
        \sure{\LAnd_{\p{x}\in X_i} \ip{x}{i}=\m{v}(\ip{x}{i})}
    \byrule{c-and}
\whichisequiv
    \LAnd_{i\in I}
      \CC{\prob\circ\inv{\proj_1}} \m{v}.
        \sure{\LAnd_{\p{x}\in X_i} \ip{x}{i}=\m{v}(\ip{x}{i})}
    \byrule{c-sure-proj}
\whichisequiv
    \CC{\prob\circ\inv{\proj_1}} \m{v}.
      \LAnd_{i\in I}
        \sure{\LAnd_{\p{x}\in X_i} \ip{x}{i}=\m{v}(\ip{x}{i})}
    \byrule{c-and}
\whichisequiv
    \CC{\prob\circ\inv{\proj_1}} \m{v}.
      \sure{\ip{x}{i}=\m{v}(\ip{x}{i})}_{\ip{x}{i}\in X}
  \end{eqexplain}
  Note that the (iterated) applications of \ref{rule:c-and}
  satisfy the side condition
  because the inner assertions are by construction on disjoint indices.
  The backward direction of \ref{rule:c-and} holds by
  the standard laws of conjunction.
\end{proof}
 \begin{lemma}
\label{proof:c-extract}
  \Cref{rule:c-extract} is sound.
\end{lemma}

\begin{proof}
  \begin{eqexplain}
    \CC{\prob_1} v_1. \bigl(
      \sure{\aexpr_1\at{i} = v_1} *
      \distAs{\aexpr_2\at{i}}{\prob_2}
    \bigr)
\whichproves*
      \CC{\prob_1} v_1.
        \bigl(
          \sure{\aexpr_1\at{i}=v_1} *
          \CC{\prob_2} v_2. \sure{\aexpr_2\at{i}=v_2}
        \bigr)
      \byrule{c-unit-r}
    \whichproves
      \CC{\prob_1} v_1.
      \CC{\prob_2} v_2.
        \bigl(
          \sure{\aexpr_1\at{i}=v_1} *
          \sure{\aexpr_2\at{i}=v_2}
        \bigr)
      \byrules{c-frame}
    \whichproves
      \CC{\prob_1} v_1.
      \CC{\prob_2} v_2.
        \sure{\aexpr_1\at{i}=v_1 \land \aexpr_2\at{i}=v_2}
      \byrules{sure-merge}
    \whichproves
      \CC{\prob_1 \pprod \prob_2} (v_1,v_2).
        \sure{(\aexpr_1\at{i},\aexpr_2\at{i})=(v_1,v_2)}
      \byrules{c-assoc}
    \whichproves
      \distAs{(\aexpr_1\at{i},\aexpr_2\at{i})}{(\prob_1 \pprod \prob_2)}
      \byrule{c-unit-r}
    \whichproves
      \distAs{\aexpr_1\at{i}}{\prob_1} *
      \distAs{\aexpr_2\at{i}}{\prob_2}
      \byrule{prod-split}
  \end{eqexplain}
\end{proof}
 \begin{lemma}
\label{proof:c-dist-proj}
  \Cref{rule:c-dist-proj} is sound.
\end{lemma}

\begin{proof}
  By \cref{proof:c-proj} and \cref{proof:dist-convex}.
\end{proof}
 
\subsubsection{Relational Lifting}
\begin{lemma}
\label{proof:rl-cons}
  \Cref{rule:rl-cons} is sound.
\end{lemma}

\begin{proof}
  \begin{eqexplain}
    \cpl{R_1}
\whichis*
      \E \prob.
        \pure{\prob(R_1) = 1} *
        \CC\prob \m{v}.
          \sure{\ip{x}{i} = \m{v}(\ip{x}{i})}_{\ip{x}{i}\in X}
\whichproves
      \E \prob.
        \pure{\prob(R_2) = 1} *
        \CC\prob \m{v}.
          \sure{\ip{x}{i} = \m{v}(\ip{x}{i})}_{\ip{x}{i}\in X}
    \by{$R_1 \subseteq R_2$}
\whichis \cpl{R_2}
  \qedhere
  \end{eqexplain}
\end{proof} \begin{lemma}
\label{proof:rl-unary}
  \Cref{rule:rl-unary} is sound.
\end{lemma}

\begin{proof}
  \begin{eqexplain}
    \cpl{R}
\whichis*
    \E \prob.
      \pure{\prob(R) = 1} *
      \CC\prob \m{v}.
        \sure{\ip{x}{i} = \m{v}(\p{x}{i})}_{\p{x}{i}\in X}
\whichproves
    \E \prob.
      \CC\prob \m{v}.
        \pure{\m{v} \in R}
        * \sure{\ip{x}{i} = \m{v}(\p{x}{i})}_{\p{x}{i}\in X}
    \byrule{c-pure}
\whichproves
    \E \prob.
      \CC\prob \m{v}.
        \sure{R(\p{x}_1\at{i}, \dots , \p{x}_n\at{i})}
\whichproves
    \E \prob.
      \sure{R(\p{x}_1\at{i}, \dots, \p{x}_n\at{i})}
    \byrule{sure-convex}
\whichproves
    \sure{R(\p{x}_1\at{i}, \dots, \p{x}_n\at{i})}
  \qedhere
  \end{eqexplain}
\end{proof}

%
 \begin{lemma}
\label{proof:rl-eq-dist}
  \Cref{rule:rl-eq-dist} is sound.
\end{lemma}

\begin{proof}
  \begin{eqexplain}
    \cpl{x\at{i} = y\at{j}}
\proves{}&
    \E \prob'.
     \CC{\prob'} (v_1,v_2).\bigl(
      \sure{\Ip{x}{i} = v_1} \land
      \sure{\Ip{y}{j} = v_2} \land
      \pure{v_1=v_2}
    \bigr)
\whichproves
    \E \prob'.
     \CC{\prob'} (v_1, v_2).\bigl(
      \sure{\Ip{x}{i} = v_1} \land
      \sure{\Ip{y}{j} = v_1 }
    \bigr)
    \byrule{c-cons}
\whichproves
    \E \prob'.
     \CC{\prob'\circ\inv{\proj_1}} v_1.\bigl(
      \sure{\Ip{x}{i} = v_1} \land
      \sure{\Ip{y}{j} = v_1 }
    \bigr)
    \byrule{c-sure-proj}
\whichproves
    \E \prob.
     \CC{\prob} v_1.\bigl(
      \sure{\Ip{x}{i} = v_1} \land
      \sure{\Ip{y}{j} = v_1 }
    \bigr)
    \by{$ \prob = \prob'\circ\inv{\proj_1} $}
\whichproves
    \E \prob.
      \bigl(
      \CC\prob v_1.
       \sure{\Ip{x}{i} = v_1}
      \bigr)
      \land
      \bigl(
      \CC\prob v_1.
        \sure{\Ip{y}{j} = v_1}
      \bigr)
\whichproves
    \E \prob.
    \distAs{\Ip{x}{i}}{\mu}
    \land
    \distAs{\Ip{y}{j}}{\mu}
    \byrule{c-unit-r}
\whichproves
    \E \prob.
    \distAs{\Ip{x}{i}}{\mu}
    \ast
    \distAs{\Ip{y}{j}}{\mu}
    \byrule{and-to-star}
  \end{eqexplain}
\end{proof} \begin{lemma}
  \Cref{rule:rl-convex} is sound.
\end{lemma}

\begin{proof}
 \begin{eqexplain}
   \CMod{\prob} a \st \cpl{R}
\whichis*
\CMod{\prob} a \st
     \E \prob'.
       \pure{\prob'(R) = 1} *
       \bigl(
         \CC{\prob'} \m{v}.
           \sure{\ip{x}{i} = \m{v}(\ip{x}{i})}_{\ip{x}{i}\in X}
       \bigr)
\whichproves
     \E \krnl.
     \CMod{\prob} a \st
       \bigl(
         \CC{\krnl(a)} \m{v}.
           \sure{\ip{x}{i} = \m{v}(\ip{x}{i})}_{\ip{x}{i}\in X}
           * \pure{R(\m{v})}
       \bigr)
   \byrules{c-pure,c-skolem}
\whichproves
     \E \hat{\prob}.
     \CMod{\hat{\prob}} (a,\m{v}) \st
       \sure{\ip{x}{i} = \m{v}(\ip{x}{i})}_{\ip{x}{i}\in X}
       * \pure{R(\m{v})}
   \byrule{c-fuse}
\whichproves
     \E \hat{\prob}.
     \pure{\hat{\prob} \circ \inv{\proj_2}(R) = 1} *
     \CMod{\hat{\prob}} (a,\m{v}) \st
       \sure{\ip{x}{i} = \m{v}(\ip{x}{i})}_{\ip{x}{i}\in X}
   \byrule{c-pure}
\whichproves
     \E \hat{\prob}.
     \pure{\hat{\prob} \circ \inv{\proj_2}(R) = 1} *
     \CMod{\hat{\prob} \circ \inv{\proj_2}} \m{v} \st
       \sure{\ip{x}{i} = \m{v}(\ip{x}{i})}_{\ip{x}{i}\in X}
   \byrule{c-sure-proj-many}
\whichproves
     \E \hat{\prob}'.
     \pure{\hat{\prob}'(R) = 1} *
     \CMod{\hat{\prob}'} \m{v} \st
       \sure{\ip{x}{i} = \m{v}(\ip{x}{i})}_{\ip{x}{i}\in X}
\whichis \cpl{R}
 \end{eqexplain}
 In the derivation we use
 $ \hat{\prob} = \prob \fuse \krnl $,
 and
 $\hat{\prob}' = \hat{\prob} \circ \inv{\proj_2}$.
\end{proof}
 \begin{lemma}
\label{proof:rl-merge}
  \Cref{rule:rl-merge} is sound.
\end{lemma}

\begin{proof}
  Let $R_1\in \Val^{X_1}$ and $R_2\in \Val^{X_2}$
  and let
  $ X = X_1 \inters X_2 $,
  $ Y_1 = X_1 \setminus X $, and
  $ Y_2 = X_2 \setminus X $, so that
  $ X_1 \union X_2 = Y_1 \dunion X \dunion Y_2 $.

  By definition, $\cpl{R_1} * \cpl{R_2}$ entails that for some
  $ \prob_1,\prob_2 $ with $ \prob_1(R_1)=1 $ and $ \prob_1(R_2)=1 $:
  \begin{eqexplain}
&
    \CC{\prob_1} \m{v}_1.
      (\sure{\ip{x}{i} = \m{v}_1(\ip{x}{i})}_{\ip{x}{i}\in X_1}) *
\CC{\prob_2} \m{v}_2.
      (\sure{\ip{x}{i} = \m{v}_2(\ip{x}{i})}_{\ip{x}{i}\in X_2})
\whichproves
  \CC{\prob_1} (\m{w}_1,\m{v}_1). (
    \sure{\ip{y}{i} = \m{w}_1(\ip{y}{i})}_{\ip{y}{i}\in Y_1}
    \land
    \sure{\ip{x}{i} = \m{v}_1(\ip{x}{i})}_{\ip{x}{i}\in X}
    * \pure{(\m{w}_1\m{v}_1) \in R_1}
  ) *
  {} \\ &
  \CC{\prob_2} (\m{w}_2,\m{v}_2). (
    \sure{\ip{y}{i} = \m{w}_2(\ip{y}{i})}_{\ip{y}{i}\in Y_2}
    \land
    \sure{\ip{x}{i} = \m{v}_2(\ip{x}{i})}_{\ip{x}{i}\in X}
    * \pure{(\m{w}_2\m{v}_2) \in R_2}
  )
  \byrule{c-pure}
\whichproves
  \CC{\prob_1} (\m{w}_1,\m{v}_1).
    \CC{\prob_2} (\m{w}_2,\m{v}_2).
    \begin{pmatrix*}[l]
    \sure{\ip{y}{i} = \m{w}_1(\ip{y}{i})}_{\ip{y}{i}\in Y_1}
          \land
          \sure{\ip{x}{i} = \m{v}_1(\ip{x}{i})}_{\ip{x}{i}\in X} * {}
    \\
    \sure{\ip{y}{i} = \m{w}_2(\ip{y}{i})}_{\ip{y}{i}\in Y_2}
          \land
          \sure{\ip{x}{i} = \m{v}_2(\ip{x}{i})}_{\ip{x}{i}\in X} * {}
    \\
      \pure{(\m{w}_1\m{v}_1) \in R_1}
    * \pure{(\m{w}_2\m{v}_2) \in R_2}
    \end{pmatrix*}
  \byrule{c-frame}
\whichproves
  \CC{\prob_1} (\m{w}_1,\m{v}_1).
    \CC{\prob_2} (\m{w}_2,\m{v}_2).
    \begin{pmatrix*}[l]
    \sure{\ip{y}{i} = \m{w}_1(\ip{y}{i})}_{\ip{y}{i}\in Y_1}
          \land
          \sure{\ip{x}{i} = \m{v}_1(\ip{x}{i})}_{\ip{x}{i}\in X} * {}
    \\
    \sure{\ip{y}{i} = \m{w}_2(\ip{y}{i})}_{\ip{y}{i}\in Y_2}
          \land
          \sure{\ip{x}{i} = \m{v}_2(\ip{x}{i})}_{\ip{x}{i}\in X} * {}
    \\
      \pure{(\m{w}_1\m{v}_1) \in R_1}
    * \pure{(\m{w}_2\m{v}_2) \in R_2}
    * \pure{\m{v}_1=\m{v}_2}
    \end{pmatrix*}
  \byrule{sure-eq-inj}
\whichproves
  \CC{\prob_1} (\m{w}_1,\m{v}_1).
  \CC{\prob_2} (\m{w}_2,\m{v}_2).
    \begin{pmatrix*}[l]
    \sure{\ip{y}{i} = \m{w}_1(\ip{y}{i})}_{\ip{y}{i}\in Y_1} \land {}
    \\
    \sure{\ip{x}{i} = \m{v}_1(\ip{x}{i})}_{\ip{x}{i}\in X} \land {}
    \\
    \sure{\ip{y}{i} = \m{w}_2(\ip{y}{i})}_{\ip{y}{i}\in Y_2} * {}
    \\
    \pure{(\m{w}_1\m{v}_1) \in R_1 \land (\m{w}_2\m{v}_1) \in R_2}
    \end{pmatrix*}
  \byrule{c-cons}
\whichproves
  \CC{\prob_1} (\m{w}_1,\m{v}_1).
  \CC{\prob_2 \circ \inv{\pi_1}} (\m{w}_2).
    \begin{pmatrix*}[l]
    \sure{\ip{y}{i} = \m{w}_1(\ip{y}{i})}_{\ip{y}{i}\in Y_1} \land {}
    \\
    \sure{\ip{x}{i} = \m{v}_1(\ip{x}{i})}_{\ip{x}{i}\in X} \land {}
    \\
    \sure{\ip{y}{i} = \m{w}_2(\ip{y}{i})}_{\ip{y}{i}\in Y_2} * {}
    \\
    \pure{(\m{w}_1\m{v}_1) \in R_1 \land (\m{w}_2\m{v}_1) \in R_2}
    \end{pmatrix*}
  \byrule{c-sure-proj}
  \end{eqexplain}



  Thus by letting $
  \prob = \prob_1 \pprod (\prob_2 \circ \inv{\pi_1})
  =\bind(\prob_1, \krnl_2)
  $ where
  \[
    \krnl_2 = \fun (\m{w}_1\m{v}_1).(
      \bind(\prob_2,\fun (\m{w}_2,\m{v}_2).
                        \return (\m{w}_1\m{w}_2\m{v}_1))
    )
  \]
  we obtain:
  \begin{eqexplain}
  &
  \CC{\prob_1} (\m{w}_1',\m{v}_1').
    \CC{\krnl_2(\m{w}_1',\m{v}_1')} (\m{w}_1,\m{w}_2,\m{w}).
    \begin{pmatrix*}[l]
    \sure{\ip{y}{i} = \m{w}_1(\ip{y}{i})}_{\ip{y}{i}\in Y_1} \land {}
    \\
    \sure{\ip{x}{i} = \m{w}(\ip{x}{i})}_{\ip{x}{i}\in X} \land {}
    \\
    \sure{\ip{y}{i} = \m{w}_2(\ip{y}{i})}_{\ip{y}{i}\in Y_2} * {}
    \\
    \pure{(\m{w}_1\m{w}) \in R_1 \land (\m{w}_2\m{w}) \in R_2}
    \end{pmatrix*}
\whichproves
  \CC{\prob} (\m{w}_1,\m{w}_2,\m{w}).
    \begin{pmatrix*}[l]
    \sure{\ip{y}{i} = \m{w}_1(\ip{y}{i})}_{\ip{y}{i}\in Y_1} \land {}
    \\
    \sure{\ip{x}{i} = \m{w}(\ip{x}{i})}_{\ip{x}{i}\in X} \land {}
    \\
    \sure{\ip{y}{i} = \m{w}_2(\ip{y}{i})}_{\ip{y}{i}\in Y_2}
    \\
    \pure{(\m{w}_1\m{w}) \in R_1 \land (\m{w}_2\m{w}) \in R_2}
    \end{pmatrix*}
  \byrules{c-assoc,c-sure-proj}
\whichproves
  \CC{\prob} \m{v}.
    \sure{\ip{x}{i} = \m{v}(\ip{x}{i})}_{\ip{x}{i}\in (X_1\union X_2)}
    * \pure{(\restr{\m{v}}{X_1}) \in R_1 \land (\restr{\m{v}}{X_2}) \in R_2}
  \end{eqexplain}
  The result gives us $ \cpl{R_1 \land R_2} $
  by \ref{rule:c-pure} and \cref{def:rel-lift}.
\end{proof} \begin{lemma}
\label{proof:rl-sure-merge}
  \Cref{rule:rl-sure-merge} is sound.
\end{lemma}

\begin{proof}
  \begin{eqexplain}
    \cpl{R} * \sure{\ip{x}{i} = \expr\at{i}}
  \whichproves*
    \E\prob.
      \CC\prob \m{v}. \bigl(
        \sure{\ip{y}{i} = \m{v}(\ip{y}{i})}_{\ip{y}{i}\in X}
        * \pure{R(\m{v})}
      \bigr)
      * \sure{\ip{x}{i} = \expr\at{i}}
  \bydef
\whichproves
    \E\prob.
      \CC\prob \m{v}. \bigl(
        \sure{\ip{y}{i} = \m{v}(\ip{y}{i})}_{\ip{y}{i}\in X}
        * \pure{R(\m{v})}
        * \sure{\ip{x}{i} = \expr\at{i}}
      \bigr)
  \byrule{c-frame}
\whichproves
    \E\prob.
      \CC\prob \m{v}. \bigl(
        \sure{\ip{y}{i} = \m{v}(\ip{y}{i})}_{\ip{y}{i}\in X}
        * \pure{R(\m{v})}
        * \sure{\ip{x}{i} = \sem{\expr\at{i}}(\m{v})}
      \bigr)
  \by{$\pvar(\expr\at{i}) \subs X$}
\whichproves
    \E\prob.
      \CC\prob \m{v}. \bigl(
        \sure{\ip{y}{i} = \m{v}(\ip{y}{i})}_{\ip{y}{i}\in X}
        * \pure{R(\m{v})}
        * \CC{\dirac{\sem{\expr\at{i}}(\m{v})}} w.\sure{\ip{x}{i} = w}
      \bigr)
  \byrule{c-unit-l}
\whichproves
    \E\prob.
      \CC\prob \m{v}.
      \CC{\dirac{\sem{\expr\at{i}}(\m{v})}} w.
      \bigl(
        \sure{\ip{y}{i} = \m{v}(\ip{y}{i})}_{\ip{y}{i}\in X}
        * \pure{R(\m{v})}
        * \sure{\ip{x}{i} = w}
      \bigr)
  \byrule{c-frame}
\whichproves
    \E\prob'.
      \CC{\prob'} \m{v}'.
      \begin{grp}
        \sure{\ip{y}{i} = \m{v}'(\ip{y}{i})}_{\ip{y}{i}\in X}
        * \sure{\ip{x}{i} = \m{v}'(\ip{x}{i})} \\ {}
        * \pure{R(\m{v}') \land \sem{\expr\at{i}}(\m{v}') = \m{v}'(\ip{x}{i})}
      \end{grp}
  \byrules{c-pure,c-assoc}
\whichproves
    \cpl{R \land \ip{x}{i} = \expr\at{i}}
  \end{eqexplain}
  where we let
  $
    \prob' \is
      \left(
      \DO{
        \m{v} <- \prob;
        \return(\m{v}\upd{\ip{x}{i} -> \sem{\expr\at{i}}(\m{v})})
      }
      \right).
  $
\end{proof} \begin{lemma}
\label{proof:coupling}
  \Cref{rule:coupling} is sound.
\end{lemma}

\begin{proof}
  Assuming
  $\prob \circ \inv{\proj_1} = \prob_1$,
  $\prob \circ \inv{\proj_2} = \prob_2$, and
  $\prob(R) = 1$, we have:
  \begin{eqexplain}
    \distAs{\p{x}_1\at{\I1}}{\prob_1} *
    \distAs{\p{x}_2\at{\I2}}{\prob_2}
\whichproves*
    \CC{\prob_1} v. \sure{x_1\at{1} = v} *
    \CC{\prob_2} w. \sure{x_2\at{2} = w}
    \byrule{c-unit-r}
\whichproves
    \CC{\prob} (v, w). \sure{x_1\at{1} = v} *
    \CC{\prob} (v, w). \sure{x_2\at{2} = w}
    \byrule{c-sure-proj}
\whichproves
    \CC{\prob} (v, w). \sure{x_1\at{1} = v} \land
    \CC{\prob} (v, w). \sure{x_2\at{2} = w}
    \byrule{and-to-star}
\whichproves
    \CC{\prob} (v, w).
      (\sure{x_1\at{1} = v} \land
      \sure{x_2\at{2} = w})
    \byrule{c-and}
\whichproves
    \cpl{R(x_1\at{1}, x_2\at{2})}
    \by{$\prob(R) = 1$}
  \end{eqexplain}
\end{proof}
 
\subsubsection{Weakest Precondition}
\begin{lemma}
\label{proof:wp-loop-0}
  \Cref{rule:wp-loop-0} is sound.
\end{lemma}

\begin{proof}
  Special case of \ref{rule:wp-loop} with $n=0$,
  which makes the premises trivial.
\end{proof} \begin{lemma}
\label{proof:wp-loop-lockstep}
  \Cref{rule:wp-loop-lockstep} is sound.
\end{lemma}

\begin{proof}
  We derive the following rule:
  \[
  \infer*{
    \forall k < n\st
      P(k) \proves \WP {\m[i: t, j: t']}{P(k+1)}
  }{
    P(0) \proves
    \WP {\m[i: (\Loop{n}{t}), j: (\Loop{n}{t'})]} {P(n)}
  }
  \]
  (for $n\in \Nat$ and $i \ne j$)
  from the standard \ref{rule:wp-loop},
  as follows.
  Let
  \[
    P'(k) \is
    \WP {\m[j: \Loop{k}{t'}]} {P(k)}
  \]
  Note that
  $P(0) \proves P'(0)$
  by \ref{rule:wp-loop-0}.
  Then we can apply the \ref{rule:wp-loop} using $P'$ as a loop invariant
  \begin{derivation}
    \infer*[Right=\ref{rule:wp-nest}]{
    \infer*{
    \infer*[Right=\ref{rule:wp-loop}]{
    \infer*{
    \infer*[Right=\ref{rule:wp-nest}]{
    \infer*[Right=\ref{rule:wp-loop-unf}]{
    \infer*[Right=\ref{rule:wp-cons}]{
    \infer*[Right=\ref{rule:wp-nest}]{
    \infer*{}{
      \forall k \leq n \st
      P(k)
      \proves
      \WP {\m[i: t, j: t']} {P(k+1)}
    }}{
      \forall k \leq n \st
      P(k)
      \proves
        \WP {\m[j: t']} {
          \WP {\m[i: t]} {P(k+1)}
      }
    }}{
      \forall k \leq n \st
      \WP {\m[j: \Loop{k}{t'}]} {P(k)}
      \proves
      \WP {\m[j: \Loop{k}{t'}]}[\big]{
        \WP {\m[j: t']} {
          \WP {\m[i: t]} {P(k+1)}
        }
      }
    }}{
      \forall k \leq n \st
      \WP {\m[j: \Loop{k}{t'}]} {P(k)}
      \proves
      \WP {\m[j: \Loop{(k+1)}{t'}]}[\big]{
        \WP {\m[i: t]} {P(k+1)}
      }
    }}{
      \forall k \leq n \st
      \WP {\m[j: \Loop{k}{t'}]} {P(k)}
      \proves
      \WP {\m[i: t]}[\big]{
        \WP {\m[j: \Loop{(k+1)}{t'}]} {P(k+1)}
      }
    }}{
      \forall k \leq n \st
        P(k)' \proves
          \WP {\m[i: t]} {P'(k+1)}
    }}{
      P'(0) \proves
      \WP {\m[i: (\Loop{n}{t})]}{
        P'(n)
      }
    }}{
      P(0) \proves
      \WP {\m[i: (\Loop{n}{t})]}{
        \WP{\m[j: (\Loop{n}{t'})]} {P(n)}
      }
    }}{
      P(0) \proves
      \WP {\m[i: (\Loop{n}{t}), j: (\Loop{n}{t'})]} {P(n)}
    }
  \end{derivation}
  From bottom to top,
    we focus on component $i$ using \ref{rule:wp-nest};
    then we use $P(0) \proves P'(0)$ and transitivity of entailment
    to rewrite the goal using the invariant $P'$;
    we then use \ref{rule:wp-loop} and unfold the invariant;
    using \ref{rule:wp-nest} twice we can swap the two components so that
    component~$j$ is the topmost WP in the assumption and conclusion;
    using \ref{rule:wp-loop-unf} we break off the first $k$ iterations at~$j$;
    finally, using \ref{rule:wp-cons} we can eliminate the topmost
    WP on both sides of the entailments.

  It is straightforward to adapt the argument for any number of components
  looping the same number of times.
\end{proof} \begin{lemma}
\label{proof:wp-rl-assign}
  \Cref{rule:wp-rl-assign} is sound.
\end{lemma}

\begin{proof}
  Define
    $\m{\permap}_R \is (\m{\permap} \setminus \ip{x}{i})/2$ and
    $ \m{\permap}_{\p{x}} \is \m{\permap} - \m{\permap}_R $;
  note that by $\m{\permap}(\ip{x}{i})=1$
  we have $\m{\permap}_{\p{x}}(\ip{x}{i})=1$.
  We first show that the following hold:
  \begin{align}
    \cpl{R}\withp{\m{\permap}}
    &\proves
    \cpl{R}\withp{\m{\permap}_R} * (\m{\permap}_{\p{x}})
    \\
    \cpl{R}\withp{\m{\permap}_R}
      * \sure{\p{x}\at{i} = \expr\at{i}}\withp{\m{\permap}_{\p{x}}}
    &\proves
    \cpl{R \land \p{x}\at{i} = \expr\at{i}}\withp{\m{\permap}}
  \end{align}
  The first entailment holds because $\cpl{R}$ is permission-scaling-invariant
  (see \cref{sec:appendix:permissions})
  and by the assumption that $\p{x}\notin \pvar(R)$.
  The second entailment holds by \ref{rule:rl-sure-merge}.

  We can then derive:
  \[
  \begin{derivation}
    \infer*[Right=\ref{rule:wp-cons}]{
    \infer*[Right=\ref{rule:wp-frame}]{
    \infer*[Right=\ref{rule:wp-assign}]{ }{
      (\m{\permap}_{\p{x}})
      \proves
      \WP {\m[i: \code{x:=}\expr]}[\big] {
          \sure{\p{x}\at{i} = \expr\at{i}}\withp{\m{\permap}_{\p{x}}}
      }
    }}{
      \cpl{R}\withp{\m{\permap}_R}
      * (\m{\permap}_{\p{x}})
      \proves
      \WP {\m[i: \code{x:=}\expr]}[\big] {
        \cpl{R}\withp{\m{\permap}_R}
          * \sure{\p{x}\at{i} = \expr\at{i}}\withp{\m{\permap}_{\p{x}}}
      }
    }}{
      \cpl{R}\withp{\m{\permap}}
      \proves
      \WP {\m[i: \code{x:=}\expr]}[\big] {
        \cpl{R \land \p{x}\at{i} = \expr\at{i}}\withp{\m{\permap}}
      }
    }
  \end{derivation}
  \qedhere
  \]
\end{proof} \begin{lemma}
\label{proof:wp-if-unary}
  \Cref{rule:wp-if-unary} is sound.
\end{lemma}

\begin{proof}
  From the premises, we derive:
  \[
  \begin{derivation}
  \infer*[right=\ref{rule:c-wp-swap}]{
  \infer*[Right={\ref{rule:c-unit-r},\ref{rule:c-frame}}]{
  \infer*[Right=\ref{rule:c-cons}]{
  \infer*[Right=\ref{rule:wp-bind}]{
  \infer*[Right=\ref{rule:wp-if-prim}]{
  \infer*{
    P * \sure{\Ip{e}{1}=1} \gproves \WP {\m[\I1:t_1]} {Q(1)}
    \\
    P * \sure{\Ip{e}{1}=0} \gproves \WP {\m[\I1:t_2]} {Q(0)}
  }{\forall b\in\set{0,1}\st
    P * \sure{\Ip{e}{1}=1}
    \gproves
    \ITE{b}{\WP{\m[\I1:t_1]}{Q(1)}}{\WP{\m[\I1:t_2]}{Q(0)}}
  }}{\forall b\in\set{0,1}\st
    P * \sure{\Ip{e}{1}=b}
    \gproves
    \WP {\m[\I1:
        (\code{if $\;b\;$ then $\;t_1\;$ else $\;t_2$})
      ]}{
      Q(b\beq 1)
    }
  }}{\forall b\in\set{0,1}\st
    P * \sure{\Ip{e}{1}=b}
    \gproves
    \WP {\m[\I1:
      (\code{if e then $\;t_1\;$ else $\;t_2$})
    ]}{
      Q(b\beq 1)
    }
  }}{\CC{\beta} b.(P * \sure{\Ip{e}{1}=b})
    \gproves
    \CC{\beta} b.
    \WP {\m[\I1:
      (\code{if e then $\;t_1\;$ else $\;t_2$})
    ]}{
      Q(b\beq 1)
    }
  }}{P * \distAs{\Ip{e}{1}}{\beta}
    \gproves
    \CC{\beta} b.
    \WP {\m[\I1:
      (\code{if e then $\;t_1\;$ else $\;t_2$})
    ]}{
      Q(b\beq 1)
    }
  }}{P * \distAs{\Ip{e}{1}}{{\beta}}
    \gproves
    \WP {\m[\I1:
        (\code{if e then $\;t_1\;$ else $\;t_2$})
      ]}{
      \CC{\beta} b.Q(b\beq 1)
    }
  }
  \end{derivation}
  \qedhere
  \]
\end{proof}
