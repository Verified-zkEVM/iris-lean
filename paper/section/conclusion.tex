\section{Conclusions and Future Work}
\thelogic's journey started as a quest to integrate unary and relational
probabilistic reasoning and ended up uncovering \supercond{}
as a key fundational tool.
Remarkably, to achieve our goal we had to deviate from Lilac's previous
proposal in both the definition of conditioning,
  to enable the encoding of relational lifting,
and of ownership (with almost measurability),
  to resolve an issue with almost sure assertions
(recently corrected~\cite{lilac2} in a different way).
In addition, our model supports mutable state without sacrificing
expressiveness.
One limitation of our current model is lack of support for continuous
distributions.
Lilac's model and recent advances in it~\cite{LiAJ0H24} could suggest a pathway for a continuous extension of \thelogic,
but it is unclear if all our rules would be still valid;
for example \cref{rule:c-fuse}'s soundness hinges on properties of discrete distributions that we could not extend to the general case in an obvious way.
\thelogic's encoding of relational lifting and the novel proof principles it uncovered for it are a demonstration of the potential
of \supercond\ as a basis for deriving high-level logics on top of an ergonomic
core logic.
Obvious candidates for such scheme are approximate couplings~\cite{apRHL}
(which have been used for \eg differential privacy),
and expectation-based calculi (à la Ellora).
