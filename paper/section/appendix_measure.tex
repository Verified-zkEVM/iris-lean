\section{Measure Theory Lemmas}

\paragraph{Notation}
In what follows,
given $n\in\Nat$ with~$n> 1$,
we write $\numlist{n}$
to denote the set $\set{1, \dots, n}$.
Moreover, for iterated summation we use the
notation
$
  \sum_{i \in I \mid \Phi(i)} f(i)
$
where~$I = \set{i_0, i_1, \dots}$ is countable
and $\Phi$ is a predicate on elements of~$I$,
to denote the sum $ f(j_0) + f(j_1) + \dots $
where $j_0, j_1, \dots$ is the sublist of $ i_0, i_1, \dots $ consisting
of the elements that satisfy~$\Phi$.
A similar convention is used for other commutative and associative operators,
\eg $\union$.
A countable partition of~$\Outcomes$ is a partition of $\Outcomes$,
$S \subs \powerset(\Outcomes)$,
with countably many sets.
For uniformity, we represent countable partitions as $S = \cpart{A}$
with the convention that when the partition has finitely many sets,
say~$n$, all the $A_i$ with $i \geq n$ are empty.

As mentioned, \thelogic\ is only concerned with discrete distributions,
\ie distributions over a countable set of outcomes.
The following lemma expresses the key property of \salgebra[s]
over countable outcomes that we exploit for proving the
other results.

\begin{lemma}
  \label{thm:countable-partition-generated}
 Let $\Outcomes$ be an countable set, and $\salg$ to be an arbitrary \salgebra{}
  on $\Outcomes$. Then there exists a countable partition $S$ of~$\Outcomes$
  such that $\salg = \closure{S}$.
\end{lemma}

\begin{proof}
  For every element $x \in \Outcomes$,
  we identify the smallest event $E_x \in \salg$ such that $x \in E_x$,
  and show that for $x, z \in \Outcomes$,
  either $E_x = E_z$ or $E_x \cap E_z = \emptyset$.
  Then the set $S = \set{E_x \mid x \in \Outcomes}$
  is a partition of~$\Outcomes$,
  and any event $E \in \salg$ can be represented as
  $\Union_{x \in E} E_x$, which suffices to show that $\salg = \closure{S}$.

  For every $x, y$, let
\begin{align*}
    A_{x, y} &=
    \begin{cases}
      \Outcomes \CASE \text{$\forall E \in \salg$, either $x , y$ both in~$E$ or $x , y$ both not in~$E$} \\
      E \OTHERWISE, \text{pick any $E \in \salg$ such that $x \in E$ and $y \notin E$}
    \end{cases}
  \end{align*}
Then we show that, for all $x$,
  $E_x = \cap_{y \in \Outcomes} A_{x, y}$ is the smallest
  event in $\salg$ such that $x \in E_x$ as follows.
  If there exists $E_x'$ such that $x \in E_x'$ and $E_x' \subset E_x$,
  then $E_x \setminus E_x'$ is not empty. Let $y$ be an element of
  $E_x \setminus E_x'$, and by the definition of $A_{x, y}$, we have
  $y \notin A_{x,y}$. Thus, $y \notin \cap_{y \in \Outcomes} A_{x, y} = E_x$,
  which contradicts with $y \in E_x \setminus E_x'$.

  Next, for any  $x, z \in \Outcomes$,
  since $E_x$ is the smallest event containing $x$ and
  $E_z$ is the smallest event containing $z$,
  the smaller event $E_z \setminus E_x$ is either equivalent to
  $E_z$ or not containing $z$.
  \begin{casesplit}
    \item
      If $E_z \setminus E_x = E_z$, then $E_x$ and $E_z$ are
      disjoint.
    \item If $z \not\in E_z \setminus E_x$, then it must $z \in E_x$,
      which implies that there exists \emph{no} $E \in \salg$ such that
      $x \in E$ and $z \notin E$. Because $\salg$ is closed under
      complement, then there exists \emph{no} $E \in \salg$ such that
      $x \notin E$ and $z \in E$ as well. Therefore,
      we have $x \in  \Inters_{y \in \Outcomes} A_{z, y} = E_z$ as well.
      Furthermore, because $E_z$ is the smallest event in
      $\salg$ that contains $z$ and $E_x$ also contains $z$,
      we have $E_z \subseteq E_x$; symmetrically, we have
      $E_x \subseteq E_z$.
      Thus, $E_x = E_z$.
    \end{casesplit}
    Hence,  the set $S = \set{E_x \mid x \in \Outcomes}$ is a
    countable partition of~$\Outcomes$.
\end{proof}

\begin{lemma}
\label{lemma:sigma-alg-representation}
  If $S = \cpart{A}$ is a partition of $\Outcomes$,
  and $\salg = \closure{S}$,
  then every event~$E\in\salg$ can be written as
  $E = \Dunion_{i \in I} A_i$ for some $I \subs \Nat$.
In other words,
  $
    \closure{S} = \set[\big]{ \Dunion_{i \in I} A_i | I \subseteq \Nat }.
  $
\end{lemma}

\begin{proof}
  Because \salgebra[s] are closed under countable union,
  for any $I \subseteq \Nat$,
  $ \Dunion_{i \in I} A_i \in \closure{S} $.
  Thus, $\closure{S} \supseteq \set[\big]{ \Dunion_{i \in I} A_i \mid I \subseteq \Nat }$.

  Also, $\set[\big]{ \Dunion_{i \in I} A_i \mid I \subseteq \Nat }$ is
  a \salgebra{}:
\begin{itemize}
    \item
      $\Outcomes = \Dunion_{i \in \Nat} A_i$.
    \item
      Given a countable sequences of events
      $E_1 = \Dunion_{i \in I_1} A_i$,
      $E_2 = \Dunion_{i \in I_2} A_i$,
      \dots,
      let
      $I = \Union_{j\in \Nat} I_j$;
      then we have
      $ \Union_{j\in\Nat} E_i = \Dunion_{i\in I} A_i $.
    \item
      If $E = \Dunion_{i \in I} A_i$, then
      the complement of $E$ is
      $
        (\Outcomes\setminus E) = \Dunion_{i \in (\Nat \setminus I)} A_i
      $.
  \end{itemize}
  Then, $\set{ \Dunion_{i \in I} A_i \mid I \subseteq \Nat }$ is
  a \salgebra{} that contains $S$.
  Therefore, $
    \closure{S} = \set[\big]{\Dunion_{i \in I} A_i \mid I \subseteq \Nat }
  $.
\end{proof}

\begin{lemma}
 \label{lemma:partition-order}
 Let $\Outcomes$ be a countable set.
 If\/ $S_1 = \cpart[i]{A}$ and
    $S_2 = \cpart[j]{B}$ are both countable partitions of $\Outcomes$,
 then $\closure{S_1} \subseteq \closure{S_2}$ implies that
 for any $B_j \in S_2$ with $B_j\ne \emptyset$,
 we can find a unique $A_i \in S_1$ such that $B_j \subseteq A_i$.
\end{lemma}

\begin{proof}
 For any $B_j \in S_2$ with $B_j\ne \emptyset$,
 pick an arbitrary element $s \in B_j$ and
 denote the unique element of $S_1$ that contains~$s$ as~$A_i$.
 Because $A_i \in S_1$ and $S_1 \subset  \closure{S_1} \subseteq \closure{S_2}$,  we have $A_i \in  \closure{S_2}$.
Note that $s \in B_j$ and $B_j$ is an element of the partition $S_2$
  that generates $\closure{S_2}$, $B_j$ must be the smallest event
  in $\closure{S_2}$ that contains $s$.
 Because $s \in A_i$ as well, $B_j$ being the smallest event containing $s$
  implies that $B_j \subseteq A_i$.
\end{proof}

\begin{lemma}
 \label{lemma:bind-extend}
 Assume we are given a \salgebra{} $\sigmaF_1$
 over a countable set $\Outcomes$,
 measure $\mu_1 \in \Dist(\sigmaF_1)$,
 a countable set $A$,
 a distribution $\mu \in \Full{A}$,
 and a function $\krnl_1 \colon A \to \Dist(\sigmaF_1)$
 such that $\mu_1 = \bind(\mu, \krnl_1)$.
 Then, for any probability space $(\sigmaF_2, \mu_2)$ such that
 $(\sigmaF_1, \mu_1) \extTo (\sigmaF_2, \mu_2)$,
 there exists $\krnl_2$ such that $\mu_2 = \bind(\mu, \krnl_2)$,
 and that for any $a \in \psupp(\mu)$,
 $(\sigmaF_1, \krnl_1(a)) \extTo (\sigmaF_2, \krnl_2(a))$.
\end{lemma}

\begin{proof}
  By~\cref{thm:countable-partition-generated},
  $\sigmaF_i = \closure{S_i}$ for some countable partition $S_i$.
  Also, $(\sigmaF_1, \mu_1) \extTo (\sigmaF_2, \mu_2)$ implies that
  $\sigmaF_1 \subseteq \sigmaF_2$.
  So we have $\closure{S_1} \subseteq \closure{S_2}$,
  which by~\cref{lemma:partition-order} implies that
  for any $B \in S_2$ with $B \ne \emptyset$,
  we can find a unique $A \in S_1$ such that $B \subseteq A$.
  Let~$f$ be the mapping associating to any $B\ne \emptyset$
  the corresponding $A = f(B)$, and $f(B)=\emptyset$ when $B=\emptyset$.

  Then, we define $\krnl_2$ as follows:
  for any $a \in A$, $E \in \sigmaF_2$,
  there exists $S \subseteq S_2$ such that
  $E = \Dunion_{B \in S} B$,
  then define
\begin{align*}
   \krnl_2(a)(E) = \sum_{B \in S} \krnl_1(a)(f(B)) \cdot h(B),
  \end{align*}
  where
  $h(B) = \mu_2(B) / \mu_2(f(B))$ if $ \mu_2(f(B)) \neq 0$
  and $h(B) = 0$ otherwise.

Then we calculate:
  \begin{align}
      &\bind(\mu, \krnl_2)(E) \notag\\
   &=  \sum_{a\in A} \mu(a) \cdot \krnl_2(E) \notag \\
   &=  \sum_{a\in A} \mu(a) \cdot \sum_{B \in S} \krnl_1(a)(f(B)) \cdot h(B) \notag \\
   &=  \sum_{B \in S}  \sum_{a\in A} \mu(a) \cdot \krnl_1(a)(f(B)) \cdot h(B)  \notag \\
   &=  \sum_{B \in S}  \bind(\mu, \krnl_1)(f(B)) \cdot h(B) \notag  \\
   &=  \sum_{B \in S}  \mu_1(f(B)) \cdot h(B) \notag \\
   &=  \sum_{B \in S \mid \mathrlap{\mu_2(f(B)) \neq 0}}  \mu_1(f(B)) \cdot \frac{\mu_2(B)}{ \mu_2(f(B))} \notag \\
   &=  \sum_{B \in S \mid \mathrlap{\mu_2(f(B)) \neq 0}}  \mu_2(f(B)) \cdot \frac{\mu_2(B)}{\mu_2(f(B))} \tag{$\mu_1(E') = \mu_2(E')$ for any $E' \in \sigmaF_1$}\\
   &=  \sum_{B \in S \mid \mathrlap{\mu_2(f(B)) \neq 0}}  \mu_2(B)  \notag \\
   &=  \sum_{B \in S \mid \mathrlap{\mu_2(f(B)) \neq 0}}  \mu_2(B)
       \quad+\quad
       \sum_{B \in S \mid  \mathrlap{\mu_2(f(B)) = 0}}  \mu_2(B)  \tag{Because $\mu_2(f(B)) = 0$ implies $\mu_2(B) = 0$} \\
   &= \sum_{B \in S}  \mu_2(B) \notag \\
   &= \mu_2(\Dunion_{B \in S} B ) \notag \\
   &= \mu_2(E) \notag
  \end{align}
Thus, $\bind(\mu, \krnl_2)= \mu_2$.

  Also, for any $a \in A_{\mu}$, for any $E \in \sigmaF_1$,
  there exists $S' \subseteq S_1$
  such that $E=\Dunion_{A \in S'} A$.
  \begin{align*}
   \krnl_2(a)(E)
   &= \krnl_2(a)\bigl(\Dunion_{A \in S'} A\bigr) \\
   &= \sum_{A \in S'} \krnl_2(a)(A) \\
   &= \sum_{A \in S'} \sum_{B \subseteq A \mid \mathrlap{B \in \sigmaF_2}} \krnl_2(a)(B)\\
&= \sum_{A \in S'}
        \sum_{B \subseteq A \mid \mathrlap{B \in \sigmaF_2, \mu_2(f(B)) \neq 0}}
          \quad\krnl_1(a)(f(B)) \cdot \frac{\mu_2(B)}{\mu_2(f(B))} \\
&= \sum_{A \in S' \mid \mathrlap{\mu_2(A) \neq 0}} \krnl_1(a)(A) \cdot \frac{\left(\sum_{B \subseteq A \mid B \in \sigmaF_2}  \mu_2(B) \right) }{\mu_2(A)} \\
   &= \sum_{A \in S' \mid \mathrlap{\mu_2(A) \neq 0}} \krnl_1(a)(A) \cdot \frac{ \mu_2(A)}{ \mu_2(A)} \\
   &= \sum_{A \in S' \mid \mathrlap{\mu_2(A) \neq 0}} \krnl_1(a)(A)  \\
   &= \sum_{A \in S'} \krnl_1(a)(A)  \\
   &= \krnl_1(a)\bigl(\Dunion_{A \in S'} A\bigr)  \\
   &= \krnl_1(a)(E)
  \end{align*}
Thus, for any $a$, $(\sigma_1, \krnl_1(a)) \extTo (\sigma_2, \krnl_2(a))$.
\end{proof}

\begin{lemma}
 \label{lemma:product-algebra}
 Given two \salgebra[s] $\sigmaF_1$ and $\sigmaF_2$
 over two countable underlying sets $\Outcomes_1, \Outcomes_2$,
 then a general element in the product \salgebra{}
 $\sigmaF_1 \otimes \sigmaF_2$ can
 be expressed as $\Dunion_{(i, j) \in I} (A_{i} \times B_{j})$
 for some $I \subseteq \Nat^2$ and
 $A_{i} \in \sigmaF_{1}, B_{j} \in \sigmaF_{2}$ for $(i,j) \in I$.
\end{lemma}

 \begin{proof}
By~\cref{thm:countable-partition-generated}, each \salgebra{}
  $\sigmaF_i$ is generated by a countable partition over $\Outcomes_i$.
  Let $S_1 = \cpart{A}$ be the countable partition that generates $\sigmaF_1$,
  $S_2 = \cpart{B}$ be the countable partition that generates $\sigmaF_2$.
    By \cref{lemma:sigma-alg-representation},
    a general element in $\sigmaF_1$ can be written as
  $\Dunion_{j \in J} A_{j}$ for some $J \subseteq \Nat$,
  and similarly,
   a general element in $\sigmaF_2$ can be written as
  $\Dunion_{k \in K} B_{k}$ for some $K \subseteq \Nat$.

  Note that $\{A_j \times B_k \}_{j, k \in \Nat}$ is a partition because:
  if $(A_j \times B_k)  \cap (A_{j'} \times B_{k'}) \neq \emptyset$ for some
  $j \neq j'$ and $k \neq k'$,
  then it must $A_j \cap A_{j'} \neq \emptyset$ and $B_k \cap B_{k'} \neq \emptyset$,
  and that imply that $A_j =A_{j'}$ and $B_j =B_{j'}$;
  therefore, $A_j \times B_k = A_{j'} \times B_{k'}$.

  We next show that $\sigmaF_1 \otimes \sigmaF_2$ is generated by
  the partition $\{A_j \times B_k \}_{j, k \in \Nat}$.
\begin{align*}
   \sigmaF_1 \otimes \sigmaF_2
   &= \closure*{\sigmaF_1 \times \sigmaF_2} \\
   &= \closure*{\set*{\Dunion_{j \in J_1} A_{j} \times \Dunion_{j \in J_2} B_{j} | J_1, J_2 \subseteq \Nat}} \\
   &= \closure*{\set*{\Dunion_{j \in J_1, k \in J_2}  (A_{j} \times B_{k}) | J_1, J_2 \subseteq \Nat }} \\
   &= \closure*{\set*{ A_{j} \times B_{k} | j, k\subseteq \Nat }}
   \end{align*}
Since each $A_j \in S_1 \subseteq \sigmaF_1$ and $B_k \in S_2 \subseteq \sigmaF_2$
   a general element in $\sigmaF_1 \otimes \sigmaF_2$ can
  be expressed as $\set*{\Dunion_{j, k \subseteq I} (A_{j} \times B_{k}) \mid
  A_{j} \in \sigmaF_{1}, B_{k} \in \sigmaF_{2}, I \subseteq \Nat^2}$
according to \cref{thm:countable-partition-generated}.
 \end{proof}


 \begin{lemma}
  \label{lemma:indep-prod-exists}
  Given two probability spaces
  $(\sigmaF_a, \mu_a), (\sigmaF_b, \mu_b) \in \ProbSp(\Outcomes)$,
  their independent product
  $(\sigmaF_a, \mu_a) \iprod (\sigmaF_b, \mu_b)$ exists
  if $\mu_a(E_a) \cdot \mu_b(E_b) = 0 $
  for any $E_a \in \sigmaF_a, E_b \in \sigmaF_b$ such that
  $E_a \cap E_b = \emptyset$.
 \end{lemma}

 \begin{proof}
We first define $\mu: \set{E_a \cap E_b \mid E_a \in \sigmaF_a, E_b \in \sigmaF_b}
  \to [0,1]$ by $\mu(E_a \cap E_b) = \mu_a(E_a) \cdot \mu_b(E_b)$
   for any $E_a \in \sigmaF_a, E_b \in \sigmaF_b$,
   and then show that $\mu$ could be extended to a probability
   measure on $\sigmaF_a \punion \sigmaF_b$.

  \begin{itemize}
   \item We first need to show that $\mu$ is \textbf{well-defined}.
    That is,
    $E_a \cap E_b = E_a' \cap E_b'$
    implies $\mu_a(E_a) \cdot \mu_b(E_b) = \mu_a(E'_a) \cdot \mu_b(E'_b)$.

     When $E_a \cap E_b = E_a' \cap E_b'$, it must $E_a \cap E_a' \supseteq
     E_a \cap E_b = E_a' \cap E_b'$, Thus, $E_a \setminus E_a' \subseteq E_a
     \setminus E_b$, and then $E_a \setminus E_a'$ is disjoint from $E_b$;
     symmetrically, $E_a' \setminus E_a$ is disjoint from $E_b'$.
     Since $E_a, E_a'$ are both in $\sigmaF_{a}$, we have $E_a \setminus E_a'$
     and $E_a' \setminus E_a$ both measurable in $\sigmaF_a$.
     Their disjointness and the result above implies that
     $\mu_a(E_a \setminus E_a') \cdot \mu_b(E_b) = 0$ and
     $\mu_a(E'_a \setminus E_a) \cdot \mu_b(E'_b) = 0$.
     Symmetric reasoning can also show that
     $E'_b \setminus E_b$ is disjoint from $E'_a \cap E_a$,
       and  $E_b \setminus E'_b$ is disjoint from $E'_a \cap E_a$,
       which implies
       $\mu_a(E_b \setminus E'_b) \cdot \mu_b(E'_a \cap E_a) = 0$ and
     $\mu_a(E'_b \setminus E_b) \cdot \mu_b(E'_a) = 0$.

     Then there are four possibilities:
     \begin{itemize}
      \item If $\mu_b(E_b) = 0$ and $\mu_b(E_b') = 0$,
       then $\mu_a(E_a) \cdot \mu_b(E_b) = 0 = \mu_a(E_a') \cdot \mu_b(E_b')$.
      \item If $\mu_a(E_a \setminus E'_a) = 0$ and $\mu_b(E'_a \setminus E_a) = 0$.
       Then
       \begin{align*}
        \mu_a(E_a) \cdot \mu_b(E_b) &=
        \mu_a((E'_a \setminus E_a) \disjunion (E'_a \cap E_a)) \cdot \mu_b(E_b) \\
          &= (\mu_a(E'_a \setminus E_a) + \mu_a(E'_a \cap E_a)) \cdot \mu_b(E_b) \\
          &= \mu_a(E'_a \cap E_a) \cdot \mu_b(E_b) \\
          &= (\mu_a(E_a \setminus E'_a) + \mu_a(E'_a \cap E_a)) \cdot \mu_b(E_b) \\
          &= \mu_a(E'_a) \cdot \mu_b(E_b)
       \end{align*}

       Thus,  either $\mu_a(E'_a \cap E_a) = 0$, which implies that
       \[
        \mu_a(E_a) \cdot \mu_b(E_b)  = (0 + 0) \cdot  \mu_b(E_b) = 0 =(0+0) \cdot \mu_b(E_b) = \mu_a(E'_a) \cdot \mu_b(E'_b),
       \]
       or we have both $\mu_b(E'_b \setminus E_b) = 0$ and $\mu_b(E_b \setminus E'_b) = 0$, which imply that
       \begin{align*}
        \mu_a(E_a) \cdot \mu_b(E_b)
          &= \mu_a(E'_a) \cdot \mu_b(E_b)\\
          &= \mu_a(E'_a) \cdot \mu_b((E_b \cap E'_b ) \disjunion (E_b \setminus E'_b)) \\
          &= \mu_a(E'_a) \cdot (\mu_b(E_b \cap E'_b ) + 0) \\
          &= \mu_a(E'_a) \cdot (\mu_b(E_b \cap E'_b ) + \mu_b(E'_b \setminus E_b)) \\
          &= \mu_a(E'_a) \cdot \mu_b(E'_b ).
       \end{align*}
       \item If $\mu_b(E'_b) = 0$ and $\mu_b(E_a \setminus E'_a) = 0$,
        then
        \begin{align*}
         \mu_a(E_a) \cdot \mu_b(E_b)
         &= (\mu_a(E_a \cap E'_a) + \mu_a(E_a \setminus E'_a)) \cdot (\mu_b(E_b \cap E'_b) + \mu_b(E_b \setminus E'_b)) \\
         &= \mu_a(E_a \cap E'_a)  \cdot \mu_b(E_b \setminus E'_b)
        \end{align*}
Because $\mu_a(E_b \setminus E'_b) \cdot \mu_b(E'_a \cap E_a) = 0$ and
     $\mu_a(E'_b \setminus E_b) \cdot \mu_b(E'_a) = 0$.
        Thus, $\mu_a(E_a) \cdot \mu_b(E_b) =0 = \mu_a(E'_a) \cdot \mu_b(E'_b)$.

       \item If $\mu_b(E_b) = 0$ and $\mu_b(E'_a \setminus E_a) = 0$,
        then symmetric as above.

     \end{itemize}

     In all these cases,
     $\mu_a(E_a) \cdot \mu_b(E_b) = \mu_a(E'_a) \cdot \mu_b(E'_b)$ as desired.

    \item Show that $\mu$ satisfy \textbf{countable additivity} in $\{E_a \cap E_b \mid E_a \in \sigmaF_a,
   E_b \in  \sigmaF_b\}$.

   We start with showing that $\mu$ is finite-additive.
   Suppose $E_a^n \cap E_b^n = \Disjunion_{i \in [n]}(A_i \cap B_i)$ where
   each $A_i \in \sigmaF_a$ and $B_i \in \sigmaF_b$.
  Fix any $A_i \cap B_i$, there is unique minimal $A \in \sigmaF_a$ containing
  $A_i \cap B_i$, because if $A \supseteq A_i \cap B_i$ and  $A' \supseteq
  A_i \cap B_i$, then  $A \cap A' \supseteq A_i \cap B_i$
  and $A \cap A' \in \sigmaF_A$ too, and $A \cap A'$ is smaller.
   Because we have shown that $\mu$ is well-defined, in the following proof,
   we can assume without loss of generality that $A_i$ is the smallest set in $\sigmaF_a$ containing $A_i \cap B_i$.
   Similarly, we let $B_i$ to be the smallest set in $\sigmaF_b$ containing $A_i \cap B_i$.
   Thus,  $E_a^n \cap E_b^n = \Disjunion_{i \in [n]}(A_i \cap B_i)$ implies
   every $A_i$ is smaller than $E_a^n$ and every $B_i$ is smaller than $E_b^n$.
   Therefore,
   $E_a^n \supseteq \cup_{i \in [n]} A_i$ and
   $E_b^n \supseteq \cup_{i \in [n]} B_i$,
   which implies that
   \[
    E_a^n \cap E_b^n \supseteq (\cup_{i \in [n]} A_i) \cap (\cup_{i \in [n]} B_i) \supseteq \cup_{i \in [n] } (A_i \cap B_i) = E_a^n \cap E_b^n,
   \]
   which implies that the $\supseteq$ in the inequalities all collapse to $=$.




   For any $I \subseteq [n]$, define
   $\alpha_I = \cap_{i \in I} A_i \setminus (\cup_{i \in [n] \setminus I} A_i)$, and $\beta_I = \cap_{i \in I} B_i \setminus (\cup_{i \in [n] \setminus I} B_i)$.
   For any $I \neq I'$, $\alpha_I \cap \alpha_{I'} = \emptyset$.
   Thus, $\{\alpha_I\}_{I \subseteq [n]}$ is a set of disjoint sets in $\cup_{i \in [n]} A_i$,
   and similarly, $\{\beta_I\}_{I \subseteq [n]}$ is a set of disjoint sets in $\cup_{i \in [n]} B_i$.
   Also, for any $i \in [n]$,
   we have
   $A_i = \cup_{I \subseteq [n] \mid i \in I} \alpha_I $
   and
   $B_i = \cup_{I \subseteq [n] \mid i \in I} \beta_I $.
   Furthermore, for any $I$,
\begin{align*}
    \alpha_I \cap \cup_{i \in [n]} B_i
    \subseteq (\cup_{i \in [n]} A_i) \cap (\cup_{i \in [n]} B_i)
    = \Dunion_{i \in [n]} A_i \cap B_i ,
   \end{align*}
and thus,
\begin{align}
    \alpha_I \cap \cup_{i \in [n]} B_i
    & = (\Dunion_{i \in [n]} A_i \cap B_i) \cap (\alpha_I \cap \cup_{i \in [n]} B_i) \notag \\
    & = \Dunion_{i \in [n]} \left( A_i \cap B_i \cap \alpha_I \cap \cup_{j \in [n]} B_j \right) \notag \\
    & = \Dunion_{i \in I} \left( A_i \cap B_i \cap \alpha_I \cap \cup_{j \in [n]} B_j \right)  \tag{$A_i \cap \alpha_I = \emptyset$ if $i \notin I$} \\
    & = \Dunion_{i \in I} \left( A_i \cap B_i \cap \alpha_I\right)
    \tag{$B_i \cap \cup_{j \in [n]} B_j = B_i$ for any $i$}\\
    & = \Dunion_{i \in I} \left( B_i \cap \alpha_I\right)
    \tag{$A_i \cap \alpha_I = \alpha_I$ for any $i \in I$ }\\
    & = \alpha_I \cap \cup_{i \in I } B_i
    \label{eq:finite-add-alpha}
   \end{align}

   Now,
  \begin{align}
   &\mu(E^n_a \cap E^n_b) \notag \\
   &= \mu((\cup_{i \in [n]} A_i) \cap (\cup_{i \in [n]} B_i)) \notag \\
   &= \mu((\Dunion_{I \subseteq [n]} \alpha_I) \cap (\cup_{i \in [n]} B_i)) \tag{By definition of $\alpha_I$}\\
   &= \mu_a(\Dunion_{I \subseteq [n]} \alpha_I) \cdot  \mu_b(\cup_{i \in [n]} B_i)  \tag{By definition of $\mu$} \\
   &= \left\lgroup\sum_{I \subseteq [n]} \mu_a (\alpha_I) \right\rgroup \cdot  \mu_b(\cup_{i \in [n]} B_i)  \tag{By finite-additivity of $\mu_a$} \\
   &= \sum_{I \subseteq [n]} \mu_a (\alpha_I) \cdot  \mu_b(\cup_{i \in [n]} B_i) \notag\\
   &= \sum_{I \subseteq [n]}  \mu(\alpha_I \cap (\cup_{i \in [n]} B_i)) \tag{By definition of $\mu$} \\
   &= \sum_{I \subseteq [n]}  \mu(\alpha_I \cap (\cup_{i \in I} B_i))
   \tag{By~\cref{eq:finite-add-alpha}}\\
   &= \sum_{I \subseteq [n]}  \mu_a(\alpha_I) \cdot \mu_b (\cup_{i \in I} B_i)\tag{By definition of $\mu$} \\
   &= \sum_{I \subseteq [n]}  \mu_a(\alpha_I) \cdot \mu_b (\cup_{i \in I} (\disjunion_{I' \subseteq [n] \mid i \in I'} \beta_{I'}))
   \tag{By definition of $\beta_I$} \\
   &= \sum_{I \subseteq [n]}  \mu_a(\alpha_I) \cdot \mu_b (\disjunion_{I' \subseteq [n] \mid I \cap I' \neq \emptyset} \beta_{I'}) \notag \\
   &= \sum_{I \subseteq [n]}  \mu_a(\alpha_I) \cdot \sum_{I' \subseteq [n] \mid \mathrlap{I \cap I' \neq \emptyset}} \mu_b (\beta_{I'}) \notag \\
   &= \sum_{I \subseteq [n]} \sum_{I' \subseteq [n] \mid \mathrlap{I \cap I' \neq \emptyset}}  \mu_a(\alpha_I) \cdot \mu_b (\beta_{I'}) \notag
  \end{align}
Meanwhile, for any $I, I'$, if $|I \cap I'| \geq 2$,
  then there exists some $j, k$ such that $j \in I \cap I'$ and $k \in I \cap I'$,
  so
\begin{align}
    \mu_{a}(\alpha_I) \cdot \mu_{b}(\beta_{I'})
    &= \mu_{a}(\cap_{i \in I} A_i \setminus (\cup_{i \in [n] \setminus I} A_i)) \cdot \mu_{b}(\cap_{i \in I} B_i \setminus (\cup_{i \in [n] \setminus I} B_i)) \notag \\
    &\leq \mu_{a}(A_j \cap A_k) \cdot \mu_{b}(B_j \cap B_k) \notag \\
    &= \mu(A_j \cap A_k \cap B_j \cap B_k) \notag \\
    &= \mu((A_j\cap B_j) \cap (A_k \cap B_k)) \notag \\
    &= \mu(\emptyset) \notag \\
    &= 0. \notag
  \end{align}
Thus, continuing our previous derivation,
\begin{align}
   &\mu(E^n_a \cap E^n_b) \notag \\
   &= \sum_{I \subseteq [n]} \sum_{I' \subseteq [n] \mid \mathrlap{I \cap I' \neq \emptyset}}  \mu_a(\alpha_I) \cdot \mu_b (\beta_{I'}) \notag \\
   &= \sum_{I \subseteq [n]}
        \sum_{I' \subseteq [n] \mid \mathrlap{1 = \card{I \cap I'}}}
          \mu_a(\alpha_I) \cdot \mu_b (\beta_{I'})
   \tag{Because $\mu_{a}(\alpha_I) \cdot \mu_{b}(\beta_{I'})=0$ if $|I \cap I'| \geq 2$}\\
   &= \sum_{i \in [n]}
        \sum_{I \subseteq [n] \mid i \in I}
          \sum_{I' \subseteq [n] \mid \mathrlap{I \cap I' = \set{i}} }
            \mu_a(\alpha_I) \cdot \mu_b (\beta_{I'})
      \notag \\
   &= \sum_{i \in [n]} \sum_{I \subseteq [n] \mid i \in I} \sum_{I' \subseteq [n] \mid \mathrlap{i \in I'}}  \mu_a(\alpha_I) \cdot \mu_b (\beta_{I'})
   \tag{Because $\mu_{a}(\alpha_I) \cdot \mu_{b}(\beta_{I'})=0$ if $|I \cap I'| \geq 2$}\\
   &= \sum_{i \in [n]}
         \left\lgroup
         \sum_{I \subseteq [n] \mid \mathrlap{i \in I}} \mu_a(\alpha_I)
         \cdot
         \sum_{I' \subseteq [n] \mid \mathrlap{i \in I'}} \mu_b (\beta_{I'})
         \right\rgroup
     \notag \\
   &= \sum_{i \in [n]} \mu_a(A_i) \cdot \mu_b(B_i) \notag \\
   &= \sum_{i \in [n]} \mu(A_i \cap B_i) \notag
  \end{align}

  Thus, we established the finite additivity.
  For countable additivity, suppose $E_a \cap E_b = \Disjunion_{i \in \Nat}(A_i \cap B_i)$. By the same reason as above, we also have
  \[
    E_a \cap E_b = (\cup_{i \in \Nat} A_i) \cap (\cup_{i \in \Nat} B_i) = \cup_{i \in \Nat} (A_i \cap B_i) = E_a \cap E_b.
  \]


Then,
  \begin{align}
   &\mu(E_a \cap E_b) \notag \\
   &= \mu((\cup_{i \in \Nat} A_i) \cap (\cup_{i \in \Nat} B_i)) \notag \\
   &= \mu_a(\cup_{i \in \Nat} A_i) \cdot \mu_b(\cup_{i \in \Nat} B_i) \notag \\
   &= \mu_a(\lim_{n \to \infty} \cup_{i \in [n]} A_i) \cdot \mu_b(\lim_{n \to \infty} \cup_{i \in [n]} B_i) \notag \\
   &= \lim_{n \to \infty} \mu_a(\cup_{i \in [n]} A_i) \cdot \lim_{n \to \infty} \mu_b( \cup_{i \in [n]} B_i) \tag{By continuity of $\mu_a$ and $\mu_b$} \\
   &= \lim_{n \to \infty} \mu_a(\cup_{i \in [n]} A_i) \cdot \mu_b( \cup_{i \in [n]} B_i) \tag{$\dagger$} \\
   &= \lim_{n \to \infty} \sum_{i \in [n]}  \mu_b ( B_i) \cdot \mu_a(A_i) \tag{By~\cref{eq:finite-add-alpha}} \\
   &= \sum_{i \in \Nat}  \mu_b ( B_i) \cdot \mu_a(A_i),
  \end{align}
where ($\dagger$) holds because that the product of limits equals to the limit of
  the product when both $\lim_{n \to \infty} \mu_a(\cup_{i \in [n]} A_i)$ and
  $\lim_{n \to \infty} \mu_b( \cup_{i \in [n]} B_i)$ are finite.
  Thus, we proved countable additivity as well.



 \item
  Next we show that we can \textbf{extend $\mu$ to a measure on
  $\sigmaF_a \punion \sigmaF_b$}.

  So far, we proved that $\mu$ is a sub-additive measure on the
  $\set{E_a \cap E_b | E_a \in \sigmaF_a, E_b \in \sigmaF_b}$,
  which forms a $\pi$-system.
  By a known theorem in probability theory
  (\eg \cite[Corollary 2.5.4]{rosenthal2006first}),
  we can extend a sub-additive measure on a
   $\pi$-system to the \salgebra{} it generates if the $\pi$-system is
   a semi-algebra.
   Thus, we can extend $\mu$ to a measure on $\closure{\{E_a \cap E_b \mid E_a \in \sigmaF_a,\ E_b \in \sigmaF_b\}}$ if we can prove $J = \{E_a \cap E_b \mid E_a \in \sigmaF_a,\ E_b \in \sigmaF_b\}$ is a semi-algebra.

   \begin{itemize}
    \item $J$ contains $\emptyset$ and $\Outcomes$: trivial.
    \item $J$ is closed under finite intersection:
     $(E_a \cap E_b) \cap (E'_a \cap E'_b) = (E_a \cap E'_a) \cap (E_b \cap E'_b)$, where $E_a \cap E'_a \in \sigmaF_a$, and $E_b \cap E'_b \in \sigmaF_b$.
    \item The complement of any element of $J$ is equal to a finite disjoint
     union of elements of $J$:
     \begin{align*}
      (E_a \cap E_b)^C &= E_a^C \cup E_b^C \\
             &= (E_a^C \cap \Outcomes) \disjunion (E_a \cap E_b^C)
     \end{align*}
     where $E_a^C, E_a \in \sigmaF_a$, and $E_b^C, \Outcomes \in \sigmaF_b$.

   \end{itemize}

     As shown in~\cite{lilac},
\begin{align}
      \closure{\{E_a \cap E_b \mid E_a \in \sigmaF_a, E_b \in \sigmaF_b\}}
     = \sigmaF_a \punion \sigmaF_b
     \end{align}
Thus, the extension of $\mu$ is a measure on $\sigmaF_a \punion \sigmaF_b$.

    \item Last, we show that $\mu$ is a \textbf{probability measure} on
     $\sigmaF_a \punion \sigmaF_b$:
     $\mu(\Outcomes) = \mu_a(\Outcomes) \cdot \mu_b(\Outcomes) = 1$.
    \qedhere
 \end{itemize}
 \end{proof}

 \begin{lemma}
  \label{lemma:fibre-prod-exists}
    Consider two probability spaces
  $(\sigmaF_1, \mu_1), (\sigmaF_2, \prob_2) \in \ProbSp(\Outcomes)$,
  and some other probability space $(\Full{A}, \prob)$ and kernel $\krnl$
  such that $\prob_1 = \bind(\prob, \krnl)$.

  Then, the independent product
  $(\sigmaF_1, \mu_1) \iprod (\sigmaF_2, \mu_2)$
  exists if and only if
   for any $a \in \psupp(\prob)$,
   the independent product
   $(\sigmaF_1, \krnl(a)) \iprod (\sigmaF_2, \prob_2)$ exists.
When they both exist,
  \[
   (\sigmaF_1, \prob_1) \iprod (\sigmaF_2, \prob_2)
  = (\sigmaF_1 \punion \sigmaF_2,
     \bind(\prob, \fun a. \krnl(a) \iprod \prob_2))
  \]
 \end{lemma}



 \begin{proof}
    We first show the backwards direction.
  By~\cref{lemma:indep-prod-exists},
  for any $a \in \psupp(\prob)$, to show that the independent product
  $(\sigmaF_1, \krnl(a)) \iprod (\sigmaF_1, \prob_1)$ exists,
  it suffices to show that for any $E_1 \in \sigmaF_1, E_2 \in \sigmaF_2$
  such that $E_1 \cap E_2 = \emptyset$,
  $\krnl(a)(E_1) \cdot \prob_2(E_2) = 0$.

  Fix any such $E_1, E_2$,
  because $(\sigmaF_1, \prob_1) \iprod (\sigmaF_2, \prob_2)$ is defined,
  we have $\prob_1(E_1) \cdot \prob_2(E_2) = 0$, then either $\prob_1(E_1) = 0$
  or $\prob_2(E_2) = 0$.
\begin{itemize}
   \item If $\prob_1(E_1) = 0$:
   Recall that
\[
    \prob_1(E_1) = \bind(\prob, \krnl)(E_1)
    = \sum_{a \in A} \prob(a) \cdot \krnl(a) (E_1)
    = \sum_{\mathclap{a \in \psupp(\prob)}} \prob(a) \cdot \krnl(a) (E_1)
   \]
Because all $\prob(a) > 0$ and $\krnl(a) (E_1) \geq 0$ for all $a \in \psupp(\prob)$
   $\sum_{a \in \psupp(\prob)} \prob(a) \cdot \krnl(a) (E_1) = 0$ implies that
   $\prob(a) \cdot \krnl(a) (E_1) = 0$ for all $a \in \psupp(\prob)$.
   Thus, for all $a \in \psupp(\prob)$, it must $\krnl(a) (E_1) = 0$.
   Therefore, $\krnl(a)(E_1) \cdot \prob_2(E_2) = 0$ for all  $a \in \psupp(\prob)$
   with this $E_1, E_2$.

  \item If $\prob_2(E_2) = 0$, then it is also clear that
    $\krnl(a)(E_1) \cdot \prob_2(E_2) = 0$ for all  $a \in \psupp(\prob)$.
  \end{itemize}
Thus, we have $\krnl(a)(E_1) \cdot \prob_2(E_2) = 0$ for any
  $E_1 \cap E_2 = \emptyset$ and $a \in \psupp(\prob)$.
  By~\cref{lemma:indep-prod-exists},
  the independent product $(\sigmaF_1, \krnl(a)) \iprod (\sigmaF_1, \prob_1)$ exists.

    For the forward direction:
    for any $E_1 \in \sigmaF_1$ and $E_2 \in \sigmaF_2$ such that $E_1 \cap E_2 = \emptyset$,
   the independent product $(\sigmaF_1, \krnl(a)) \iprod (\sigmaF_2, \mu_2)$ exists implies that
   \begin{align*}
     \krnl(a) (E_1) \cdot \mu_2(E_2) = 0.
   \end{align*}
Thus,
   \begin{align*}
    \mu_1(E_1) \cdot \mu_2(E_2)
    &= \bind(\mu, \krnl)(E_1)  \cdot \mu_2(E_2) \\
    &= \left\lgroup\sum_{a \in A} \mu(a) \cdot \krnl(a) (E_1) \right\rgroup  \cdot \mu_2(E_2) \\
    &= \sum_{a \in A_{\mu}} \mu(a) \cdot \left(\krnl(a) (E_1)  \cdot \mu_2(E_2) \right) \\
    &= \sum_{a \in A_{\mu}} \mu(a) \cdot 0
    = 0
   \end{align*}

   Thus, by~\cref{lemma:indep-prod-exists},
   the independent product $(\sigmaF_1, \mu_1) \iprod (\sigmaF_2, \mu_2)$ exists.
For any $E_1 \in \sigmaF_1$ and $E_2 \in \sigmaF_2$,
\begin{align*}
   &\bind(\prob, \fun a. \krnl(a) \iprod \prob_2 ) (E_1 \inters E_2)\\
  &= \sum_{\mathclap{a \in \psupp(\prob)}}
     \prob(a) \cdot
     \left(\krnl(a) \iprod  \prob_2\right)(E_1 \inters E_2) \\
  &= \sum_{\mathclap{a \in \psupp(\prob)}}
      \prob(a) \cdot \krnl(a)(E_1) \cdot \prob_2(E_2) \\
  &= \left\lgroup
     \sum_{a \in \psupp(\prob)}
      \prob(a) \cdot \krnl(a)(E_1)
    \right\rgroup \cdot \prob_2(E_2) \\
  &= \bind(\prob, \krnl)(E_1) \cdot \prob_2(E_2) \\
  &= \prob_1(E_1) \cdot \prob_2(E_2) \\
  &= (\prob_1 \iprod \prob_2)(E_1 \inters E_2)
  \end{align*}
Thus,
  $
   (\sigmaF_1, \prob_1) \iprod (\sigmaF_2, \prob_2)
  = (\sigmaF_1 \punion \sigmaF_2,
     \bind(\prob, \fun a. \krnl(a) \iprod \prob_2)).
  $
 \end{proof}

