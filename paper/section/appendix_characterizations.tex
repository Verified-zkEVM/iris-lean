\section{Characterizations of \SuperCond\ and Relational Lifting}
\label{sec:appendix:alt-cond-rl}

Interestingly, it is possible to characterize the conditioning modality
using the other connectives of the logic.
\begin{proposition}[Alternative Characterization of \Supercond]
\label{prop:cond-as-wand}
  The following is a logically equivalent characterization
  of the \supercond\ modality:
  \begin{align*}
    \CMod{\prob} K &\lequiv
    \begin{array}[t]{@{}r@{\,}l@{}}
      \E \m{\salg}, \m{\prob}, \m{\permap}, \m{\krnl}.
        & \Own{\m{\salg}, \m{\prob}, \m{\permap}} *
        \pure{\forall i\in I\st
        \m{\mu}(i) = \bind(\prob, \m{\krnl}(i))}
     \\ & * \;
     \A v \in \psupp(\prob).
     \Own{\m{\sigmaF}, \m{\krnl}(I)(v), \m{\permap}}
     \wand
     K(v)
    \end{array}
\end{align*}
\end{proposition}


\begin{proof}
	In the following, we sometimes abbreviate
	${\forall i\in I\st
                \m{\mu}(i) = \bind(\prob, \m{\krnl}(i))}$
	by writing just ${\m{\mu} = \bind(\prob, \m{\kappa})}$.

	We start with the embedding:

	\begin{align*}
				&\begin{array}[t]{@{}r@{\,}l@{}}
        \E \m{\salg}, \m{\prob}, \m{\permap}, \m{\krnl}.
        & \Own{\m{\salg}, \m{\prob}, \m{\permap}} *
        \pure{\forall i\in I\st
                \m{\mu}(i) = \bind(\prob, \m{\krnl}(i))}
        \\ & * \;
        \A a \in \psupp(\prob).
          \Own{\m{\sigmaF}, \m{\krnl}(I)(a), \m{\permap}}
          \wand
            K(a)
 			 \end{array}\\
{}\lequiv {} &
				\fun r.
        \exists \m{\sigmaF}, \m{\mu}, \m{\permap}, \m{\kappa}.
        \big(\Own{\m{\sigmaF}, \m{\mu'}, \m{\permap}} \sepand
				\pprop{\m{\mu} = \bind(\prob, \m{\kappa})} \sepand \\
					&\qquad \qquad
        (\forall a \in \psupp(\prob). \Own{\m{\sigmaF}, \m{\kappa} a, \m{\permap}} \sepimp K(a)) \big) (r) \\
{}\lequiv {} &
			 \fun r.
				\exists \m{\sigmaF}, \m{\mu},  \m{\permap},  \m{\kappa},
				\m{\sigmaF_1}, \m{\mu_1},  \m{\permap_1},
				\m{\sigmaF_2}, \m{\mu_2}, \m{\permap_2},
				\m{\sigmaF_3}, \m{\mu_3}, \m{\permap_3}, \\
				&\qquad \qquad
				r \extOf (\m{\sigmaF_1}, \m{\mu_1}, \m{\permap_1}) \raOp
				(\m{\sigmaF_2}, \m{\mu_2}, \m{\permap_2}) \raOp (\m{\sigmaF_3}, \m{\mu_3}, \m{\permap_3}) \land\\
				&\qquad \qquad
				(\m{\sigmaF_1}, \m{\mu_1}, \m{\permap_1}) \extOf (\m{\sigmaF}, \m{\mu}, \m{\permap})  \land
				\pprop{\m{\mu} = \bind(\prob, \m{\kappa})} \land \\
				&\qquad \qquad
				(\forall a \in \psupp(\prob). \forall r_1, r_2 \st
				r_1 \raOp (\m{\sigmaF_3}, \m{\mu_3}, \m{\permap_3}) = r_2 \land
				r_1 \extOf (\m{\sigmaF}, \m{\kappa} a, \m{\permap})
				\implies K(a)(r_2))\\
{}\lequiv {} &
				\fun r.
				\exists \m{\sigmaF}, \m{\mu}, \m{\permap},
				\m{\sigmaF_3}, \m{\mu_3}, \m{\permap_3}, \m{\kappa}. \\
				& \qquad \qquad
				r \extOf
				(\m{\sigmaF}, \m{\mu}, \m{\permap})  \raOp (\m{\sigmaF_3}, \m{\mu_3}) \land
				\pprop{\m{\mu} = \bind(\prob, \m{\kappa})} \land \\
				& \qquad \qquad
				(\forall a \in \psupp(\prob). \forall r_1, r_2 \st
				r_1 \raOp (\m{\sigmaF_3}, \m{\mu_3}, \m{\permap_3}) = r_2 \land
				r_1 \extOf (\m{\sigmaF}, \m{\kappa} a, \m{\permap})
				\implies K(a)(r_2))
	\end{align*}
For the last equivalence, the forward direction holds because
\begin{align*}
		r &\extOf (\m{\sigmaF_1}, \m{\mu_1}, \m{\permap_1}) \raOp
		(\m{\sigmaF_2}, \m{\mu_2}, \m{\permap_2}) \raOp (\m{\sigmaF_3}, \m{\mu_3}, \m{\permap_3})\\
			&\extOf  (\m{\sigmaF_1}, \m{\mu_1}, \m{\permap_1}) \raOp (\m{\sigmaF_3}, \m{\mu_3}, \m{\permap_3}) \\
			&\extOf  (\m{\sigmaF}, \m{\mu}, \m{\permap}) \raOp (\m{\sigmaF_3}, \m{\mu_3}, \m{\permap_3}).
	\end{align*}
The backward direction holds because we can pick
	$(\m{\sigmaF_1}, \m{\mu_1}, \m{\permap_1}) = (\m{\sigmaF}, \m{\mu}, \m{\permap})$,
	$(\m{\sigmaF_2}, \m{\mu_2})$ be the trivial probability space on $\store$ and
	$\m{\permap_2} = \fun \wtv. 0$.

	\begin{itemize}
		\item To show that the embedding implies the original assertion $\CMod{\prob} K $,
	we start with $\m{\mu}(i) \iprod \m{\mu_3}(i)$. For any $i$, we have
	$\m{\mu}(i) = \bind(\prob, \m{\kappa}(i))$, and thus
\begin{align*}
											\m{\mu }(i) \iprod \m{\mu_3}(i)
											&= \bind(\mu, \m{\kappa}(i)) \iprod \m{\mu_3}(i).
										\end{align*}
According to~\cref{lemma:fibre-prod-exists},
										$\m{\mu }(i) \iprod \m{\mu_3}(i)$ is defined implies that
										$\m{\kappa}(i)(a) \iprod  \m{\mu_3}(i)$ is defined for any $a \in $.
										Furthermore,
\begin{align*}
											\m{\mu }(i) \iprod \m{\mu_3}(i)
											&= \bind(\mu, \fun a. \m{\kappa}(i)(a) \iprod  \m{\mu_3}(i) )
										\end{align*}
										We abbreviate the hyperkernel $\m[i: \fun a. \m{\kappa}(i)(a) \iprod  \m{\mu_3}(i) \mid i \in I]$
										as $\m{\kappa'} $.
For any $a \in \psupp(\prob)$,
										the assertion
										\[
											\forall a \in \psupp(\prob). \forall r_1, r_2.
											r_1 \iprod (\m{\sigmaF_3}, \m{\mu_3}, \m{\permap_3}) = r_2 \land
											r_1 \extOf (\m{\sigmaF}, \m{\kappa}(I)a, \m{\permap})
											\implies K(a)(r_2)
										\]
										applies with the specific case
										$r_1 =  (\m{\sigmaF }, \m{\kappa}(I)(a), \m{\permap})$,
										gives us
										\[
K(a) ((\m{\sigmaF}, \m{\kappa}(I)(a), \m{\permap}) \raOp (\m{\sigmaF_3} , \m{\mu_3}, \m{\permap_3} )])
										\]
										By the definition of composition in our resource algebra,
										we have that $K(a)$ holds on $(\m{\sigmaF} \punion \m{\sigmaF_3},  \m{\kappa}'(I)(a), \m{\permap} + \m{\permap_3})$.

										For any $r$,
										\begin{itemize}
											\item If $\raValid(r)$, then there exists $\m{\sigmaF'}, \m{\mu'}, \m{\permap'}$ such that
												$r = (\m{\sigmaF'}, \m{\mu'}, \m{\permap'})$.
												Note that
												\begin{align*}
													r = (\m{\sigmaF'}, \m{\mu'}, \m{\permap'})
													&\extOf (\m{\sigmaF}, \m{\mu}, \m{\permap}) \raOp  (\m{\sigmaF_3} , \m{\mu_3}, \m{\permap_3})
													=  (\m{\sigmaF} \punion \m{\sigmaF_3} , \m{\mu} \iprod \m{\mu_3}, \m{\permap} + \m{\permap_3})
												\end{align*}

												By~\cref{lemma:bind-extend}, $\m{\mu} \iprod \m{\mu_3} = \bind(\mu, \m{\kappa'})$ implies
												that there exists $\m{\kappa''}$ such that
												$\m{\mu}(i) = \bind(\mu, \m{\kappa''}(i)) $, and that for any $a \in \psupp{\mu}$,
												$(\m{\sigmaF} \punion \m{\sigmaF_3}, \m{\kappa'}(I)(a)) \extTo (\m{\sigmaF'}, \m{\kappa''}(I)(a)) $.
												Thus, by monotonicity with respect to the extension order,
												that would imply 	$K(a)$ holds on $(\m{\sigmaF'}, \m{\kappa''}(I)(a), \m{\permap'})$.
												And $K(a)$ holds on $(\m{\sigmaF'}, \m{\kappa''}(I)(a), \m{\permap'})$ for any
												$a \in \psupp{\mu}$ together with
												$\m{\mu}(i) = \bind(\mu, \m{\kappa''}(i))$ implies that $r$ satisfy the original assertion
												of conditioning modality.
\item If not $\raValid(r)$, then $r$ satisfies any assertions, so $r$ satisfy the original
												assertion of conditioning modality.
										\end{itemize}

									\item
										To show the other direction that having the original assertion implies the
										embedded assertion.
										Assume $\CMod{\mu} K(r)$,
										that is,
										\begin{align*}
											    \begin{array}[t]{@{}r@{\,}l@{}}
    													\E \m{\sigmaF}, \m{\mu}, \m{\permap}, \m{\krnl}.
     													 & (\m{\sigmaF}, \m{\mu}, \m{\permap}) \raLeq r
     													 \land
     													   \forall i\in I\st
     													     \m{\mu}(i) = \bind(\prob, \m{\krnl}(i))
     													 \\ & \land \;
     													   \forall v \in \psupp(\prob).
     													     K(v)(\m{\sigmaF}, \m{\krnl}(I)(v), \m{\permap})
    											\end{array}
													(r)
										\end{align*}

										To show that $r$ also satisfy the embedding,
										we pick the witness for the existential quantifier as follows:
										let $(\m{\sigmaF_3}, \m{\mu_3})$ be the trivial probability space on
										$\Store$;
										let $\m{\permap_3} = \fun \wtv. 0$;
										pick $(\m{\sigmaF}_{\text{embd}}, \m{\mu}_{\text{embd}}, \m{\permap}_{\text{embd}})$
										be the $(\m{\sigmaF}_{\text{orig}}, \m{\mu}_{\text{orig}}, \m{\permap}_{\text{orig}})$
										that witness $\CMod{\mu} K (r)$,
										and $\m{\kappa}_{\text{embd}} = \m{\kappa}_{\text{orig}}$.

										Then:
										\begin{itemize}
											\item First we show
												\begin{align*}
													r
													&\raGeq
													(\m{\sigmaF}_{\text{orig}}, \m{\mu}_{\text{orig}}, \m{\permap}_{\text{orig}}) \\
													&= (\m{\sigmaF}_{\text{orig}}, \m{\mu}_{\text{orig}}, \m{\permap}_{\text{orig}}) \raOp (\m{\sigmaF_3}, \m{\mu_3}, \m{\permap_3}) \\
													&= (\m{\sigmaF}_{\text{embd}}, \m{\mu}_{\text{embd}}, \m{\permap}_{\text{embd}}) \raOp (\m{\sigmaF_3}, \m{\mu_3}, \m{\permap_3})
												\end{align*}
\item $\m{\mu}_{\text{orig}} = \bind(\mu, \m{\kappa}_{\text{orig}} (I) (a))$ 
											implies $\m{\mu}_{\text{embd}} = \bind(\mu, \m{\kappa}_{\text{embd}} (I) (a))$.
\item For any $r_1, r_2$,
										\[
											r_1 \raOp (\m{\sigmaF_3}, \m{\mu_3}, \m{\permap_3}) = r_2 \land
											r_1 \extOf (\m{\sigmaF}_{\text{embd}}, \m{\kappa}_{\text{embd}} (I)(a), \m{\permap}_{\text{embd}})
										\]
										implies that $r_2 = r_1 \extOf (\m{\sigmaF}_{\text{orig}}, \m{\kappa}_{\text{orig}} (I)(a), \m{\permap}_{\text{orig}})$.
										By the assumption that the orig assertion holds,
										we have $K(a) (\m{\sigmaF}_{\text{orig}}, \m{\kappa}_{\text{orig}} (I)(a), \m{\permap}_{\text{orig}})$,
                which implies $K(a)(r_2)$.
										\end{itemize}

										Therefore, $r$ also satisfy the embedding.\qedhere
								\end{itemize}
									\end{proof}






\begin{lemma}[Alternative Characterization of Relational Lifting]\label{lemma:cpl-is-cpl}
Given a relation $R$ over $\Store_1 \times \Store_2$,
	then $\cpl{R}(\m{\sigmaF}, \m{\mu}, \m{\permap})$ holds
iff there exists $\wh{\mu}$ over $\m{\sigmaF}(1) \otimes \m{\sigmaF}(2)$ such that
	$\wh{\mu}(R) = 1$,
  $\wh{\mu} \circ \pi_1^{-1} = \m{\mu}(1)$,
  and $\wh{\mu} \circ \pi_2^{-1} = \m{\mu}(2)$.
\end{lemma}

\begin{proof}
	We first unfold the definition of the coupling modality:
	\begin{align*}
		&  \cpl{R} (\m{\sigmaF}, \m{\mu}, \m{\permap})\\
{} \iff {} &
		\left( \exists \mu. \mu(R) = 1 \land
		\CC{\prob} \m{v}.
		\LAnd_{\ip{x}{i} \in X}
    \sure{\ip{x}{i} = \m{v}(\ip{x}{i})} \right)
    (\m{\sigmaF}, \m{\mu}, \m{\permap})
    \\
{} \iff {} &
		\exists \mu. \mu(R) = 1 \land
		\left( \CC{\prob} \m{v}.
		\LAnd_{\ip{x}{i} \in X}
    \sure{\ip{x}{i} = \m{v}(\ip{x}{i})} \right)
    (\m{\sigmaF}, \m{\mu}, \m{\permap})
    \\
{} \iff {}&
		\exists \prob. \prob(R) = 1 \land
    \exists \m{\sigmaF'}, \m{\mu'}, \m{\permap'}, \m{\krnl} \st
      (\m{\sigmaF}, \m{\mu}, \m{\permap}) \raGeq (\m{\sigmaF'}, \m{\mu'}, \m{\permap'}) \land
      \forall i \in I.\bind(\prob, \m{\krnl}(i) )= \m{\prob}(i) {}\land{} \\
		&\qquad \qquad 	\forall \m{v}\in \psupp(\prob) \st
    \left( \LAnd_{\ip{x}{i} \in X} \sure{\ip{x}{i} = \m{v}(\ip{x}{i})} \right)
    (\m{\sigmaF'}, \m{\mu'}, \m{\permap'})\\
{} \iff {} &
		\exists \prob. \prob(R) = 1 \land
		\exists \m{\sigmaF'}, \m{\mu'}, \m{\permap'}, \m{\krnl} \st
       (\m{\sigmaF}, \m{\mu}, \m{\permap}) \raGeq (\m{\sigmaF'}, \m{\mu'}, \m{\permap'}) \land
			\forall i \in I.\bind(\prob, \m{\krnl}(i) )= \prob_i {}\land{}
		  \\
			&\qquad \qquad \qquad \forall \m{v}\in \psupp(\prob) \st
      \LAnd_{i \in \{1,2\}}
		  \LAnd_{\ip{x}{i} \in X}
        \pure{\almostM{(\ip{x}{i} = \m{v}(\ip{x}{i}))}{(\sigmaF'(i), \m{\krnl}(i)(\m{v}))} }
        \land \\
      & \qquad \qquad \qquad \qquad \qquad \qquad \qquad  \pure{\m{\krnl}(i)(\m{v}) \circ \inv{(\ip{x}{i} = \m{v}(\ip{x}{i}))} = \delta_{\True} }
	\end{align*}

	Now, to show that
	$\cpl{R} (\m{\sigmaF}, \m{\mu}, \m{\permap})$
  implies there exists $\wh{\mu}$ over $\m{\sigmaF}(1) \otimes \m{\sigmaF}(2)$ such that
	$\wh{\mu}(R) = 1$,
	$\wh{\mu} \circ \pi_1^{-1} = \mu_1$,
	and $\wh{\mu} \circ \pi_2^{-1} = \mu_2$, we define $\wh{\mu}$ over
	$\sigmaF_1 \otimes \sigmaF_2$  as
	$\bind(\mu, \fun \m{v}. \m{\krnl}(1)(\m{v}) \pprod \m{\krnl}(2)(\m{v}))$.
	Then,
\begin{align*}
		\wh{\mu}(R) &= \bind(\mu, \m{v}. \m{\krnl}(1)(\m{v}) \pprod \m{\krnl}(2)(\m{v}))(R) \\
								&= \sum_{\mathclap{\m{v} \in \psupp(\prob)}} \mu(\m{v}) \cdot \left( \m{\krnl}(1)(\m{v}) \pprod \m{\krnl}(2)(\m{v}) \right)(R)
	\end{align*}
Since $\mu(R) = 1$, then for all $\m{v} \in \psupp_{\prob}$, and $\m{v} \in R$, by additivity:
	\begin{align}
		\left( \m{\krnl}(1)(\m{v}) \pprod \m{\krnl}(2)(\m{v}) \right)(R) &\geq \left( \m{\krnl}(1)(\m{v}) \pprod \m{\krnl}(2)(\m{v}) \right)(\m{v}) \notag \\
   &=  \m{\krnl}(1)(\m{v}) (\pi_1 (\m{v})) \cdot \m{\krnl}(2)(\m{v}) (\pi_2 (\m{v})) \label{eq:cpladequacy:explain}\\
	 &= \m{\krnl}(1)(\m{v})(\LAnd_{\ip{x}{1} \in X} \ip{x}{1} = \m{v}(\ip{x}{1})) \cdot \m{\krnl}(2)(\m{v})(\LAnd_{\ip{x}{2} \in X} \ip{x}{2} = \m{v}(\ip{x}{2})) \notag \\
	 &= 1		  \notag
	\end{align}
where~\cref{eq:cpladequacy:explain} is because $\m{v}$ as a singleton can also be thought of as a Cartesian product.
Thus,
	\begin{align*}
		\wh{\mu}(R) &= \sum_{\mathclap{\m{v} \in \psupp(\prob)}} \mu(\m{v}) \cdot 1 \cdot 1
								= \sum_{\mathclap{\m{v} \in \psupp(\prob)}} \mu(\m{v})
								= 1
	\end{align*}
Meanwhile, for $E_i \in \sigmaF_i$, and
  let $j = 2$ if $i = 1$ and $j = 1$ if $i = 2$,
\begin{align}
		(\wh{\mu} \circ \pi_i^{-1}) (E_i)
&= \wh{\mu} (E_i \times \Store_j) \\
		&= \bind(\mu, \fun \m{v} . \m{\krnl}(1)(\m{v}) \pprod \m{\krnl}(2)(\m{v}))(  E_i \times \Store_j) \notag \\
		&= \sum_{\mathclap{\m{v} \in \psupp{\mu}}} \left( \m{\krnl}(1)(\m{v}) \pprod \m{\krnl}(2)(\m{v}) \right) (E_i \times \Store_j) \notag \\
		&= \sum_{\mathclap{\m{v} \in \psupp{\mu}}} \m{\krnl}(i)(\m{v})(E_i) \cdot \m{\krnl}(j)(\m{v}) (\Store_j) \notag \\
		&= \sum_{\mathclap{\m{v} \in \psupp{\mu}}} \m{\krnl}(i)(\m{v})(E_i) \cdot 1 \notag \\
		&= \bind(\mu, \fun \m{v} . \m{\krnl}(i)) (E_i) \notag \\
		&= \mu_i(E_i) \notag
	\end{align}
Thus, we complete the forward direction.

	For the backwards direction,
	we pick $\mu = \wh{\mu}$,
  $\m{\mu'}(i) = \pi_{i}\wh{\mu}$ ($\m{\sigmaF'}(i)$ accordingly),
  $\m{\permap'}=\m{\permap'}$,
	and $\m{\krnl}(i) = \fun \m{v}. \dirac{\pi_{\ip{x}{i} \in X} \m{v}}$.
Then,
\begin{align*}
		\bind(\mu, \m{\krnl}(i))
		&= \bind(\wh{\mu}, \fun \m{v}. \dirac{\pi_{\ip{x}{i} \in X} \m{v}})\\
		&= \wh{\mu} \circ \pi_i^{-1} = \m{\mu}(i)
	\end{align*}
Also, by definition,
	$\m{k}(i)(\m{v}) = \dirac{\pi_{\ip{x}{i} \in X} \m{v}}$.
Thus,
  $\almostM{\{\ip{x}{i} = \m{v}(\ip{x}{i})\}}{(\m{\sigmaF}(i), \m{\krnl}(i)(\m{v}))}$ and
  $\m{\krnl}(i)(\m{v}) \circ \inv{\ip{x}{i} = \m{v}(\ip{x}{i})} = \dirac{\True}$.
\end{proof}

