\subsection{One-time Pad (Relational)}
\label{sec:appendix:examples:onetimerel}
\label{sec:appendix:ex:otp-rel}

  To wrap up the proof of \Cref{sec:overview}
we first observe that the assertion~$P$
of~(\ref{ex:xor:start}) can be easily obtained by
using the WP rules for assignments and sequencing, proving:
\[
    \True\withp{\m{\permap}}
    \proves
    \WP {\m<
      \I1: \code{encrypt()},
      \I2: \code{c:~Ber(1/2)}
    >}*{
      \begin{pmatrix}
        \distAs{\Ip{k}{1}}{\Ber{\onehalf}}
          *
        \distAs{\Ip{m}{1}}{\Ber{p}}
          *
        \distAs{\Ip{c}{2}}{\Ber{\onehalf}}
          * {}\\
        \sure{\Ip{c}{1} = \Ip{k}{1} \xor \Ip{m}{1}}
      \end{pmatrix}
      \withp{\m{\permap}}
    }
\]
where $\m{\permap} = \m[\Ip{k}{1}:1,\Ip{m}{1}:1,\Ip{c}{1}:1,\Ip{c}{2}:1]$
(\ie we have full permissions on the variables we modify).


We can prove the entailment:
\[
  \CC{\Ber{p}} v.
  \left(
    \sure{\Ip{m}{1}=v} *
    \begin{pmatrix}
    \distAs{\Ip{k}{1}}{\Ber{\onehalf}}
    \\ {}*
    \distAs{\Ip{c}{2}}{\Ber{\onehalf}}
    \end{pmatrix}
  \right)
  \proves
  \CC{\Ber{p}} v.
    \left(
      \sure{\Ip{m}{1}=v}
      *
      \begin{cases}
        \cpl{ \Ip{k}{1} = \Ip{c}{2} }     \CASE v=0 \\
        \cpl{ \Ip{k}{1} = \neg\Ip{c}{2} } \CASE v=1
      \end{cases}
    \right)
\]
by using \ref{rule:c-cons}, which asks us to prove that the
two assertions inside the conditioning are in the entailment relation
for each value of~$v$.
This leads to these two cases:
\begin{align*}
  \sure{\Ip{m}{1}=0}
  * \distAs{\Ip{k}{1}}{\Ber{\onehalf}}
  * \distAs{\Ip{c}{2}}{\Ber{\onehalf}}
  & \proves
  \sure{\Ip{m}{1}=0} * \cpl{ \Ip{k}{1} = \Ip{c}{2} }
\\
  \sure{\Ip{m}{1}=1}
  * \distAs{\Ip{k}{1}}{\Ber{\onehalf}}
  * \distAs{\Ip{c}{2}}{\Ber{\onehalf}}
  & \proves
  \sure{\Ip{m}{1}=1} * \cpl{ \Ip{k}{1} = \neg\Ip{c}{2} }
\end{align*}
which are straightforward consequences of the two couplings we proved
in~\eqref{ex:xor-two-cpl}.

Finally, the assignment to \p{c} in \p{encrypt} generated the fact
$\sure{\Ip{c}{1} = \Ip{k}{1} \xor \Ip{m}{1}}$.
By routine propagation of this fact
(using \ref{rule:c-frame} and \ref{rule:sure-merge})
we can establish:
\begin{eqexplain}
  &
  \CC{\Ber{p}} v.
    \left(
      \sure{\Ip{m}{1}=v}
      *
      \begin{cases}
        \cpl{ \Ip{k}{1} = \Ip{c}{2} }     \CASE v=0 \\
        \cpl{ \Ip{k}{1} = \neg\Ip{c}{2} } \CASE v=1
      \end{cases}
    \right)
    *
    \sure{\Ip{c}{1} = \Ip{k}{1} \xor \Ip{m}{1}}
\whichproves
  \CC{\Ber{p}} v.
    \left(
      \sure{\Ip{m}{1}=v}
      *
      \begin{cases}
        \cpl{ \Ip{k}{1} = \Ip{c}{2} } * \sure{ \Ip{c}{1} = \Ip{k}{1} \xor 0 }
          \CASE v=0 \\
        \cpl{ \Ip{k}{1} = \neg\Ip{c}{2} } * \sure{ \Ip{c}{1} = \Ip{k}{1} \xor 1 }
          \CASE v=1
      \end{cases}
    \right)
\whichproves
  \CC{\Ber{p}} v.
    \left(
      \sure{\Ip{m}{1}=v}
      *
      \begin{cases}
        \cpl{ \Ip{c}{1} = \Ip{c}{2} } \CASE v=0 \\
        \cpl{ \Ip{c}{1} = \Ip{c}{2} } \CASE v=1
      \end{cases}
    \right)
\whichproves
  \CC{\Ber{p}} v.
        \cpl{ \Ip{c}{1} = \Ip{c}{2} }
\whichproves
  \cpl{ \Ip{c}{1} = \Ip{c}{2} }
  \byrule{rl-merge}
\end{eqexplain}

In particular, the entailments
\begin{align*}
\cpl{ \Ip{k}{1} = \Ip{c}{2} } * \sure{ \Ip{c}{1} = \Ip{k}{1} \xor 0 }
&\proves
\cpl{ \Ip{c}{1} = \Ip{c}{2} }
\\
\cpl{ \Ip{k}{1} = \neg\Ip{c}{2} } * \sure{ \Ip{c}{1} = \Ip{k}{1} \xor 1 }
&\proves
\cpl{ \Ip{c}{1} = \Ip{c}{2} }
\end{align*}
can be proved by applying \ref{rule:rl-sure-merge} and \ref{rule:rl-cons}.
