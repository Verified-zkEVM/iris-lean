\section{Soundness}
\label{sec:appendix:soundness}

\subsection{Soundness of Primitive Rules}
\label{sec:appendix:primitive-rules}


\subsubsection{Soundness of Distribution Ownership Rules}
\begin{lemma}
\label{proof:and-to-star}
  \Cref{rule:and-to-star} is sound.
\end{lemma}

\begin{proof}
  Assume a valid $a\in\Model_I$ is such that
  it satisfies $ P \land Q $.
  This means that for some $(\m{\salg},\m{\prob},\m{\permap}) \raLeq a$, both
    $P(\m{\salg},\m{\prob},\m{\permap})$ and
    $Q(\m{\salg},\m{\prob},\m{\permap})$
  hold.
  We want to prove $(P \ast Q)(a)$ holds.
  To this end, let
  $ (\m{\salg}_1, \m{\prob}_1, \m{\permap}_1) $ and
  $ (\m{\salg}_2, \m{\prob}_2, \m{\permap}_2) $
  be such that
  for every $i \in \idx(P)$:
  \begin{align*}
    \m{\salg}_1(i) &= \m{\salg}(i)
    &
    \m{\salg}_2(i) &= \set{\emptyset, \Outcomes}
    \\
    \m{\prob}_1(i) &= \m{\prob}(i)
    &
    \m{\prob}_2(i) &= \fun \event. \ITE{\event=\Outcomes}{1}{0}
    \\
    \m{\permap}_1(i) &= \m{\permap}(i)
    &
    \m{\permap}_2(i) &= \fun \wtv. 0
  \intertext{
    and for all $i\in I \setminus \idx(P)$:
  }
    \m{\salg}_2(i) &= \m{\salg}(i)
    &
    \m{\salg}_1(i) &= \set{\emptyset, \Outcomes}
    \\
    \m{\prob}_2(i) &= \m{\prob}(i)
    &
    \m{\prob}_1(i) &= \fun \event. \ITE{\event=\Outcomes}{1}{0}
    \\
    \m{\permap}_2(i) &= \m{\permap}(i)
    &
    \m{\permap}_1(i) &= \fun \wtv. 0
  \end{align*}
Clearly, by construction,
  $
    (\m{\salg}_1, \m{\prob}_1, \m{\permap}_1)
    \iprod
    (\m{\salg}_2, \m{\prob}_2, \m{\permap}_2)
    =
    (\m{\salg},\m{\prob},\m{\permap}).
  $
  and
  $P(\m{\salg}_1, \m{\prob}_1, \m{\permap}_1)$.
  Since $\idx(P) \inters \idx(Q) = \emptyset$,
  we also have
  $Q(\m{\salg}_2, \m{\prob}_2, \m{\permap}_2)$.
  Therefore,
  $(P \ast Q)(\m{\salg}, \m{\prob}, \m{\permap})$,
  and so $(P \ast Q)(a)$ by upward closure.
\end{proof} \begin{lemma}
\label{proof:dist-inj}
  \Cref{rule:dist-inj} is sound.
\end{lemma}

\begin{proof}
  Assume a valid $a\in\Model_I$ is such that both
  $ \distAs{E\at{i}}{\prob }(a) $ and
  $ \distAs{E\at{i}}{\prob'}(a) $
  hold.
  Let $ a = (\m{\salg}, \m{\prob}_0,\m{\permap}) $,
  then we know
  $ \prob = \m{\prob}_0 \circ \inv{E\at{i}} = \prob'$,
  which proves the claim.
\end{proof}
 \begin{lemma}
\label{proof:sure-merge}
  \Cref{rule:sure-merge} is sound.
\end{lemma}

\begin{proof}
  The proof for the forward direction is very similar to
  the one for~\cref{rule:sure-eq-inj}.
  For $a \in \Model_I$,
  if $(\sure{E_1\at{i}} \ast \sure{E_2 \at{i}})(a)$.
  Then there exists
  $a_1, a_2$ such that $a_1 \raOp a_2 \raLeq a$ and
  $\sure{E_1 \at{i}}(a_1)$,
  $\sure{E_2 \at{i}}(a_2)$.
  Say $a = (\m{\sigmaF}, \m{\mu}, \m{\permap})$,
  $a_1 = (\m{\sigmaF}_1, \m{\mu}_1, \m{\permap}_1)$
  and $a_2 = (\m{\sigmaF}_2, \m{\mu}_2, \m{\permap}_2)$.
  Then $\sure{E_1\at{i}}(a_1)$ implies that
  \begin{align*}
    \m{\mu}_1 (\inv{E_1\at{i}}(\True)) = 1
  \end{align*}
  And similarly,
  \begin{align*}
    \m{\mu}_2 (\inv{E_2\at{i} }(\True)) = 1
  \end{align*}
  Thus,
  \begin{align*}
    \m{\mu} (\inv{E_1\at{i}}(\True) \cap \inv{E_2\at{i}}(\True) )
    &= \m{\mu}_1 (\inv{E_1\at{i}}(\True))  \cdot \m{\mu}_2 (\inv{E_2\at{i}}(\True))
     = 1.
  \end{align*}
  Hence,
  \begin{align*}
    \m{\mu} (\inv{E_1\at{i} \land E_2\at{i}}(\True) )
    &= \m{\mu} (\inv{E_1\at{i}}(\True) \cap \inv{E_2\at{i}}(\True)) = 1
  \end{align*}
  Thus, $\sure{E_1\at{i} \land E_2\at{j}} (a)$.

  Now we prove the backwards direction:
  Say $a = (\m{\sigmaF}, \m{\mu}, \m{\permap})$.
  if  $\sure{E_1\at{i} \land E_2\at{j}} (a)$,
  then $\m{\mu} (\inv{E_1\at{i} \land E_2\at{i}}(\True)) = 1$,
  and then
  \begin{align*}
    \m{\mu} (\inv{E_1\at{i}}(\True))  &\geq \m{\mu} (\inv{E_1\at{i} \land E_2\at{i}}(\True)) = 1 \\
    \m{\mu} (\inv{E_2\at{i}}(\True))  &\geq \m{\mu} (\inv{E_1\at{i} \land E_2\at{i}}(\True)) = 1
  \end{align*}

  Let $\m{\sigmaF_1} = \closure{\inv{E_1\at{i}}(\True)}$
  and $\m{\sigmaF_2} = \closure{\inv{E_2\at{i}}(\True)}$.
  Then,
  \begin{gather*}
    \sure{E_1\at{i}} (\m{\sigmaF_1}, \constrain{\m{\mu}}{\m{\sigmaF_1}}, \fun \wtv. 0) \\
    \sure{E_2\at{i}} (\m{\sigmaF_2}, \constrain{\m{\mu}}{\m{\sigmaF_2}}, \fun \wtv. 0) \\
    (\m{\sigmaF_1}, \constrain{\m{\mu}}{\m{\sigmaF_1}}, \fun \wtv. 0) \ast  (\m{\sigmaF_2}, \constrain{\m{\mu}}{\m{\sigmaF_2}}, \fun \wtv. 0) \raLeq a
  \end{gather*}
  Thus, $\sure{E_1\at{i}} \ast \sure{E_2\at{i}}$ holds on $a$.
\end{proof} \begin{lemma}
\label{proof:sure-and-star}
  \Cref{rule:sure-and-star} is sound.
\end{lemma}

\begin{proof}
  Assume $a = (\m{\salg}, \m{\prob}, \m{\permap}) \in \Model_I$ and
  $(\sure{E\at{i}} \land P)(a)$ holds.
  We want to show that
  $(\sure{E\at{i}} \ast P)(a)$ holds.
  First note that:
  \begin{align*}
    (\sure{E\at{i}} \land P)(a)
    & \implies \sure{E\at{i}}(a) \land P(a) \\
    & \implies \almostM{E}{(\m{\salg}(i), \m{\prob}(i))}
    \land \m{\prob} \circ \inv{E\at{i}}(\true) = \dirac{\True}
    \land P(a)
  \end{align*}

  Define $\m{\salg}', \m{\permap}_{\aexpr}, \m{\permap}_{P}$ such that,
  for any $j \in I$:
  \begin{align*}
    \m{\salg}'(j) &=
    \begin{cases}
      \set{\emptyset, \Store} \CASE j \neq i\\
      \set{\emptyset, \Store, \inv{E\at{i}}(\True), \Store \setminus
      \inv{E\at{i}}(\True)} \OTHERWISE
    \end{cases}
    \\
    \m{\permap}_{\aexpr}(j) &=
    \begin{cases}
      \fun \wtv.0 \CASE j \ne i \\
      \fun \ip{x}{i}.
        \ITE{\p{x} \in \pvar(\aexpr)}{\m{\permap}(i)(\ip{x}{i})/2}{0}
    \CASE j = i
    \end{cases}
    \\
    \m{\permap}_{P}(j) &=
    \begin{cases}
      \m{\permap}(j) \CASE j \ne i \\
      \fun \ip{x}{i}.
        \ITE{\p{x} \in \pvar(\aexpr)}{\m{\permap}(i)(\ip{x}{i})/2}{\m{\permap}(i)(\ip{x}{i})}
    \CASE j = i
    \end{cases}
  \end{align*}
  By construction, we have
  $
    \m{\permap} = \m{\permap}_{\aexpr} \raOp \m{\permap}_{P}.
  $
  Now let:
  \begin{align*}
    b &= (\m{\salg}', \restr{\m{\prob}}{\m{\salg}'}, \m{\permap}_{\aexpr})
    &
    a' &= (\m{\salg}, \m{\prob}, \m{\permap}_{P})
  \end{align*}
  note that $\raValid(b)$ holds because $\m{\salg}'(i)$ can at best be non-trivial on $\pvar(\aexpr)$.
  The resource $a'$ is also valid, since $\m{\permap}_{P}$ has the same non-zero components as $\m{\permap}$.
  Then
  $\sure{E\at{i}}(b)$ holds because
  $\almostM{E}{( \m{\salg}'(i), \restr{\m{\prob}}{\m{\salg}'}(i) )}$
  and $\restr{\m{\prob}}{\m{\salg}'} \circ \inv{E\at{i}}
  = \m{\prob} \circ \inv{E\at{i}} = \dirac{\True}$.
  By applying~\cref{lemma:indep-prod-exists}, it is easy to show that
  $(\m{\salg}', \restr{\m{\prob}}{\m{\salg}'}) \iprod (\m{\salg}, \m{\prob})$
  is defined and is equal to $(\m{\salg}, \m{\prob})$.
  Therefore,
  $\raValid(b \raOp a)$ and $b \raOp a = a$.
  By the side condition $\psinv(P, \pvar(E\at{i}))$ and the fact that
  $\m{\permap}_{P}$ is a scaled down version of $\m{\permap}$,
  we obtain from $P(a)$ that $P(a')$ holds too.
  This proves that
  $(\sure{E\at{i}} \ast P)(a)$ holds, as desired.
\end{proof}
 \begin{lemma}
\label{proof:prod-split}
  \Cref{rule:prod-split} is sound.
\end{lemma}

\begin{proof}
  For any $(\m{\salg},\m{\prob}, \m{\permap})$ such that
  $(\distAs{(\aexpr_1\at{i}, \aexpr_2\at{i})}{\mu_1 \pprod \mu_2}) (\m{\salg},\m{\prob}, \m{\permap})$,
  by definition, it must
  \begin{align*}
      \E \m{\salg'},\m{\prob'}.
      (\Own{\m{\salg'},\m{\prob'}})(\m{\salg},\m{\prob}, \m{\permap}) *
    \almostM{(\aexpr_1, \aexpr_2)}{(\m{\salg'}(i),\m{\prob'}(i))}
    \land
    \mu_1 \pprod \mu_2 = \m{\prob'}(i) \circ \inv{(\aexpr_1, \aexpr_2)}
    .
  \end{align*}
  We can derive from it that
  \begin{align*}
    \E \m{\salg'},\m{\prob'}, \m{\permap'}.
      & (\m{\salg'},\m{\prob'}) \raLeq (\m{\salg},\m{\prob}, \m{\permap}) * \\
      & \Big( \forall a, b  \in A. \exists L_{a, b},U_{a,b} \in \salg'(i)  \st
        L_{a, b} \subs \inv{(\aexpr_1, \aexpr_2)}(a, b) \subs U_{a, b}
       \land
        \prob'(L_{a, b})=\prob'(U_{a, b}) \land  \\
      &
       \mu_1 \pprod \mu_2 (a, b)
= \m{\prob'}(i) (L_{a, b})
       = \m{\prob'}(i) (U_{a, b})
    \Big)
  \end{align*}
  Also, for any $a, b, a', b' \in A$ such that $a \neq a'$
  or $b \neq b'$, we have
  $L_{a,b}$ disjoint from $L_{a',b'}$ because on $L_{a,b} \inters L_{a',b'}$,
  the random variable $(\aexpr_1, \aexpr_2)$ maps to both
  $(a, b)$ and $(a',b')$.

  Define
  \[
    \m{\salg}_1(i) = \closure{\set{(\Union_{b \in A} L_{a, b} ) \mid a \in A} \union \set{(\Union_{b \in A} U_{a, b})  \mid a \in A}},
  \]
  and similarly define
    \[
      \m{\salg}_2(i) = \closure{\set{(\Union_{a \in A} L_{a, b} ) \mid b \in A} \union \set{(\Union_{a \in A} U_{a, b} )  \mid b \in A}}.
  \]
  Denote $\m{\prob'}$ restricted to $\m{\salg}_1$ as $\m{\prob'}_1$
  and $\m{\prob'}$ restricted to $\m{\salg}_2$ as $\m{\prob'}_2$.


  We want to show that
  $(\m{\salg}_1(i), \m{\prob'}_1(i)) \indepcomb (\m{\salg}_2(i), \m{\prob'}_2(i)) \extTo (\m{\salg'}(i), \m{\prob'}(i))$,
  which boils down to show that for any $\event_1 \in \m{\salg}_1(i)$, any
  $\event_2 \in \m{\salg}_2(i)$,
  \begin{align*}
    \m{\prob'}(\event_1 \inters \event_2) = \m{\prob'}_1(\event_1) \cdot  \m{\prob'}_2(\event_2)
  \end{align*}

      For convenience, we will
      denote $\union_{b \in A} L_{a, b}$ as $L_a$,
      denote $\union_{a \in A} L_{a, b}$ as $L_b$,
      denote $\union_{b \in A} U_{a, b}$ as $U_a$,
      and denote $\union_{a \in A} U_{a, b}$ as $U_b$.




      First, using a standard construction in measure theory proofs,
      we rewrite $\salg_1$ and $\salg_2$ as sigma algebra generated
      by sets of partitions.
      Specifically, $\salg_1$ is equivalent to
      \[
        \closure{\set{\Inters_{a \in S_1} L_a \inters \Inters_{a \in S_2} U_a \setminus (\Union_{a \in A \setminus S_1} L_a \union \Union_{a \in A \setminus S_2} U_a)  \mid S_1, S_2 \subseteq A}}
      \]
      and similarly, $\salg_2$ is equivalent to
      \[
        \closure{\set{\Inters_{b \in T_1} L_b \inters \Inters_{b \in T_2} U_b \setminus (\Union_{b \in A \setminus T_1} L_b \union \Union_{b \in A \setminus T_2} U_b)  \mid T_1, T_2 \subseteq A}}.
      \]
      Thus, by~\cref{lemma:sigma-alg-representation}, any event $\event_1$ in
      $\salg_1$ can be represented by
      \[
        \Dunion_{S_1 \in I_1, S_2 \in I_2}
        \Inters_{a \in S_1} L_a \inters \Inters_{a \in S_2} U_a \setminus (\Union_{a \in A \setminus S_1} L_a \union \Union_{a \in A \setminus S_2} U_a)
      \]
      for some $I_1, I_2 \subseteq \mathcal{P}(A)$, where
      $\mathcal{P}$ is the powerset over $A$.
      Similarly, any event $\event_2$ in $\salg_2$ can be represented by
      \[
        \Dunion_{S_3 \in I_3, S_4 \in I_4}
        \Inters_{b \in S_3} L_b \inters \Inters_{b \in S_4} U_b \setminus (\Union_{b \in A \setminus S_3} L_b \union \Union_{b \in A \setminus S_2} U_b)
      \]
      for some  $I_3, I_4 \subseteq \mathcal{P}(A)$.
      Thus, $\event_1 \inters \event_2$ can be represented as
      \begin{align*}
        \event_1 \inters \event_2
        &=(\Dunion_{S_1 \in I_1, S_2 \in I_2}
        \Inters_{a \in S_1} L_a \inters \Inters_{a \in S_2} U_a \setminus (\Union_{a \in A \setminus S_1} L_a \union \Union_{a \in A \setminus S_2} U_a) ) \\
        &\Inters
        (\Dunion_{S_3 \in I_3, S_4 \in I_4}
        \Inters_{b \in S_3} L_b \inters \Inters_{b \in S_4} U_b \setminus (\Union_{b \in A \setminus S_3} L_b \union \Union_{b \in A \setminus S_2} U_b) )\\
        = & \Dunion_{S_1 \in I_1, S_2 \in I_2, S_3 \in I_3, S_4 \in I_4} (\Inters_{a \in S_1} L_a \inters \Inters_{a \in S_2} U_a \setminus (\Union_{a \in A \setminus S_1} L_a \union \Union_{a \in A \setminus S_2} U_a) ) \\
          &\inters ( \Inters_{b \in S_3} L_b \inters \Inters_{b \in S_4} U_b \setminus (\Union_{b \in A \setminus S_3} L_b \union \Union_{b \in A \setminus S_2} U_b) )
      \end{align*}

      Because $L_{a,b}$ and $L_{a',b'}$ are disjoint as long as not
      $a = a'$ and $b = b'$,
      we have $L_a$ disjoint from $L_{a'}$ if $a \neq a'$.
      Thus,
      $\Inters_{a \in S_1} L_a \inters \Inters_{a \in S_2} U_a \setminus (\Union_{a \in A \setminus S_1} L_a \union \Union_{a \in A \setminus S_2} U_a)$
      is not empty only when $S_1$ is singleton and empty.
      \begin{itemize}
        \item If $S_1$ is empty,
      then
      \[
        \Inters_{a \in S_1} L_a \inters \Inters_{a \in S_2} U_a \setminus (\Union_{a \in A \setminus S_1} L_a \union \Union_{a \in A \setminus S_2} U_a)
        =  \Inters_{a \in S_2} U_a \setminus (\Union_{a \in A} L_a \union \Union_{a \in A \setminus S_2} U_a)
      \]
      has measure 0 because $\Union_{a \in A} L_a$ has measure 1.
        \item Otherwise, if $S_1$ is singleton, say $S_1 = \{a'\}$,
      then
      \begin{align*}
        \Inters_{a \in S_1} L_a \inters \Inters_{a \in S_2} U_a \setminus (\Union_{a \in A \setminus S_1} L_a \union \Union_{a \in A \setminus S_2} U_a)
        &= L_{a'} \inters \Inters_{a \in S_2} U_a \setminus \Union_{a \in A \setminus S_2} U_a).
      \end{align*}
Furthermore,
      \begin{align*}
        \m{\prob'}(\Inters_{a \in S_2} U_a)
        &= \m{\prob'}(\Inters_{a \in S_2} L_a \disjunion (U_a \setminus L_a)) \\
        &= \m{\prob'}(\Inters_{a \in S_2} L_a) + 0
      \end{align*}
      And $\Inters_{a \in S_2} L_a$ is non-empty only if
      $S_2$ is a singleton set or empty set.
      Thus, $L_{a'} \inters \Inters_{a \in S_2} U_a \setminus \Union_{a \in A \setminus S_2} U_a) \subseteq \Inters_{a \in S_2} U_a$ has non-zero measure only if
      $S_2$ is empty or a singleton set.
\begin{itemize}
        \item When $S_2$ is empty,
      \begin{align*}
        L_{a'} \inters \Inters_{a \in S_2} U_a \setminus \Union_{a \in A \setminus S_2} U_a
        &= L_{a'} \setminus \Union_{a \in A} U_a
        \subseteq L_{a'} \setminus  U_{a'}
        =\emptyset
      \end{align*}
        \item When $S_2 = \{a'\}$,
      \begin{align*}
        L_{a'} \inters \Inters_{a \in S_2} U_a \setminus \Union_{a \in A \setminus S_2} U_a
        &= L_{a'} \setminus \Union_{a \in A, a \neq a'} U_a .
      \end{align*}
        \item
      When $S_2 = \{a''\}$ for some $a'' \neq a'$
      \begin{align*}
        L_{a'} \inters \Inters_{a \in S_2} U_a \setminus \Union_{a \in A \setminus S_2} U_a
        &= L_{a'} \inters U_{a''} \setminus \Union_{a \in A, a \neq a''} U_a \\
        &= \emptyset
      \end{align*}
      \end{itemize}
      \end{itemize}

Thus,
      \begin{align*}
         \m{\prob'}(\event_1)
        = & \m{\prob'}\Big(\Union_{S_1 \in I_1, S_2 \in I_2} \Inters_{a \in S_1} L_a \inters \Inters_{a \in S_2} U_a \setminus (\Union_{a \in A \setminus S_1} L_a \union \Union_{a \in A \setminus S_2} U_a) \inters) \\
          = & \m{\prob'}\Big(\Union_{\{a'\} \in I_1, S_2 \in I_2} ( L_{a'} \inters \Inters_{a \in S_2} U_a \setminus \Union_{a \in A \setminus S_2} U_a) \Big) \\
          = & \m{\prob'}\Big(\Union_{\{a'\} \in I_1 \inters I_2} L_{a'} \inters U_{a'} \setminus \Union_{a \in A, a \neq a'} U_a \Big)  \\
          = & \m{\prob'}\Big(\Union_{\{a'\} \in I_1 \inters I_2} ( L_{a'} \setminus \Union_{a \in A, a \neq a'} U_a )  \Big) \\
          = & \m{\prob'}\Big(\Union_{\{a'\} \in I_1 \inters I_2} ( L_{a'} \setminus \Union_{a \in A, a \neq a'} (L_a \Union (U_a \setminus L_a)) )  \Big) \\
          = & \m{\prob'}\Big(\Union_{\{a'\} \in I_1 \inters I_2} ( L_{a'} \setminus \Union_{a \in A, a \neq a'} (L_a ) ) \Big) \\
          = & \m{\prob'}\Big(\Union_{\{a'\} \in I_1 \inters I_2} L_{a'} \Big)
      \end{align*}
      Denote $\Union_{\{a'\} \in I_1 \inters I_2} L_{a'}$
      as $\event'_1$.
      And $\event_1 \setminus \event'_1$ and $\event'_1 \setminus \event_1$ both have measure 0.

      Similar results hold for $\event_2$ as well, and we can show that
      \begin{align*}
        \m{\prob'}(\event_2)
         = & \m{\prob'}\Big(\Union_{\{b'\} \in I_3 \inters I_4} L_{b'} \Big)
      \end{align*}
      Denote $\Union_{\{b'\} \in I_3 \inters I_4} L_{b'}$
      as $\event'_2$.
      And $\event_2 \setminus \event'_2$ and $\event'_2 \setminus \event_2$ both have measure 0.


      Thus,
      \begin{align*}
         \m{\prob'}(\event_1 \inters \event_2)
        =& \m{\prob'}(\event_1 \inters \event_2 \inters \event'_1)
        + \m{\prob'}((\event_1 \inters \event_2) \setminus \event'_1)\\
        =& \m{\prob'}(\event_1 \inters \event_2 \inters \event'_1)
        + 0 \\
        =& \m{\prob'}(\event_1 \inters \event_2 \inters \event'_1 \inters \event'_2) + \m{\prob'}((\event_1 \inters \event_2 \inters \event'_1) \setminus \event'_2) + 0 \\
        =& \m{\prob'}(\event_1 \inters \event_2 \inters \event'_1 \inters \event'_2) + 0 + 0 \\
        =&  \m{\prob'}(\event_1 \inters \event_2 \inters \event'_1 \inters \event'_2) +  \m{\prob'}((\event_2 \inters \event'_1 \inters \event'_2 ) \setminus \event_1) \\
        =&  \m{\prob'}(\event_2 \inters \event'_1 \inters \event'_2)  \\
        =&  \m{\prob'}(\event_2 \inters \event'_1 \inters \event'_2) + \m{\prob'}((\event'_1 \inters \event'_2 ) \setminus \event_2) \\
        =&  \m{\prob'}(\event'_1 \inters \event'_2) \\
        =&  \m{\prob'}\left((\Union_{\{a'\} \in I_1 \inters I_2} L_{a'}) \inters (\Union_{\{b'\} \in I_3 \inters I_4} L_{b'})\right) \\
         =&  \m{\prob'}\left(\Union_{\{a'\} \in I_1 \inters I_2, \{b'\} \in I_3 \inters I_4} L_{a', b'}\right) \\
         =&  \sum_{\substack{\{a'\} \in I_1 \inters I_2 \\ \{b'\} \in I_3 \inters I_4}} \m{\prob'}(L_{a', b'})
       \end{align*}

Next we show that
       $         \m{\prob'}(i) (L_{a, b})
         = \m{\prob'}(i) (\event_1) \cdot \m{\prob'}(i) (\event_2)$.
         Note that
      $\m{\prob'}(L_a)  = \sum_{b} \m{\prob'}(L_{a,b}) = \m{\prob'}(\inv{\aexpr_1}(a))$,
      and
      $\m{\prob'}(L_b)  = \sum_{a} \m{\prob'}(L_{a,b}) = \m{\prob'}(\inv{\aexpr_2}(b))$.
      And $\mu_1 \pprod \mu_2 = \m{\prob'}(i) \circ \inv{(\aexpr_1, \aexpr_2)}$
      implies that
      \begin{align*}
        \m{\prob'}(i) (L_{a, b})
        &= \mu_1 \pprod \mu_2 (a, b)\\
        &=\mu_1(a) \cdot \mu_2(b)
      \end{align*}
      Then
      \begin{align*}
        \mu_1(a)
        &= \mu_1(a) \cdot \sum_{b \in A} \mu_2(b)\\
        &= \sum_{b \in A} \mu_1(a) \cdot \mu_2(b) \\
        &= \sum_{b \in A} \m{\prob'}(i) (L_{a, b}) \\
        &= \m{\prob'}(i) \left(\sum_{b \in A} L_{a, b}\right)  \\
        &= \m{\prob'}(i) (L_a),
      \end{align*}
      and similarly,
      \begin{align*}
        \mu_2(b)
        &= \left(\sum_{a \in A} \mu_1(a)\right) \cdot \mu_2(b)\\
        &= \sum_{a \in A} (\mu_1(a) \cdot \mu_2(b))\\
        &= \sum_{a \in A} \m{\prob'}(i) (L_{a, b}) \\
        &= \m{\prob'}(i) \left(\sum_{a \in A} L_{a, b}\right)  \\
        &= \m{\prob'}(i) (L_b).
      \end{align*}
      Thus,
      \begin{align*}
         \m{\prob'}(i) (L_{a, b})
         &=\mu_1(a) \cdot \mu_2(b)
         = \m{\prob'}(i) (L_a) \cdot \m{\prob'}(i) (L_b)
      \end{align*}



Therefore,
       \begin{align*}
         \m{\prob'}(\event_1 \inters \event_2)
         =&  \sum_{\substack{\{a'\} \in I_1 \inters I_2 \\ \{b'\} \in I_3 \inters I_4}} \m{\prob'}(L_{a', b'}) \\
         =&  \sum_{\substack{\{a'\} \in I_1 \inters I_2 \\ \{b'\} \in I_3 \inters I_4}} \m{\prob'}(L_{a'}) \cdot \m{\prob'}(L_{b'}) \\
       =&  \sum_{\mathclap{\{a'\} \in I_1 \inters I_2}}  \m{\prob'}(L_{a'}) \cdot \sum_{\mathclap{\{b'\} \in I_3 \inters I_4}} \m{\prob'}(L_{b'}) \\
       =& \m{\prob'}(\event_1) \cdot \m{\prob'}(\event_2)\\
       =& \m{\prob_1'}(\event_1) \cdot \m{\prob_2'}(\event_2)
      \end{align*}

  Thus we have
  $(\m{\salg}_1, \m{\prob'}_1) \iprod (\m{\salg}_2, \m{\prob'}_2) \extTo (\m{\salg'}, \m{\prob'})$.
  Let $\m{\permap_1} = \m{\permap_2} = \fun x. \m{\permap'}(x)/2$.

  Next we show that $\distAs{\aexpr_1}{\mu_1} (\m{\salg}_1, \m{\prob'}_1, \m{\permap_1}) $ and $\distAs{\aexpr_2}{\mu_2} (\m{\salg}_2, \m{\prob'}_2, \m{\permap_2})$.
  By definition,
  $\distAs{\aexpr_1}{\mu_1} (\m{\salg}_1, \m{\prob'}_1, \m{\permap_1})$
  is equivalent to
\begin{align*}
      \E \m{\salg''},\m{\prob''}.
      (\Own{\m{\salg''},\m{\prob''}})(\m{\salg}_1, \m{\prob'}_1, \m{\permap_1}) *
    \almostM{\aexpr_1}{(\m{\salg''}(i), \m{\prob''}(i))}
    \land
    \mu_1 = \m{\prob''}(i) \circ \inv{\aexpr_1}
    ,
  \end{align*}
which is equivalent to
\begin{multline*}
      \E \m{\salg''},\m{\prob''}.
      (\m{\salg''},\m{\prob''}) \raLeq (\m{\salg}_1, \m{\prob'}_1) *
      \bigl(\forall a \in A. \exists S_a, T_a \in \ \m{\salg''}(i).\\
      S_a \subseteq \inv{\aexpr_1}(a) \subseteq T_a  \land
      \m{\prob''}(i)(S_a) =  \m{\prob''}(i)(S_a) \land
      \mu_1(a) = \m{\prob''}(i)(S_a) = \m{\prob''}(i)(T_{a})
      \bigr)
  \end{multline*}
We can pick the existential witness to be
  $\m{\salg}_1, \m{\prob'}_1$.
  For any $a \in A$,
  $ \inv{\aexpr_1}(a) = \Union_{b \in A}\inv{(\aexpr_1, \aexpr_2)} (a, b)$.
  Because we have $L_{a, b} \subseteq \inv{(\aexpr_1, \aexpr_2)} (a, b) \subseteq U_{a,b}$,
  then
  \[
   \Union_{b \in A} L_{a, b} \subseteq
   \inv{\aexpr_1}(a) = \Union_{b \in A}\inv{(\aexpr_1, \aexpr_2)} (a, b)
   \subseteq \Union_{b \in A} U_{a, b} .
 \]
  By definition, for each $a$,
  $\Union_{b \in A} L_{a, b} \in \m{\salg}_1(i)$ and
  $\Union_{b \in A} U_{a, b} \in \m{\salg}_1(i)$,
  and we also have
  \begin{align*}
     \m{\prob'}_1(i) (\Union_{b \in A} L_{a, b})
     &= \sum_{b \in A} \m{\prob'}_1(i) (L_{a,b})\\
     &= \sum_{b \in A} \m{\prob'}_1(i) (U_{a,b})\\
     &= \m{\prob'}_1(i) \bigl(\Union_{b \in A} U_{a, b}\bigr)\\
     &= \mu_1(a)
  \end{align*}
  Thus, $S_a = \Union_{b \in A} L_{a, b}$ and
  $T_a = \Union_{b \in A} U_{a, b}$ witnesses the conditions needed
  for
  $\distAs{\aexpr_1}{\mu_1} (\m{\salg}_1, \m{\prob'}_1, \m{\permap_1}) $.
  And similarly, we have $\distAs{\aexpr_2}{\mu_2} (\m{\salg}_2, \m{\prob'}_2, \m{\permap_2}) $.
\end{proof}
 
\subsubsection{Soundness of Conditioning Rules}
\begin{lemma}
\label{proof:c-true}
  \Cref{rule:c-true} is sound.
\end{lemma}

\begin{proof}
  Let $\m{{\raUnit}} = (\m{\salg}_{\m{{\raUnit}}}, \m{\prob}_{\m{{\raUnit}}}, \m{\permap}_{\m{{\raUnit}}}) \in \Model_I$
  be the unit of $\Model_I$ and
  $\m{\krnl} = \fun v. \m{\prob}_{\m{{\raUnit}}}$.
  Then,
  \begin{eqexplain}
    \True
\whichproves*
    \Own{\m{\salg}_{\m{{\raUnit}}}, \m{\prob}_{\m{{\raUnit}}}}
\whichproves
    \Own{\m{\salg}_{\m{{\raUnit}}}, \m{\prob}_{\m{{\raUnit}}}} *
      \pure{
        \forall i\in I\st
          \m{\prob}_{\m{{\raUnit}}}(i) = \bind(\prob, \m{\krnl}(i))
      }
\whichproves
      \Own{\m{\salg}_{\m{{\raUnit}}}, \m{\prob}_{\m{{\raUnit}}}}
      * \pure{
        \forall i\in I\st
          \m{\prob}_{\m{{\raUnit}}}(i) = \bind(\prob, \m{\krnl}(i))
      }
      * \True
\whichproves
    \E \m{\salg}_{\m{{\raUnit}}}, \m{\prob}_{\m{{\raUnit}}}, \m{\krnl}.
      \Own{\m{\salg}_{\m{{\raUnit}}}, \m{\prob}_{\m{{\raUnit}}}}
      \begin{array}[t]{@{}>{{}}l}
      * \pure{
        \forall i\in I\st
          \m{\prob}_{\m{{\raUnit}}}(i) = \bind(\prob, \m{\krnl}(i))
      } \\
      * (
        \forall v \in \psupp(\prob).
         \Own{\m{\salg}_{\m{{\raUnit}}}, \m{\krnl}(I)(v), \m{\permap}_{\m{{\raUnit}}}} \wand \True
      )
      \end{array}
\whichproves
    \CMod{\prob} \wtv. \True
  \qedhere
  \end{eqexplain}
\end{proof} \begin{lemma}
\label{proof:c-false}
  \Cref{rule:c-false} is sound.
\end{lemma}

\begin{proof}
  Assume $a \in \Model_I$ is such that
  $\raValid(a)$ and that it satisfies
  $ \CC{\prob} v.\False $.
  By definition, this means that,
  for some $ \m{\salg}_0, \m{\prob}_0,\m{\permap}_0,$ and $ \m{\krnl}_0 $:
  \begin{gather}
    (\m{\salg}_0, \m{\prob}_0,\m{\permap}_0) \raLeq a
    \label{c-false:prob0}
    \\
    \forall i\in I\st
      \m{\prob}_0(i) = \bind(\prob, \m{\krnl}_0(i))
    \label{c-false:prob0-bind}
    \\
    \forall v \in \psupp(\prob) \st
      \False(\m{\salg}_0, \m{\krnl}_0(I)(v), \m{\permap}_0)
    \label{c-false:False-krnl0}
  \end{gather}
  Let $v_0 \in \psupp(\prob)$---we know one exists because $\prob$
  is a (discrete) probability distribution.
  Then by \eqref{c-false:False-krnl0} on~$v_0$
  we get $\False(\m{\salg}_0, \m{\krnl}_0(I)(v_0), \m{\permap}_0)$ holds.
  Since $\False(\wtv)$ is by definition false,
  we get $\False(a)$ holds \emph{ex falso}.
\end{proof} \begin{lemma}
\label{proof:c-cons}
  \Cref{rule:c-cons} is sound.
\end{lemma}

\begin{proof}
  Assume $a \in \Model_I$ is such that
  $\raValid(a)$ and that it satisfies
  $ \CC{\prob} v.K(v) $.
  By definition, this means that,
  for some $ \m{\salg}_0, \m{\prob}_0,\m{\permap}_0,$ and $ \m{\krnl}_0 $:
  \begin{gather}
    (\m{\salg}_0, \m{\prob}_0,\m{\permap}_0) \raLeq a
    \label{c-cons:prob0}
    \\
    \forall i\in I\st
      \m{\prob}_0(i) = \bind(\prob, \m{\krnl}_0(i))
    \label{c-cons:prob0-bind}
    \\
    \forall v \in \psupp(\prob) \st
      K(v)(\m{\salg}_0, \m{\krnl}_0(I)(v), \m{\permap}_0)
    \label{c-cons:K-krnl0}
  \end{gather}
  Then by the premise $\forall v\st K(v) \proves K'(v)$
  and \eqref{c-cons:K-krnl0} we obtain
  \begin{equation}
    \forall v \in \psupp(\prob) \st
      K'(v)(\m{\salg}_0, \m{\krnl}_0(I)(v), \m{\permap}_0)
    \label{c-cons:K'-krnl0}
  \end{equation}
  By
  \eqref{c-cons:prob0}, \eqref{c-cons:prob0-bind}, and \eqref{c-cons:K'-krnl0}
  we get $ \CC{\prob} v.K'(v) $ as desired.
\end{proof} \begin{lemma}
\label{proof:c-frame}
  \Cref{rule:c-frame} is sound.
\end{lemma}

\begin{proof}
  Assume $a \in \Model_I$ is such that
  $\raValid(a)$ and that it satisfies
  $ P * \CC{\prob} v.K(v) $.
  By definition, this means that
  there exist some
  $(\m{\salg}_1, \m{\prob}_1, \m{\permap}_1)$,
  $(\m{\salg}_2, \m{\prob}_2, \m{\permap}_2)$,
  and $\m{\krnl}$
  such that
  \begin{gather}
    (\m{\salg}_1, \m{\prob}_1, \m{\permap}_1)
    \raOp
    (\m{\salg}_2, \m{\prob}_2, \m{\permap}_2)
    \raLeq a
    \\
    P(\m{\salg}_1, \m{\prob}_1, \m{\permap}_1)
    \\
\forall i \in I.
      \m{\prob}_2(i) = \bind(\prob, \m{\krnl}(i))
    \\
    \forall v\in \psupp(\prob) \st
      K(v)(\m{\salg}_2, \m{\krnl}(I)(v), \m{\permap}_2)
  \end{gather}
Now let:
  \begin{align*}
    (\m{\salg}',\m{\prob}',\m{\permap}')
    &=
    (\m{\salg}_1(i), \m{\prob}_1(i)) \iprod (\m{\salg}_2(i), \m{\prob}_2(i))
  &
    \m{\krnl}'(i) &= \fun v. \m{\prob}_1(i) \iprod \m{\krnl}(i)(v)
  \end{align*}
  By~\cref{lemma:fibre-prod-exists}, for each~$i\in I$:
  \begin{align*}
    (\m{\salg}',\m{\prob}',\m{\permap}')
    &= (\m{\salg}_1(i), \m{\prob}_1(i)) \iprod (\m{\salg}_2(i), \m{\prob}_2(i))
    \\
    &= (\m{\salg}_1(i) \punion \m{\salg}_2(i),
       \bind(\prob, \fun v. \m{\prob}_1(i) \iprod \m{\krnl}(i)(v)))
   \tag*{(By \cref{lemma:fibre-prod-exists})}
    \\
    &= (\m{\salg}_1(i) \punion \m{\salg}_2(i),
       \bind(\prob, \m{\krnl}'(i)))
  \end{align*}
  Notice that $ \m{\krnl}'(I)(v) = \m{\prob}_1 \iprod \m{\krnl}(I)(v) $.
  Thus we obtain:
  \begin{gather}
    (\m{\salg}',\m{\prob}',\m{\permap}')
    \raLeq a
    \\
\forall i \in I.
      \m{\prob}'(i) = \bind(\prob, \m{\krnl}'(i))
    \\
    \intertext{and for all $v \in \psupp(\prob)$,}
(\m{\salg}_1, \m{\prob}_1, \m{\permap}_1)
      \iprod
    (\m{\salg}_2, \m{\krnl}(I)(v), \m{\permap}_2)
    =
    (\m{\salg}', \m{\prob}_1 \iprod \m{\krnl}(I)(v), \m{\permap}')
    \raLeq
    (\m{\salg}', \m{\krnl}'(I)(v),\m{\permap}')
    \\
    P(\m{\salg}_1, \m{\prob}_1, \m{\permap}_1)
    \\
    K(v)(\m{\salg}_2, \m{\krnl}(I)(v), \m{\permap}_2)
  \end{gather}
  which gives us that $a$ satisfies
  $ \CC{\prob} v.(P * K(v)) $ as desired.
\end{proof} \begin{lemma}
\label{proof:c-unit-l}
  \Cref{rule:c-unit-l} is sound.
\end{lemma}

\begin{proof}
  Straightforward.
\end{proof} \begin{lemma}
\label{proof:c-unit-r}
  \Cref{rule:c-unit-r} is sound.
\end{lemma}

\begin{proof}
  We prove the two directions separately.

  \begin{casesplit}
  \case*[Forward direction $\distAs{\aexpr\at{i}}{\prob} \proves \CC\prob v.\sure{\aexpr\at{i} = v}$]
    By unfolding the assumption $\distAs{\aexpr\at{i}}{\prob}$ we get
    that there exist $\m{\salg},\m{\prob}$ such that:
    \[
      \Own{\m{\salg},\m{\prob}}
      *
      \pure{
        \almostM{\aexpr}{(\m{\salg}(i),\m{\prob}(i))}
      }
      *
      \pure{
        \prob = \m{\prob}(i) \circ \inv{\aexpr}
      }
    \]
    holds.
    Let
    \begin{align*}
      \m{\krnl} &\is
      \fun j.
        \begin{cases}
          \fun v. \m{\prob}(j) \CASE j \ne i
          \\
          \fun v. \gamma_v     \CASE j=i
        \end{cases}
      &
      \gamma_v &\is
        \fun \event \of \m{\salg}(i).
          \frac{\m{\prob}(i)(\event \inters \inv{(\aexpr=v)})}
               {\m{\prob}(i)(\inv{(\aexpr=v)})}
    \end{align*}
    That is, $\m{\krnl}(j)$ maps every~$v$ to $\m{\prob}(j)$ when $i\ne j$,
    while when $i=j$ it maps~$v$ to the distribution $\m{\prob}(i)$ conditioned on $\aexpr=v$.
    Note that $\m{\krnl}$ is well defined because
    \begin{enumerate*}
      \item
        although the events
        $\event \inters \inv{(\aexpr=v)}$ and
        $\inv{(\aexpr=v)}$
        might not belong to $\m{\salg}(i)$,
        their probability is uniquely determined
        by almost measurability of $\aexpr$;
      \item
        we are only interested in the cases where~$v \in \psupp(\prob)$,
        which implies that the denominator is not zero:
        $\m{\prob}(i)(\inv{(\aexpr=v)}) = \prob(v) > 0$.
    \end{enumerate*}
    By construction we obtain that
    \begin{gather}
      \forall j \in I\st
        \m{\prob}(j) = \bind(\prob, \m{\krnl}(j))
      \label{c-unit-r:bind}
      \\
      \forall v\in \psupp(\prob)\st
        \m{\krnl}(i)(v)(\inv{(E=v)}) = 1
      \label{c-unit-r:prob1}
    \end{gather}
    From \eqref{c-unit-r:prob1} we get that
    $\sure{\aexpr\at{i} = v}$ holds on
    $(\m{\salg}(i), \m{\krnl}(i)(v), \m{\permap}(i))$,
    from which it follows that:
    \[
      \Own{\m{\salg}, \m{\krnl}(I)(v), \m{\permap}}
      \wand \sure{\aexpr\at{i} = v}
    \]
    Therefore we obtain
    \begin{align*}
      & \E \m{\salg},\m{\prob}, \m{\krnl}, \m{\permap}.
          \Own{\m{\salg},\m{\prob}, \m{\permap}} *
          \pure{\forall j \in I. \m{\prob}(j) = \bind(\prob, \m{\krnl}(j))} \\
      & \qquad \qquad  *
          (
            \forall v \in A_{\prob}.
              \Own{\m{\salg}, \m{\krnl}(I)(v), \m{\permap}}
              \wand \sure{\aexpr\at{i} = v}
          )
    \end{align*}
    which gives us $ \CC\prob v.\sure{\aexpr\at{i} = v} $
    by \cref{prop:cond-as-wand}.

\case*[Backward direction $\CC\prob v.\sure{\aexpr\at{i} = v} \proves \distAs{\aexpr\at{i}}{\prob}$]
    First note that
    \begin{align*}
      \sure{\aexpr\at{i} = v} &(\m{\salg}, \m{\krnl}(v), \m{\permap})
      \\
      &\iff
        \bigl(\distAs{((\aexpr = v) \in \true)\at{i}}{\delta_{\True}}\bigr)
          (\m{\salg}, \m{\krnl}(I)(v), \m{\permap})
      \\
      &\iff
        \almostM{((\aexpr = v) \in \true)}{(\m{\salg}(i), \m{\krnl}(i)(v))}
        \land
        \delta_{\True} =
          \m{\krnl}(i)(v) \circ \inv{((\aexpr = v) \in \true)}
      \\
      &\iff
        \almostM{((\aexpr = v) \in \true)}{(\m{\salg}(i), \m{\krnl}(i)(v))}
        \land
        \delta_{v} = \m{\krnl}(i)(v) \circ \inv{\aexpr}
    \end{align*}

for some $\m{\krnl}$.
    This implies
    $\pure{\almostM{E}{\m{\salg}(i), \m{\krnl}(i)(v)}}$.
    Then, for any value $v \in \psupp(\prob)$,
    \begin{align*}
      \m{\prob}(i) \circ \inv{\aexpr}(v)
      &=(\bind(\prob, \m{\krnl}(i) ) \circ \inv{\aexpr})(v)\\
      &=  \bind(\prob, \m{\krnl}(i) )  (\inv{\aexpr}(v)) \\
      &=  \sum_{\mathclap{v'\in\psupp(\prob)}}  \prob(v') \cdot \m{\krnl}(i)(v') (\inv{\aexpr}(v)) \\
      &=  \sum_{\mathclap{v'\in\psupp(\prob)}}  \prob(v') \cdot (\m{\krnl}(i)(v') \circ \inv{\aexpr}) (v) \\
      &=  \sum_{\mathclap{v'\in\psupp(\prob)}}  \prob(v') \cdot \dirac{v'} (v) \\
&=  \prob(v)
    \end{align*}
    This implies the pure facts that
    $ \almostM{\aexpr}{(\m{\salg}(i),\m{\prob}(i))}$ and
    $\prob = \m{\prob}(i) \circ \inv{\aexpr}$.
    Therefore:
    \begin{eqexplain}
      \CC\prob v. \sure{\aexpr\at{i} = v} \notag
      \whichproves*
      \E \m{\salg},\m{\prob}, \m{\krnl}, \m{\permap}.
          \Own{\m{\salg},\m{\prob}, \m{\permap}} *
          \pure{\forall j \in I. \m{\prob}(j) = \bind(\prob, \m{\krnl}(j))} \notag \\
          & \qquad \qquad  *
          (\forall v \in A_{\prob}.
          \Own{\m{\salg}, \m{\krnl}(I)(v), \m{\permap}}
          \wand
          \sure{\aexpr\at{i} = v}
          )
      \whichproves
      \E \m{\salg},\m{\prob}.
        \Own{\m{\salg},\m{\prob}} *
          \pure{
            \almostM{\aexpr}{(\m{\salg}(i),\m{\prob}(i))}}
            \ast
          \pure{
            \prob = \m{\prob}(i) \circ \inv{\aexpr}
        }
       \whichproves
       \distAs{\aexpr\at{i}}{\prob}
       \qedhere
    \end{eqexplain}
  \end{casesplit}
\end{proof}
 \begin{lemma}
\label{proof:c-assoc}
  \Cref{rule:c-assoc} is sound.
\end{lemma}

\begin{proof}
  Define $\krnl' = \fun v . \bind(\krnl(v), \fun w . \return(v, w))$.
  We start by rewriting the assumption $\CC{\prob} v.\CC{\krnl(v)} w.K(v,w)$ so that $k'$ is used and~$K$ depends only on the binding of the innermost modality:
  \begin{eqexplain}
    \CC{\prob} v.\CC{\krnl(v)} w.K(v,w)
    \whichproves*
    \CC{\prob} v.\CC{\krnl'(v)} (v',w).K(v,w)
    \byrules{c-transf,c-cons}
    \whichproves
    \CC{\prob} v.\CC{\krnl'(v)} (v',w).K(v',w)
    \byrules{c-pure,c-cons}
  \end{eqexplain}
  \Cref{rule:c-transf} is applied to the innermost modality
  by using the bijection $f_v(w) = (v,w)$.
  Then, since $(v',w) \in \psupp(k'(v)) \implies v=v'$,
  we can replace~$v'$ for~$v$ in~$K$.

  Our goal is now to prove:
  \[
    \CC{\prob} v.\CC{\krnl'(v)} (v',w).K(v',w)
    \proves
    \CC{\bind(\prob,\krnl')} (v',w).K(v',w)
  \]

  Let $a\in\Model_I$ be such that $ \raValid(a) $ and that it satisfies
  $ \CC{\prob} v.\CC{\krnl'(v)} (v',w).K(v',w). $
  From this assumption we know that,
  for some $ \m{\salg}_0, \m{\prob}_0,\m{\permap}_0,$ and $ \m{\krnl}_0 $:
  \begin{gather}
    (\m{\salg}_0, \m{\prob}_0,\m{\permap}_0) \raLeq a
    \label{c-assoc:prob0}
    \\
    \forall i\in I\st
      \m{\prob}_0(i) = \bind(\prob, \m{\krnl}_0(i))
    \label{c-assoc:prob0-bind}
  \end{gather}
  such that $\forall v \in \psupp(\prob)$,
  there are some
  $ \m{\salg}_1^v, \m{\prob}_1^v,\m{\permap}_1^v,$ and $ \m{\krnl}_1^v $
  satisfying:
  \begin{gather}
    (\m{\salg}_1^v, \m{\prob}_1^v,\m{\permap}_1^v)
    \raLeq
    (\m{\salg}_0, \m{\krnl}_0(I)(v),\m{\permap}_0)
    \label{c-assoc:prob1}
    \\
    \forall i\in I\st
      \m{\prob}_1^v(i) = \bind(\krnl'(v), \m{\krnl}_1^v(i))
    \label{c-assoc:prob1-bind}
    \\
    \forall (v',w) \in \psupp(\krnl'(v)) \st
      K(v',w)(\m{\salg}_1^v, \m{\krnl}_1^v(I)(v',w), \m{\permap}_1^v)
    \label{c-assoc:K-krnl1}
  \end{gather}

  Our goal is to prove
  $ \CC{\bind(\prob,\krnl')} (v',w).K(v',w) $ holds on $a$.
  To this end, we want to show that
  there exists $\m{\krnl}_2'$ such that:
\begin{gather}
    \forall i\in I\st
      \m{\prob}_0(i) = \bind(\bind(\prob,\krnl'), \m{\krnl}_2'(i))
    \label{c-assoc:goal1}
    \\
    \forall (v',w) \in \psupp(\bind(\prob,\krnl')) \st
      K(v', w)(\m{\salg}_0 , \m{\krnl}_2'(I) (v'), \m{\permap}_0)
    \label{c-assoc:goal2}
  \end{gather}

  Now let
  \[
    \m{\krnl}_2(i) = \fun (v', w). \m{\krnl}_1^{v'}(i)(v', w).
  \]
  which by construction and \cref{c-assoc:prob1-bind} gives us
  \[
    \m{\prob}_1^v(i)
    = \bind(\krnl'(v), \m{\krnl}_1^v(i))
    = \bind(\krnl'(v), \m{\krnl}_2(i))
  \]
  Therefore, by \cref{c-assoc:prob1}, we can apply \cref{lemma:bind-extend}
  and obtain that there exists a $\m{\krnl}_2'$ such that
  \begin{gather}
    \m{\krnl}_0(i)(v) = \bind(\krnl'(v), \m{\krnl}_2'(i))
    \label{c-assoc:k0-bind2}
    \\
    \bigl(\m{\salg}_0, \m{\krnl}_2'(i)(v',w)\bigr)
    \extOf
    \bigl(\m{\salg}_1^{v'}, \m{\krnl}_2(i)(v',w)\bigr)
    =
    \bigl(\m{\salg}_1^{v'},\m{\krnl}_1^{v'}(i)(v',w)\bigr)
    \label{c-assoc:kgeq}
  \end{gather}
  By \cref{c-assoc:prob0-bind,c-assoc:k0-bind2}
  we have:
  \begin{align*}
    \m{\prob}_0(i)
    &= \bind(\prob, \m{\krnl}_0(i)) \\
    &= \bind(\prob, \fun v.\bind(\krnl'(v), \m{\krnl}_2'(i)))
      &\text{By \eqref{prop:bind-assoc}}\\
    &= \bind(\bind(\prob, \krnl'), \m{\krnl}_2'(i))
  \end{align*}
  which proves \cref{c-assoc:goal1}.

  Finally, to prove \cref{c-assoc:goal2}, we can observe that
  $(v',w) \in \psupp(\bind(\prob,\krnl'))$ implies $v'\in \psupp(\prob)$;
  therefore, by \eqref{c-assoc:K-krnl1}, upward closure of $K(v',w)$, and
  \eqref{c-assoc:kgeq} and \eqref{c-assoc:prob1},
  we can conclude~$K(v',w)$ holds on
  $(\m{\salg}_0 , \m{\krnl}_2'(I) (v'), \m{\permap}_0)$,
  as desired.
\end{proof} \begin{lemma}
\label{proof:c-unassoc}
  \Cref{rule:c-unassoc} is sound.
\end{lemma}

\begin{proof}
  Assume $a \in \Model_I$ is such that
  $\raValid(a)$ and that it satisfies
  $ \CC{\bind(\prob,\krnl)} w.K(w) $.
  By definition, this means that,
  for some $ \m{\salg}_0, \m{\prob}_0,\m{\permap}_0,$ and $ \m{\krnl}_0 $:
  \begin{gather}
    (\m{\salg}_0, \m{\prob}_0,\m{\permap}_0) \raLeq a
    \label{c-unassoc:prob0}
    \\
    \forall i\in I\st
      \m{\prob}_0(i) = \bind(\bind(\prob, \krnl), \m{\krnl}_0(i))
    \label{c-unassoc:prob0-bind}
    \\
    \forall w \in \psupp(\bind(\prob, \krnl)) \st
      K(w)(\m{\salg}_0, \m{\krnl}_0(I)(w), \m{\permap}_0)
    \label{c-unassoc:K-krnl0}
  \end{gather}
  Our goal is to show that~$a$ satisfies
  $\CC\prob v. \CC{\krnl(v)} w.K(w)$,
  for which it would suffice to show that there is a $\m{\krnl}_1$
  such that:
  \begin{gather}
    \forall i\in I\st
      \m{\prob}_0(i) = \bind(\prob, \m{\krnl}_1(i))
    \label{c-unassoc:prob0-bind1}
    \\
    \intertext{
      and for all $v \in \psupp(\prob)$
      there is a $\m{\krnl}_2^v$ with
    }
    \forall i\in I\st
      \m{\krnl}_1(i)(v) = \bind(\krnl(v), \m{\krnl}_2^v(i))
    \label{c-unassoc:krnl1-bind}
    \\
    \forall w \in \psupp(\krnl(v)) \st
      K(w)(\m{\salg}_0, \m{\krnl}_2^v(I)(w), \m{\permap}_0)
    \label{c-unassoc:K-krnl1}
  \end{gather}

  To prove this we let
  \begin{align*}
    \m{\krnl}_1(i) &= \fun v.\bind(\krnl(v), \m{\krnl}_0(i))
    &
    \m{\krnl}_2^v(i) &= \m{\krnl}_0(i)
  \end{align*}

  By \eqref{prop:bind-assoc} we have
  \[
    \m{\prob}_0(i)
    = \bind(\bind(\prob, \krnl), \m{\krnl}_0(i))
    = \bind(\prob, \fun v.\bind(\krnl(v), \m{\krnl}_0(i)))
    = \bind(\prob, \m{\krnl}_1(i))
  \]
  which proves \eqref{c-unassoc:prob0-bind1}.
  By construction,
  \[
    \m{\krnl}_1(i)(v)
    = \bind(\krnl(v), \m{\krnl}_0(i))
    = \bind(\krnl(v), \m{\krnl}_2^v(i))
  \]
  proving \eqref{c-unassoc:krnl1-bind}.
  Finally,
  $v \in \psupp(\prob)$ and $w \in \psupp(\krnl(v))$
  imply $w \in \psupp(\bind(\prob, \krnl))$,
  so by \eqref{c-unassoc:K-krnl0} we proved
  \eqref{c-unassoc:K-krnl1}, concluding the proof.
\end{proof}
 \begin{lemma}
\label{proof:c-and}
  \Cref{rule:c-and} is sound.
\end{lemma}

\begin{proof}
  Let~$I_1 = \idx(K_1)$ and $I_2 = I \setminus I_1$;
  by $\idx(K_1) \inters \idx(K_2) = \emptyset$
  we have $I_2 \sups \idx(K_2)$.
  Assume $a\in \Model_I$ is such that~$\raValid(a)$ holds and
  that it satisfies $
    \CC{\prob} v. K_1(v)
      \land
    \CC{\prob} v. K_2(v)
  $.
  This means that
  for each $j \in \set{1,2}$,
  for some $ \m{\salg}_j, \m{\prob}_j,\m{\permap}_j,$ and $ \m{\krnl}_j $:
  \begin{gather}
    (\m{\salg}_j, \m{\prob}_j,\m{\permap}_j) \raLeq a
    \label{c-and:prob}
    \\
    \forall i\in I\st
      \m{\prob}_j(i) = \bind(\prob, \m{\krnl}_j(i))
    \label{c-and:prob-bind}
    \\
    \forall v \in \psupp(\prob) \st
      K_j(v)(\m{\salg}_j, \m{\krnl}_j(I)(v), \m{\permap}_j)
    \label{c-and:K-krnl}
  \end{gather}
  Now let
  \begin{align*}
    \hat{\m{\salg}} &=
    \begin{cases}
      \m{\salg}_1(i) \CASE i \in I_1 \\
      \m{\salg}_2(i) \CASE i \in I_2
    \end{cases}
    &
    \hat{\m{\prob}} &=
    \begin{cases}
      \m{\prob}_1(i) \CASE i \in I_1 \\
      \m{\prob}_2(i) \CASE i \in I_2
    \end{cases}
    &
    \hat{\m{\permap}} &=
    \begin{cases}
      \m{\permap}_1(i) \CASE i \in I_1 \\
      \m{\permap}_2(i) \CASE i \in I_2
    \end{cases}
    &
    \hat{\m{\krnl}}(i) &=
    \begin{cases}
      \m{\krnl}_1(i) \CASE i \in I_1 \\
      \m{\krnl}_2(i) \CASE i \in I_2
    \end{cases}
  \end{align*}
  By construction, we have:
  \begin{gather*}
    (\hat{\m{\salg}}, \hat{\m{\prob}},\hat{\m{\permap}}) \raLeq a
    \\
    \forall i\in I\st
      \hat{\m{\prob}}(i) = \bind(\prob, \hat{\m{\krnl}}(i))
  \end{gather*}
  Moreover, for any~$v \in \psupp(\prob)$ and any $j \in \set{1,2}$,
  since $I_j \sups \idx(K_j)$, condition \eqref{c-and:K-krnl} implies
  \[
    K_j(v)(\hat{\m{\salg}}, \hat{\m{\krnl}}(I)(v), \hat{\m{\permap}})
  \]
  This means
  $(\hat{\m{\salg}}, \hat{\m{\krnl}}(I)(v), \hat{\m{\permap}})$
  satisfies
  $(K_1(v) \land K_2(v))$,
  and thus~$a$ satisfies
  $ \CC\prob v. (K_1(v) \land K_2(v)) $,
  as desired.
\end{proof} \begin{lemma}
\label{proof:c-skolem}
  \Cref{rule:c-skolem} is sound.
\end{lemma}

\begin{proof}
  For any resource $r = (\m{\sigmaF}, \m{\mu}, \m{\permap})$,
  \begin{align*}
    &
    \left(\CC\prob v. \E x \of \Var. Q(v, x) \right) (\m{\sigmaF}, \m{\mu}, \m{\permap}) \\
    {}\iff {} &
    \exists \m{\krnl} \st
             \forall i \in I.
             \m{\mu}(i) = \bind(\prob, \m{\krnl}(i))
    {}\land{}
    \forall v\in  \psupp(\prob) \st
     (\E x \of X. Q(v, x))(\m{\sigmaF}, \m{\krnl}(I)(v), \m{\permap})
  \end{align*}

   For all $v\in \psupp(\prob)$,
   $\E x \of X. Q(v, x)$ holds on $(\m{\sigmaF}, \m{\krnl}(I)(v), \m{\permap})$.
   Thus,
   $Q(v, x_v)(\m{\sigmaF}, \m{\krnl}(I)(v), \m{\permap})$
   holds for some $x_v$.
   Then define $f: A \to \Var$ by letting $f(v) = x_v$ for $v \in \psupp(\mu)$.
   Then,
   \begin{align*}
   \exists \m{\krnl} \st
             \forall i \in I.
             \m{\mu}(i) = \bind(\prob, \m{\krnl}(i))
               {}\land{}
              \forall v\in  \psupp(\prob) \st
      Q(v, f(v))(\m{\sigmaF}, \m{\krnl}(I)(v), \m{\permap})
  \end{align*}
  And therefore $\m{\sigmaF}, \m{\mu}, \m{\permap}$
  satisfies ${\E f \of A \to \Var. \CC\prob v. Q(v, x)}$.
\end{proof} \begin{lemma}
\label{proof:c-transf}
  \Cref{rule:c-transf} is sound.
\end{lemma}

\begin{proof}
  For any resource $a = (\m{\sigmaF}, \m{\mu}, \m{\permap})$,
  if $ \model{\CC\prob v.K(v)}{(\m{\sigmaF}, \m{\mu}, \m{\permap})}$,
  then
  \begin{align*}
    \begin{array}[t]{@{}r@{\,}l@{}}
      \E \m{\krnl}.
      & (\m{\sigmaF}, \m{\mu}, \m{\permap}) \raLeq a
      \land
        \forall i\in I\st
        \m{\mu}(i) = \bind(\prob, \m{\krnl}(i))
      \\ & \land \;
        \forall v \in \psupp(\prob).
          \model{K(v)}{ (\m{\sigmaF}, \m{\krnl}(I)(v), \m{\permap}) }
    \end{array}
  \end{align*}

  $\m{\mu} = \bind(\prob, \m{\krnl})$  says that for any
  $E \in \m{\sigmaF}$,
  \begin{align}
    \m{\mu}(E)
    &= \sum_{\mathclap{v \in \psupp(\mu)}} \prob(v) \cdot \m{\krnl}(I)(v)(E) \notag \\
    &= \sum_{\mathclap{v \mid f(v) \in \psupp(\mu)}} \prob(f(v)) \cdot \m{\krnl}(I)(f(v))(E) \tag{Because $f$ is bijective}\\
    &= \sum_{\mathclap{v \in \psupp(\mu')}} \mu'(v) \cdot \m{\krnl}(I)(f(v))(E) \tag{Because $\mu'(v) = \mu(f(v))$} \\
    &= \bind(\prob', \fun v . \m{\krnl}(I)(f(v)))(E) \notag
  \end{align}
  Thus, $\m{\mu} = \bind(\prob', \fun v . \m{\krnl}(I)(f(v)))$.
  Furthermore, $\model{K(f(v))}{ (\m{\sigmaF}, \m{\krnl}(I)(f(v)), \m{\permap}) }$.

  Thus, if we denote $\fun v . \m{\krnl}(I)(f(v))$ as $\m{\krnl'}$, it satisfies
  \begin{align*}
    \begin{array}[t]{@{}r@{\,}l@{}}
      & (\m{\sigmaF}, \m{\mu}, \m{\permap}) \raLeq a
      \land
        \forall i\in I\st
        \m{\mu}(i) = \bind(\prob', \m{\krnl'}(i))
      \\ & \land \;
        \forall v \in \psupp(\prob).
          \model{K(v)}{ (\m{\sigmaF}, \m{\krnl'}(I)(v), \m{\permap})}
    \end{array}
  \end{align*}
  Thus, $ \model{\CC\prob' v.K(f(v))}{(\m{\sigmaF}, \m{\mu}, \m{\permap})}$.
\end{proof}
 \begin{lemma}
\label{proof:sure-str-convex}
  \Cref{rule:sure-str-convex} is sound.
\end{lemma}

\begin{proof}
  Assume $a \in \Model_I$ is a valid resource that
  satisfies $\CMod{\prob} v. (K(v) \ast  \sure{\aexpr\at{i}})$.
  Then, by definition, we know that,
  for some $ (\m{\salg}_0, \m{\prob}_0, \m{\permap}_0) $ and $\m{\krnl}_0$:
  \begin{gather}
    (\m{\salg}_0, \m{\prob}_0, \m{\permap}_0) \raLeq a
    \label{sure-str-convex:a}
    \\
    \forall i\in I\st
       \m{\prob}_0(i) = \bind(\prob, \m{\krnl}_0(i))
    \label{sure-str-convex:bind-a}
    \\
    \intertext{and, for all $v \in \psupp(\prob)$, there are
      $(\m{\salg}^v_1, \m{\prob}^v_1, \m{\permap}^v_1)$,
      $(\m{\salg}^v_2, \m{\prob}^v_2, \m{\permap}^v_2)$
      such that}
(\m{\salg}^v_1, \m{\prob}^v_1, \m{\permap}^v_1)
    \raOp
    (\m{\salg}^v_2, \m{\prob}^v_2, \m{\permap}^v_2)
    \raLeq
      (\m{\salg}_0, \m{\krnl}_0(I)(v), \m{\permap}_0)
    \label{sure-str-convex:star}
    \\
    K(v)(\m{\salg}^v_1, \m{\prob}^v_1, \m{\permap}^v_1)
    \label{sure-str-convex:K}
    \\
    \sure{\aexpr\at{i}}(\m{\salg}^v_2, \m{\prob}^v_2, \m{\permap}^v_2)
    \label{sure-str-convex:sure}
  \end{gather}
  From \eqref{sure-str-convex:sure} we know that for all~$v \in \psupp(\prob)$
  there are $L^v_1,L^v_0,U^v_1,U^v_0 \in \m{\salg}^v_2(i)$ such that:
  \begin{align*}
    L^v_0 &\subs \inv{\aexpr}(\False) \subs U^v_0
    &
    \m{\prob}^v_2(L^v_0) &= \m{\prob}^v_2(U^v_0) = 0
    \\
    L^v_1 &\subs \inv{\aexpr}(\True) \subs U^v_1
    &
    \m{\prob}^v_2(L^v_1) &= \m{\prob}^v_2(U^v_1) = 1
  \end{align*}
  Without loss of generality, all $L^v_0, L^v_1, U^v_0, U^v_1$ can be assumed
  to be only non-trivial on $\pvar(\aexpr)$.
  Consequently, we can also assume that
  $\m{\permap}^v_2(\ip{x}{j})<1$ for every $\ip{x}{j}$,
and in addition
  $\m{\permap}^v_2(\ip{x}{j})>0$
  if and only if $\p{x}\in \pvar{\aexpr}$ and $j=i$.
  From these components we can construct a new resource:
  \begin{align*}
    \m{\salg}_3(j) &\is
      \begin{cases}
        \sigcl*{\set*{
          \Inters_{v \in \psupp(\prob)} L^v_1,
          \Union_{v \in \psupp(\prob)} U^v_1
        }}
        \CASE j  =  i \\
        \set{\Store, \emptyset}
        \CASE j \ne i
      \end{cases}
    \\
    \m{\prob}_3 &\is \restr{\m{\prob}_0}{\m{\salg}_3}
    \\
    \m{\permap}_3 &\is
      \fun \ip{x}{j}.
      \begin{cases}
        \min
          \set{\m{\permap}^v_2(\ip{x}{i}) | v \in \psupp(\prob)}
        \CASE j=i \land \p{x} \in \pvar(\aexpr) \\
        0 \OTHERWISE
      \end{cases}
  \end{align*}
  By construction we obtain that
  $ \forall j\in I\st \m{\salg}_3(j) \subs \m{\salg}_0(j) $,
and that
  $\raValid(\m{\salg}_3, \m{\prob}_3, \m{\permap}_3)$.
Now letting
  $
   \m{\permap}_1' = {\m{\permap}_0-\m{\permap}_3}
  $,
  we obtain a valid resource
  $(\m{\salg}_0, \m{\prob}_0, \m{\permap}_1')$.


  Moreover,
  we have
  $
    \m{\salg}_0 = \m{\salg}_0 \punion \m{\salg}_3
  $
  and
  $
    \forall j\in I\st
    \forall\event\in\m{\salg}_3(j)\st
      \m{\prob}_3(\event)\in\set{0,1}
  $,
  which means that
  for any $X \in \m{\salg}_3$ and $Y \in \m{\salg}_0$,
  $\m{\prob}_3(X) \cdot \m{\prob}_0(Y) = \m{\prob}_0(X \cap Y)$.
  Then, by \eqref{sure-str-convex:bind-a}:
  \[
    (\m{\salg}_0, \bind(\prob, \m{\krnl}_0), \m{\permap}_1')
    \iprod
    (\m{\salg}_3, \m{\prob}_3, \m{\permap}_3)
    \raLeq
    (\m{\salg}_0, \m{\prob}_0, \m{\permap}_0) = a
  \]
  To close the proof it would then suffice to show that
  $ \CMod{\prob} v. K(v) $
  holds on
  $(\m{\salg}_0, \bind(\prob, \m{\krnl}_0), \m{\permap}_1')$
  and that
  $ \sure{\aexpr\at{i}} $
  holds on
  $ (\m{\salg}_3(j),\m{\prob}_3,\m{\permap}_3) $.
  The latter is obvious.
  The former follows from
  the fact that $ \restr{\m{\krnl}_0(j)(v)}{\m{\salg}^v_1} = \m{\prob}^v_1(j) $;
  by upward-closure and \eqref{sure-str-convex:K}
  this means that, for all $v \in \psupp(\prob)$:
  \[
    K(v)(\m{\salg}^v_1, \m{\prob}^v_1, \m{\permap}^v_1)
    \implies
    K(v)(\m{\salg}_0, \m{\krnl}_0(I)(v), \m{\permap}_1')
  \]
  which proves our claim.
\end{proof}
 \begin{lemma}
\label{proof:c-for-all}
  \Cref{rule:c-for-all} is sound.
\end{lemma}

\begin{proof}
  By unfolding the definitions,
\begin{align*}
      &\CMod\prob \m{v}. \forall x: X. Q(\m{v}) \\
    \iff &
    \begin{array}[t]{r@{\,}l}
    \E \m{\sigmaF}, \m{\mu}_0, \m{\krnl}.
      &\Own{(\m{\sigmaF}, \m{\mu}_0)} *
      \pure{
        \forall i\in I\st
          \m{\mu}_0(i) = \bind(\prob, \m{\krnl}(i))
      }\\ & * \; (
        \forall a \in A_{\prob}.
\Own{(\m{\sigmaF}, \m[i: \m{\krnl}(i)(a) | i \in I])} \wand \forall x: X. Q(\m{v})
          )
    \end{array}
    \\
    \implies &
    \begin{array}[t]{r@{\,}l}
    \forall x: X.
    \E \m{\sigmaF}, \m{\mu}_0, \m{\krnl}.
      &\Own{(\m{\sigmaF}, \m{\mu}_0)} *
      \pure{
        \forall i\in I\st
          \m{\mu}_0(i) = \bind(\prob, \m{\krnl}(i))
      }\\ & * \; (
        \forall a \in A_{\prob}.
\Own{(\m{\sigmaF}, \m[i: \m{\krnl}(i)(a) | i \in I])} \wand Q(\m{v})
          )
    \end{array}\\
    \iff & \forall x: X. \CMod\prob \m{v}. Q(\m{v})
  \end{align*}
\end{proof} \begin{lemma}
\label{proof:c-pure}
  \Cref{rule:c-pure} is sound.
\end{lemma}

\begin{proof}
  We first prove the forward direction:
  For any $a \in \Model_I$,
  if $\model{\pure{\mu(X) = 1} \ast \CMod{\mu}. K(v)}{(a)}$,
  then there exists some
  $\m{\salg}_0, \m{\prob}_0,\m{\permap}_0,$ and $ \m{\krnl}_0 $:
  \begin{gather*}
    (\m{\salg}_0, \m{\prob}_0,\m{\permap}_0) \raLeq a
    \\
    \forall i\in I\st
      \m{\prob}_0(i) = \bind(\prob, \m{\krnl}_0(i)) \\
      \forall v \in \psupp(\prob) \st
      \model{K(v)}{(\m{\salg}_0, \m{\krnl}_0(I)(v), \m{\permap}_0)}
  \end{gather*}

   The pure fact $\pure{\mu(X) = 1}$ implies that
   $X \supseteq \psupp(\mu)$ , and thus
   for every $v \in \psupp(\mu)$, $\pure{v \in X}$.
   Therefore, $\model{K(v)}{(\m{\salg}_0, \m{\krnl}_0(I)(v), \m{\permap}_0)}$,
   which witnesses that
   $\model{\CMod{\mu}. \pure{v \in X} \ast K(v)}{(a)}$.

  We then prove the backward direction:
  if $\CMod{\mu}. \pure{v \in X} \ast K(v) $,
  then there exists
    $\m{\salg}_0, \m{\prob}_0,\m{\permap}_0,$ and~$ \m{\krnl}_0 $:
  \begin{gather*}
    (\m{\salg}_0, \m{\prob}_0,\m{\permap}_0) \raLeq a
    \\
    \forall i\in I\st
      \m{\prob}_0(i) = \bind(\prob, \m{\krnl}_0(i)) \\
      \forall v \in \psupp(\prob) \st
      \model{\pure{v \in X} \ast K(v)}{(\m{\salg}_0, \m{\krnl}_0(I)(v), \m{\permap}_0)}
  \end{gather*}
  Then it must $X \supseteq \psupp(\mu)$,
  which implies that $\pure{\mu(X) = 1}$.
  Meanwhile, $\pure{v \in X} \ast K(v)$ holding on ${(\m{\salg}_0, \m{\krnl}_0(I)(v), \m{\permap}_0)}$
  implies that $K(v)$ holds on
  ${(\m{\salg}_0, \m{\krnl}_0(I)(v), \m{\permap}_0)}$
  Therefore, $\pure{\mu(X) = 1} \ast \CMod{\mu}. K(v)$ holds on $a$.
\end{proof}
 

\subsection{Soundness of Primitive WP Rules}
\label{sec:appendix:wp-rules}


\subsubsection{Structural Rules}
\begin{lemma}
\label{proof:wp-cons}
  \Cref{rule:wp-cons} is sound.
\end{lemma}

\begin{proof}
  For any resource $a$,
  if $(\WP {\m{t}} {Q})(a)$, then
  \[
    \forall \m{\prob}_0 \st
    \forall c \st
    (a \raOp c) \raLeq \m{\prob}_0
    \implies
    \exists b \st
    \bigl(
      (b \raOp c) \raLeq \sem{\m{t}}(\m{\prob}_0)
      \land
      \model{Q}{(b)}
    \bigr)
  \]
  From the premise $Q \proves Q'$, and the fact that $b$ must be valid for $ (b \raOp c) \raLeq \sem{\m{t}}(\m{\prob}_0) $ to hold, we have that $Q(b)$ implies $Q'(b)$.
  Thus, it must
  \[
    \forall \m{\prob}_0 \st
    \forall c \st
    (a \raOp c) \raLeq \m{\prob}_0
    \implies
    \exists b \st
    \bigl(
      (b \raOp c) \raLeq \sem{\m{t}}(\m{\prob}_0)
      \land
      Q'(b)
    \bigr),
  \]
  which says $(\WP {\m{t}} {Q'})(a)$.
\end{proof} \begin{lemma}
\label{proof:wp-frame}
  \Cref{rule:wp-frame} is sound.
\end{lemma}

\begin{proof}
  Let $a\in\Model_I$ be a valid resource such that
  it satisfies $P \ast \WP {\m{t}} {Q}$.
  By definition, this means that, for some $a_1,a_2$:
  \begin{gather}
    a_1 \raOp a_2 \raLeq a
    \\
    P(a_1)
    \label{wp-frame:pa1}
    \\
    \forall \m{\prob}_0, c \st
      (a_2 \raOp c) \raLeq \m{\prob}_0
      \implies
        \exists b \st
        \bigl(
          (b \raOp c) \raLeq \sem{\m{t}}(\m{\prob}_0)
          \land
          Q(b)
        \bigr)
    \label{wp-frame:assumed-wp}
  \end{gather}
  Our goal is to prove $a$ satisfies
  $\WP {\m{t}} {P * Q}$, which,
  by unfolding the definitions, amounts to:
  \begin{equation}
    \exists a' \raLeq a \st
    \forall \m{\prob}_0, c' \st
      (a' \raOp c') \raLeq \m{\prob}_0
      \implies
      \exists b_1,b \st
        ((b_1 \raOp b) \raOp c') \raLeq \sem{\m{t}}(\m{\prob}_0)
        \land
        P(b_1) \land Q(b)
    \label{wp-frame:goal}
  \end{equation}

  Our goal can be proven by instantiating
  $a' = (a_1 \raOp a_2)$ and $b_1 = a_1$,
  from which we reduce the goal
  to proving, for all $\m{\prob}_0, c'$:
  \begin{equation}
      ((a_1 \raOp a_2) \raOp c') \raLeq \m{\prob}_0
      \implies
      \exists b \st
        ((a_1 \raOp b) \raOp c') \raLeq \sem{\m{t}}(\m{\prob}_0)
        \land
        P(a_1) \land Q(b)
  \end{equation}
  We have that $P(a_1)$ holds by \eqref{wp-frame:pa1}.
  By associativity and commutativity of the RA operation,
  we reduce the goal to:
  \begin{equation}
      (a_2 \raOp (a_1 \raOp c')) \raLeq \m{\prob}_0
      \implies
      \exists b \st
        (b \raOp (a_1 \raOp c')) \raLeq \sem{\m{t}}(\m{\prob}_0)
        \land
        Q(b)
  \end{equation}
  This follows by applying assumption~\eqref{wp-frame:assumed-wp}
  with $c = (a_1 \raOp c')$.
\end{proof}
 \begin{lemma}
\label{proof:wp-nest}
  \Cref{rule:wp-nest} is sound.
\end{lemma}

\begin{proof}
  Because $\m{t_1 \m. t_2}$ is defined,
  it must $\supp{\m{t_1}} \cap \supp{\m{t_2}} = \emptyset$.
  By definition
because $\supp{\m{t_1}} \cap \supp{\m{t_2}} = \emptyset$,
  we have
$
    \sem{(\m{t_1 \m. t_2)}}(\m{\prob})
    = \sem{\m{t_2}} (\sem{\m{t_1}} (\m{\prob}))
  $.


  \begin{itemize}
    \item For the $\vdash$ case:
Assume $a$ is a valid resource such that
  $(\WP {\m{t_1}} {\WP {\m{t_2}} {Q}})(a)$ holds.
  Our goal is to prove $(\WP {\m{t_1 \m. t_2}} {Q})(a_0)$ holds,
  which unfolds by definition of WP into:
  \begin{equation}
    \forall \m{\prob}_0\st
    \forall c_0 \st
    (a_0 \raOp c_0) \raLeq \m{\prob}_0
    \implies
      \exists a_2 \st
      \bigl(
        (a_2 \raOp c_0) \raLeq \sem{\m{t_1 \m. t_2}}(\m{\prob}_0)
      \land Q(a_2)
      \bigr)
    \label{wp-nest:goal}
  \end{equation}

  Take an arbitrary $\m{\prob}_0$ and $c_0$ such that
  $ (a_0 \raOp c_0) \raLeq \m{\prob}_0 $.
  By unfolding the WPs in the assumption,
  we have that there exists a
  $a_1 \in \Model_I$ such that:
  \begin{gather}
    (a_1 \raOp c_0) \raLeq \sem{\m{t_1}}(\m{\prob}_0)
    \label{wp-nest:a1}
    \\
    \forall \m{\prob}_1\st
    \forall c_1 \st
      (a_1 \raOp c_1) \raLeq \m{\prob}_1
      \implies
      \exists a_2 \st
      (a_2 \raOp c_1) \raLeq \sem{\m{t_2}}(\m{\prob}_1)
      \land  Q(a_2)
    \label{wp-nest:a2}
  \end{gather}
  We can apply \eqref{wp-nest:a2} to \eqref{wp-nest:a1}
  by instantiating $\m{\prob}_1$ with $\sem{\m{t_1}}(\m{\prob}_0)$,
  and $c_1$ with $c_0$,
  obtaining:
  \[
    \exists a_2 \st
    ((a_2 \raOp c_0) \raLeq \sem{\m{t_1}}(\sem{\m{t_2}}(\m{\prob}_0))
    \land  Q(a_2))
  \]
  When $\m{t_1 \m. t_2}$ is defined,
  the terms $\m{t_1}$ and $\m{t_2}$ are on disjoint indices,
  and thus
  $ \sem{ \m{t_1} \m. \m{t_2}}(\prob_0) = \sem{\m{t_1}}(\sem{\m{t_2}}(\prob_0)) $,
  we obtain the goal \eqref{wp-nest:goal} as desired.

    \item For the $\dashv$ case:
      For any resource $a$,
      \begin{align}
      &
      \WP {(\m{t_1} \m. \m{t_2})} {Q} (a) \notag \\
      {}\iff{} &
        \forall \m{\prob}_0 \st
        \forall c \st
        (a \raOp c) \raLeq \m{\prob}_0
        \implies
        \exists b \st
        \bigl(
          (b \raOp c) \raLeq \sem{\m{t_1} \m. \m{t_2}}(\m{\prob}_0)
          \land
          \model{Q}{(b)}
       \bigr)
       \label{helper:wp-nest:1}
      \end{align}

      Since $(b \raOp c) \raLeq \sem{\m{t_1} \m. \m{t_2}}(\m{\prob}_0)
      $, we have $\raValid(b \raOp c)$ and thus $\raValid(b)$. Say
      \begin{align*}
        b &= \m[i: \m{\sigmaF_b}(i), \m{\mu_b}(i), \m{\permap_b}(i)] \\
        c &= \m[i: \m{\sigmaF_c}(i), \m{\mu_c}(i), \m{\permap_c}(i)]
\end{align*}
      Let
      \begin{align*}
        b' &=
        \begin{cases}
          i:  (\m{\sigmaF_b}(i), \m{\mu_b}(i), \m{\permap_b}(i)) \CASE \text{ if $i \in \supp{\m{t_1}}$} \\
          i : (\m{\sigmaF}(i), \m{\mu}(i), \m{\permap}(i)) \CASE \text{ if $i \notin \supp{\m{t_1}}$ }
        \end{cases}
      \end{align*}
      Since $\raValid(b \raOp c)$ and  $\raValid(a \raOp c)$,
      on each index $i \in I$, we have $\raValid(b'(i) \raOp c(i))$  .
      Thus, $\raValid(b' \raOp c)$.
      Also, for each $i \in I$,
      $(b'(i) \raOp c(i)) \raLeq \sem{\m{t_1}(i)}(\m{\prob}_0(i))$,
      which implies that
      \[
          (b' \raOp c) \raLeq \sem{\m{t_1}}(\m{\prob}_0)
      \]
      We want to show next that
      $\model{\WP {\m{t_2}} {Q}}{b'}$.
      For any $c' = \m[i: \m{\sigmaF'_c}(i), \m{\mu'_c}(i), \m{\permap'_c}(i)]$
      such that $\raValid(b' \raOp c')$,
      it must
      \begin{align*}
        &\raValid((\m{\sigmaF_b}(i), \m{\mu_b}(i), \m{\permap_b}(i)) \raOp (\m{\sigmaF'_c}(i), \m{\mu'_c}(i), \m{\permap'_c}(i)))  &\text{if } i \in \supp{\m{t_1}} \\
        &\raValid((\m{\sigmaF}(i), \m{\mu}(i), \m{\permap}(i)) \raOp (\m{\sigmaF'_c}(i), \m{\mu'_c}(i), \m{\permap'_c}(i)))  &\text{if } i  \in \supp{\m{t_2}}
      \end{align*}
      By~\cref{helper:wp-nest:1},
      $\raValid((\m{\sigmaF}(i), \m{\mu}(i), \m{\permap}(i)) \raOp (\m{\sigmaF'_c}(i), \m{\mu'_c}(i), \m{\permap'_c}(i)))$
      also implies
      \[
        \raValid((\m{\sigmaF_b}(i), \m{\mu_b}(i), \m{\permap_b}(i)) \raOp (\m{\sigmaF'_c}(i), \m{\mu'_c}(i), \m{\permap'_c}(i))).
      \]
      Thus,
      $(b \raOp c) \raLeq \sem{\m{t_1} \m. \m{t_2}}(\m{\prob}_0)
          \land
          Q(b)$
          witnesses that
           $\WP {\m{t_2}} {Q}(b')$.
  \end{itemize}
\end{proof}
 \begin{lemma}
\label{proof:wp-conj}
  \Cref{rule:wp-conj} is sound.
\end{lemma}

\begin{proof}
     For any resource $a$,
      \begin{align*}
        & \bigl(\WP {\m{t_1}} {Q_1} \land   \WP {\m{t_2}} {Q_2}\bigr)(a)
        \\
        \iff &
        \forall \m{\prob}_0 \st
        \forall c \st
        (a \raOp c) \raLeq \m{\prob}_0
        \implies \\
             & \exists b \st
        \bigl(
          (b \raOp c) \raLeq \sem{\m{t_1}}(\m{\prob}_0)
          \land
          Q(b)
       \bigr) \land
         \exists b' \st
        \bigl(
          (b' \raOp c) \raLeq \sem{\m{t_2}}(\m{\prob}_0)
          \land
          Q(b')
       \bigr)
      \end{align*}

      Define $b''$ such that
      \begin{align*}
        b''(i) & =
        \begin{cases}
          b(i) \CASE i \in \idx(Q_1) \\
          b'(i) \OTHERWISE
        \end{cases}
      \end{align*}

      For any $c$, $\raValid(a \raOp c)$ implies
       $\raValid(b'' \raOp c)$ because
       $\raValid(b'(i) \raOp c(i))$ and
       $\raValid(b(i) \raOp c(i))$ for all $i$.
       Furthermore,
       $b''(i) = b(i)$ for $i \in \idx(Q_1)$ implies
       that $Q_1(b'')$.
       Also,
       $\idx(Q_2) \inters \supp{\m{t}_1} \subs \supp{\m{t}_2}$
       implies
       that $Q_2(b'')$.
       Therefore, $(Q_1 \land Q_2)(b'')$, witnessing
       $\model{\WP {\m{t_1} + \m{t_2}} {Q_1 \land Q_2}}{(a)} $.
\end{proof} \begin{lemma}
\label{proof:c-wp-swap}
  \Cref{rule:c-wp-swap} is sound.
\end{lemma}

\begin{proof}
    By the meaning of conditioning modality and weakest precondition transformer,
    \begin{align*}
    &(\ownall \land \CMod\prob v. \WP {\m{t}} {Q(v)})(a) \\
    {}\iff{}
    &\ownall(a) \land
    \begin{array}[t]{@{}r@{\,}l@{}}
    \E \m{\sigmaF}, \m{\mu}, \m{\permap}, \m{\krnl}.
      & (\m{\sigmaF}, \m{\mu}, \m{\permap}) \raLeq a
      \land
        \forall i\in I\st
        \m{\mu}(i) = \bind(\prob, \m{\krnl}(i))
      \\ & \land \;
        \forall v \in \psupp(\prob).
          (\WP {\m{t}} {Q(v)})(\m{\sigmaF}, \m{\krnl}(I)(v), \m{\permap})
    \end{array}
  \end{align*}
Intuitively, for each $v$,
  running $\m{t}$ on each fibre $(\m{\sigmaF}, \m{\krnl}(I)(v), \m{\permap})$
  gives a output resource that satisfies $Q(v)$.

  Assume $\raValid(a)$ holds and let
  $a = (\m{\sigmaF}_a, \m{\mu}_a, \m{\permap}_a)$.
  By~\cref{lemma:bind-extend},
  when $ (\m{\sigmaF}, \m{\mu}, \m{\permap}) \raLeq a$,
  $\m{\mu} = \bind(\prob, \m{\krnl})$ iff that there exists $\m{\krnl}''$ such that
  $\m{\mu}_a = \bind(\prob, \m{\krnl}'')$  and $\m{\krnl}(I)(v) \extTo \m{\krnl}''(I)(v)$ for every $v$.
  Thus,
\begin{align*}
    (\CMod\prob v. \WP {\m{t}} {Q(v)})(\m{\sigmaF}_a, \m{\mu}_a, \m{\permap}_a)
    \iff
    \begin{array}[t]{@{}r@{\,}l@{}}
    \E \m{\krnl}.
      &\forall i\in I\st
        \m{\mu}_a (i) = \bind(\prob, \m{\krnl}''(i))
        \\ & \land \;
        \forall v \in \psupp(\prob).
          (\WP {\m{t}} {Q(v)})(\m{\sigmaF}, \m{\krnl}(I)(v), \m{\permap})
    \end{array}
  \end{align*}

  We want to show that
  \[
      \WP {\m{t}}{\CMod\prob v. {Q(v)}} (a)
    \]
  which is equivalent to
  \[
        \A \m{\prob'}.
        \forall c\st
        a\raOp c \raLeq \m{\prob'}
          \implies
            \exists a'\st
            a'\raOp c \raLeq {\sem{\m{t}} (\m{\prob'})}
            \land
            (\CMod \prob Q(v))(a).
  \]
  Let's fix an arbitrary $ \m{\prob'}, c $ that satisfy
  $\raValid(a \raOp c) \land  a\raOp c \raLeq a_{\m{\prob'}}$,
  we try to construct a corresponding $a'$.
  The high-level approach that we will take is to show that running $\m{t}$ on $a$ takes us to a resource that is equivalent to bind the set of output resource satisfying $Q(v)$ to $\mu$.



  Recall that $a  = (\m{\sigmaF}_a, \m{\mu}_a, \m{\permap}_a)$ also satisfies
  $\ownall$, which says $\m{\sigmaF}_a =  \Full{\Var}$.
  We claim that
  $ a\raOp c  \raLeq (\Full{\Var}, \m{\mu'}, \permap_1)$ holds implies that
  the probability space $c$ is trivial.
  Say $c = (\m{\sigmaF}_c, \m{\mu}_c, \m{\permap}_c)$, then for any $E \in \m{\sigmaF}_c$,
  the event $E$ must also in $\m{\sigmaF}_a$ and $\Full{\Var}$  because they are
  the full sigma algebra.
  By definition of $ a\raOp c  \raLeq (\Full{\Var}, \m{\mu'}, \permap_1)$,
  we have
\begin{align}
    &\m{\mu}_c(E) \cdot \m{\mu}_a(E) =  \m{\mu'}(E \cap E)  = \m{\mu'}(E).
    \label{eq:c-wp-swap-c-trivial}
  \end{align}
  Another implication of  $ a\raOp c  \raLeq (\Full{\Var}, \m{\mu'}, \permap_1)$ is that
  we have $\m{\mu}_c(E)  = \m{\mu'}(E)$ and  $\m{\mu}_a(E)  = \m{\mu'}(E)$.
  Combining with~\cref{eq:c-wp-swap-c-trivial}, we can conclude
  \begin{align*}
    \m{\mu'}(E) \cdot \m{\mu'}(E) =  \m{\mu'}(E),
  \end{align*}
  which implies that
  $\m{\mu}_c(E) = \m{\mu'}(E)  \in \{0,1\}$.
  Therefore, $c$ is a trivial probability space and
  \begin{align*}
    (\m{\salg}_a, \m{\krnl}(I)(v), \m{\permap}_a) \raOp c \raLeq
    (\m{\salg}_a, \m{\krnl}(I)(v), \m{\permap}_a)
  \end{align*}

  Furthermore, for every $v \in \psupp(\mu)$, we have
  $(\WP {\m{t}} {Q(v)})(\m{\sigmaF}, \m{\krnl}(I)(v), \m{\permap})$
  which implies
  \begin{align}
    \label{eq:c-wp-swap:helper}
        \A \m{\krnl'}. &
(\m{\salg}_a, \m{\krnl}(I)(v), \m{\permap}_a) \raOp c
          \raLeq \m{\krnl'}(I)(v)
        \\
        & {} \implies {}
        \exists a_v\st
(a_v \raOp c \raLeq \sem{\m{t}} (\m{\krnl'}(I)(v))) \land Q(v)(a_v) .
      \end{align}
Therefore,
  \begin{align}
   & a\raOp c \raLeq a_{\m{\prob'}}
  {}\implies{} \forall v \in \psupp(\mu). (\raValid((\m{\salg}_a, \m{\krnl}(I)(v), \m{\permap}_a) \raOp c) \land  (\m{\salg}_a, \m{\krnl}(I)(v), \m{\permap}_a) \raOp c \raLeq (\Full{\Var}, \m{\krnl}(I)(v), \fullp) \tag{By~\ref{lemma:fibre-prod-exists} and~\ref{lemma:bind-extend}}\\
  {}\implies{} & \forall v \in \psupp(\mu).  \exists a_v\st
        \raValid(a_v \raOp c) \land  \left(a_v \raOp c \raLeq (\Full{\Var}, \sem{\m{t}} (\m{\krnl}(I)(v)), \fullp)\right)  \land  Q(v)(a_v)
        \tag{By~\cref{eq:c-wp-swap:helper}} \\
  {}\implies{} & \forall v \in \psupp(\mu). \m{\permap}_{a_v} + \m{\permap}_c \raLeq \fullp \land    Q(v)(\Full{\Var}, \sem{\m{t}} (\m{\krnl'}(I)(v)), \fullp).
        \tag{By upwards closure}
  \end{align}

  Let $a'_v =  (\Full{\Var}, \sem{\m{t}} (\m{\krnl'}(I)(v)), \m{\permap}_{a})$.
  Because $\mu_c(E) \in \{0, 1\}$ for any $E \in \sigmaF_c$, for every $v$,
  we have $ (\Full{\Var}, \sem{\m{t}} (\m{\krnl'}(I)(v)))  \raOp (\m{\sigmaF}_c, \m{\mu}_c)$
  defined and thus
  $a'_v \raOp c$ valid.
  Define
  \[
    a' =
     (\Full{\Var},
      \bind(\mu, \fun v . \sem{\m{t}} (\m{\krnl'}(I)(v)),
      \m{\permap}_a)
  \]
  By~\cref{lemma:fibre-prod-exists},
  $ \raValid(a'_v \raOp c)$ for all $v \in \psupp_{\mu}$ implies
  $\raValid(a' \raOp c)$.
  Also, because  $Q(v)(a_v)$ for all $v \in A_{\mu}$,
  $(\CMod{\prob} v. Q(v))(a') $.
Thus, $(\WP {\m{t}}{\CMod\prob v. {Q(v)}}) (a)$.
\end{proof}
 
\subsubsection{Program Rules}
\begin{lemma}
\label{proof:wp-skip}
  \Cref{rule:wp-skip} is sound.
\end{lemma}

\begin{proof}
  Assume $a \in \Model_I$ is valid and such that~$P(a)$ holds.
  By unfolding the definition of WP, we need to prove
  \[
    \forall \m{\prob}_0.
      \forall c \st
      (a \raOp c) \raLeq \m{\prob}_0
      \implies
      \exists b \st
      \bigl(
        (b \raOp c) \raLeq \sem{\m{t}}(\m{\prob}_0)
        \land
        P(b)
      \bigr)
  \]
  which follows trivially
  by $\sem{\m[i:\code{skip}]}(\m{\prob}_0)=\m{\prob}_0$ and picking $b=a$.
\end{proof}
 \begin{lemma}
\label{proof:wp-seq}
  \Cref{rule:wp-seq} is sound.
\end{lemma}

\begin{proof}
Assume $a_0 \in \Model_I$ is a valid resource such that
  $(\WP {\m[i: t]} {\WP {\m[i: t']} {Q}})(a_0)$ holds.
  Our goal is to prove $(\WP {(\m[i: t\p; t'])} {Q})(a_0)$ holds,
  which unfolds by definition of WP into:
  \begin{equation}
    \forall \m{\prob}_0\st
    \forall c_0 \st
    (a_0 \raOp c_0) \raLeq \m{\prob}_0
    \implies
      \exists a_2 \st
      \bigl(
      (a_2 \raOp c_0) \raLeq \sem{\m[i: t\p; t']}(\m{\prob}_0)
      \land Q(a_2)
      \bigr)
    \label{wp-seq:goal}
  \end{equation}

  Take an arbitrary $\m{\prob}_0$ and $c_0$ such that
  $ (a_0 \raOp c_0) \raLeq \m{\prob}_0 $.
  By unfolding the WPs in the assumption,
  we have that there exists a
  $a_1 \in \Model_I$ such that:
  \begin{gather}
    (a_1 \raOp c_0) \raLeq \sem{\m[i: t]}(\m{\prob}_0)
    \label{wp-seq:a1}
    \\
    \forall \m{\prob}_1\st
    \forall c_1 \st
      (a_1 \raOp c_1) \raLeq \m{\prob}_1
      \implies
      \exists a_2 \st
      ((a_2 \raOp c_1) \raLeq \sem{\m[i: t']}(\m{\prob}_1)
      \land  Q(a_2))
    \label{wp-seq:a2}
  \end{gather}
  We can apply \eqref{wp-seq:a2} to \eqref{wp-seq:a1}
  by instantiating $\m{\prob}_1$ with $\sem{\m[i: t]}(\m{\prob}_0)$,
  and $c_1$ with $c_0$,
  obtaining:
  \[
    \exists a_2 \st
    ((a_2 \raOp c_0) \raLeq \sem{\m[i: t']}(\sem{\m[i: t]}(\m{\prob}_0))
    \land  Q(a_2))
  \]
  Since by definition,
  $ \sem{t\p; t'}(\prob_0) = \sem{t'}(\sem{t}(\prob_0)) $,
  we obtain the goal \eqref{wp-seq:goal} as desired.
\end{proof} \begin{lemma}
\label{proof:wp-assign}
  \Cref{rule:wp-assign} is sound.
\end{lemma}

\begin{proof}
  \newcommand{\specialevent}{A}
  Let $a \in \Model_I$ be a valid resource,
  and let $a(i) = (\salg, \prob, \permap)$.
  By assumption we have
  $\permap(\p{x}) = 1$ and
  $\permap(\p{y}) > 0$ for all $ \p{y} \in \FV(\expr)$.
  We want to show that $a$ satisfies
  $\WP {\m[i: \code{x:=}\expr]} {\sure{\ip{x}{i} = \expr\at{i}}}$.
This is equivalent to
\[
    \forall \m{\prob}_0 \st
      \forall c\st
      (a\raOp c \raLeq  \m{\prob}_0)
      \implies
        \exists b\st (
          b\raOp c \raLeq  \sem{\m[i: \code{x:=}\expr]}(\m{\prob}_0) \land
          \sure{\ip{x}{i} = \expr\at{i}}(b)
        )
  \]
  We show this holds by picking~$b$ as follows:
  \begin{align*}
    b &\is a\m[i: {(\salg_b, \prob_b, \permap)}]
    &
    \salg_b &\is \set{
      \Store,\emptyset,\specialevent,\Store\setminus\specialevent
    }
    &
    \specialevent &\is \set{
      \store\upd{\p{x}->\sem{\expr}(\store)} | \store \in \Store
}
  \end{align*}
  where $\prob_b$ is determined by setting $ \prob_b(\specialevent)=1. $

  By construction we have that $\sure{\ip{x}{i} = \expr\at{i}}(b)$ holds.
  To close the proof we then need to show that
$(b \raOp c) \raLeq \sem{\m[i: \code{x:=}\expr]}(\m{\prob}_0)$.

  Let $c(i) = (\salg_c,\prob_c,\permap_c)$.
  Observe that by the assumptions on $\permap$,
  we have $\raValid(b)$ since $\salg_b$ is
  only non-trivial on $\pvar(\expr)\union\set{\p{x}}$;
  moreover, by the assumption $\raValid(a \raOp c)$
  we have that $\raValid(\permap + \permap_c)$ holds,
  which means that $\permap_c(\p{x})=0$,
  and thus $\salg_c$ is trivial on \p{x}.


  \newcommand{\Pre}{\operatorname{pre}}
Let us define the function
  $
    \Pre \from \powerset(\Store) \to \powerset(\Store)
  $
  as:
  \[
    \Pre(\event) \is
      \set{ \store
          | \store\upd{\p{x}->\sem{\expr}(\store)} \in \event
        }.
  \]
  That is, $\Pre(\event)$ is the weakest precondition (in the standard sense)
  of the assignment.
  By construction, we have:
  \begin{align*}
    \Pre(\specialevent) &= \Store
    &
    \Pre(\event_1 \inters \event_2) &=
      \Pre(\event_1) \inters \Pre(\event_2)
    \\
    \Pre(\Store \setminus \specialevent) &= \emptyset
    &
    \Pre(\event_c) &= \event_c
      \text{ for all } \event_c \in \salg_c
  \end{align*}
  In particular, the latter holds because $\salg_c$ is trivial in \p{x}.

  By unfolding the definition of $\sem{\hole}$,
  it is easy to check that for every $\event \in \Full{\Store}$:
  \[
    \sem{\code{x:=}\expr}(\mu_0)(\event) = \mu_0(\Pre(\event))
  \]

  We are now ready to show
  $(b \raOp c) \raLeq \sem{\m[i: \code{x:=}\expr]}(\m{\prob}_0)$
  by showing that
  $
    (\salg_b,\prob_b)\iprod(\salg_c,\prob_c)
    =
    (\salg_b \punion \salg_c,
     \restr{\sem{\code{x:=}\expr}(\prob_0)}{(\salg_b \punion \salg_c)})
  $
  where $\prob_0 = \m{\prob}_0(i)$.
  To show this it suffices to prove that
  for every $\event_b \in \salg_b$ and every $\event_c \in \salg_c$,
  $
    \sem{\code{x:=}\expr}(\prob_0)(\event_b \inters \event_c)
    = \prob_b(\event_b) \cdot \prob_c(\event_c).
  $
  We proceed by case analysis on $\event_b$:
  \begin{casesplit}
  \case[$\event_b = \specialevent$]
    Then:
    \begin{align*}
      \sem{\code{x:=}\expr}(\prob_0)(\specialevent \inters \event_c)
        &= \prob_0(\Pre(\specialevent \inters \event_c))
      \\&= \prob_0(\Pre(\specialevent) \inters \Pre(\event_c))
      \\&= \prob_0(\Store \inters \Pre(\event_c))
      \\&= \prob_0(\Pre(\event_c))
      \\&= \prob_b(\specialevent) \cdot \prob_0(\event_c)
      \\&= \prob_b(\specialevent) \cdot \prob_c(\event_c)
    \end{align*}
  \case[$\event_b = \Store \setminus \specialevent$]
    Then:
    \begin{align*}
      \sem{\code{x:=}\expr}(\prob_0)
          (\Store \setminus \specialevent \inters \event_c)
        &= \prob_0(\Pre((\Store \setminus \specialevent) \inters \event_c))
      \\&= \prob_0(\Pre(\Store \setminus \specialevent) \inters \Pre(\event_c))
      \\&= \prob_0(\emptyset \inters \Pre(\event_c))
      \\&= 0
      \\&= \prob_b(\Store \setminus \specialevent) \cdot \prob_c(\event_c)
    \end{align*}
  \case[$\event_b = \Store$ or $\event_b = \emptyset$]
    Analogous to the previous cases.
    \qedhere
  \end{casesplit}
\end{proof}
 \begin{lemma}
\label{proof:wp-samp}
  \Cref{rule:wp-samp} is sound.
\end{lemma}

\begin{proof}
  \newcommand{\specialevent}{A}
  Assume $a \in \Model_I$ is valid and such that
  $ a(i) = (\salg, \prob, \permap)$, with $\permap(x) = 1$.
  Our goal is to show that~$a$ satisfies
  $
    \WP {\m[i: \code{x:~$\dist$($\vec{v}$)}]} {\distAs{\ip{x}{i}}{d(\vec{v})}}
  $
which is equivalent to proving, for all $\m{\prob}_0$ and for all~$c$:
\begin{equation}
    (a\raOp c \raLeq  \m{\prob}_0)
    \implies
      \exists b\st \bigl(
        b\raOp c \raLeq \sem{\m[i: \code{x:~$\dist$($\vec{v}$)}]}(\m{\prob}_0)
        \land
        (\distAs{x}{d(\vec{v})})(b)
      \bigr)
    \label{wp-samp:goal}
  \end{equation}

  Let $\prob_0 = \m{\prob}_0(i)$ and
      $\prob_1 = \sem{\code{x:~$\dist$($\vec{v}$)}}(\prob_0)$.
  Moreover, let $ c(i) = (\salg_c,\prob_c,\permap_c) $.
  Observe that by the assumptions on $\permap$ and validity of~$a \raOp c$,
  we have $\permap_c(\p{x})=0$,
  which means $\salg_c$ is trivial on \p{x}.
  We aim to prove~\eqref{wp-samp:goal} by letting
  \begin{align*}
    b &\is a\m[i: {(\salg_b, \prob_b, \permap_b)}]
    &
    \prob_b &\is
      \restr{\prob_1}{\salg_b}
    \\
    \salg_b &\is
      \sigcl*{
        \set[\big]{
          \set{ \store \in \Store | \store(\p{x}) = v }
        | v\in \Val
        }
      }
    &
    \permap_b &\is
      \perm{\p{x}: 1}
\end{align*}
  Note that by construction $ \raValid(\permap_b + \permap_c) $,
  and $\raValid(b)$ since $\salg_b$ is only non-trivial in \p{x}.
  \newcommand{\Pre}{\operatorname{pre}}
  Similarly to the proof for~\cref{rule:wp-assign},
  we define the function
  $
    \Pre \from \powerset(\Store) \to \powerset(\Store)
  $
  as:
  \[
    \Pre(\event) \is
      \set{ \store
          | \exists v \in \Val \st \store\upd{\p{x}->v} \in \event
        }.
  \]
  Since $\salg_c$ is trivial on \p{x},
  for all $ \event_c \in \salg_c $,
  $\Pre(\event_c) = \event_c$.
  Moreover,
  for all $ \event_b \in \salg_b \setminus \set{\emptyset} $,
  $ \Pre(\event_b) = \Store $,
  since $\event_b$ is trivial on every variable except~\p{x}.

  By unfolding the definitions, we have:
  \begin{align*}
    \mu_1(\event) &=
      \sem{\code{x:~$\dist$($\vec{v}$)}}(\prob_0)(\event)
\\&=
      \Sum_{\store \in \event}
        \prob_0(\Pre(\store)) \cdot \sem{\dist}(\vec{v})(\store(\p{x}))
  \end{align*}
  We now show that
  $
    (\salg_b, \prob_b) \iprod (\salg_c, \prob_c)
    =
    (\salg_b \punion \salg_c, \restr{\prob_1}{(\salg_b \punion \salg_c)})
  $
  by showing that
  for all $\event_b \in \salg_b$ and
          $\event_c \in \salg_c$:
  $
    \prob_1(\event_b \inters \event_c)
    =
    \prob_b(\event_b) \cdot \prob_c(\event_c).
  $
  \newcommand{\valOf}{\operatorname{V}}
  To prove this we first
  define $\valOf \from \powerset(\Store) \to \powerset(\Val)$ as
  $
    \valOf(\event) \is
      \set{ \store(\p{x}) | \store \in \event },
  $
  and $ S_w \is \set{ \store | \store(\p{x}) = w } $.
  We observe that
  $
    \event_b   = \Dunion_{w \in \valOf(\event_b)} S_w
  $, and thus
  $
    \event_b \inters \event_c
    = \Dunion_{w \in \valOf(\event_b)} (\event_c \inters S_w)
  $;
  moreover,
  $
    \Pre(\event_c \inters S_w)
    = \set{ \store | \store\upd{x->w} \in \event_c } = \event_c
  $.
  Thus, we can calculate:
  \begin{align*}
    \prob_1(\event_b \inters \event_c)
      &=
    \sum_{\mathclap{\store \in \event_b \inters \event_c}}
      \prob_0(\Pre(\store)) \cdot \sem{\dist}(\vec{v})(\store(\p{x}))
    \\&=
    \sum_{w \in \valOf(\event_b)}
    \sum_{\store \in \event_c \inters S_{w}}
      \prob_0(\Pre(\store)) \cdot
      \sem{\dist}(\vec{v})(w)
    \\&=
    \sum_{w \in \valOf(\event_b)}
    \left(
      \sem{\dist}(\vec{v})(w) \cdot
      \sum_{\mathclap{\store \in \event_c \inters S_{w}}}
      \prob_0(\Pre(\store))
    \right)
    \\&=
    \left(
    \sum_{w \in \valOf(\event_b)}
      \sem{\dist}(\vec{v})(w)
      \cdot \prob_0(\Pre(\event_c \inters S_w))
    \right)
    \\&=
    \left(
    \sum_{w \in \valOf(\event_b)}
      \sem{\dist}(\vec{v})(w)
    \right)
    \cdot \prob_0(\event_c)
    \\&=
    \prob_b(\event_b) \cdot  \prob_c(\event_c)
  \end{align*}
The last equation is given by
  $ a\raOp c \raLeq  \m{\prob}_0 $ which implies that $ \prob_c = \restr{\prob_0}{\salg_c} $, and by:
  \begin{align*}
    \prob_b(\event_b)
      =
    \prob_1(\event_b)
      &=
    \sum_{\store \in \event_b}
      \prob_0(\Pre(\store)) \cdot \sem{\dist}(\vec{v})(\store(\p{x}))
    \\&=
    \sum_{w \in \valOf(\event_b)}
      \sum_{\store \in S_w}
        \prob_0(\Pre(\store)) \cdot \sem{\dist}(\vec{v})(w)
    \\&=
    \sum_{\mathclap{w \in \valOf(\event_b)}}
      \sem{\dist}(\vec{v})(w)
  \end{align*}

  Finally, we need to show $ (\distAs{x}{d(\vec{v})})(b) $
  which amounts to proving
  $\almostM{\p{x}}{(\salg_b,\prob_b)}$
  and
  $\sem{\dist}(\vec{v}) = \prob_b \circ \inv{\p{x}}$.
  The former holds because by construction \p{x} is measurable in $\salg_b$.
  For the latter, for all $W \subs \Val$:
  \[
    (\prob_b \circ \inv{\p{x}})(W)
    =
    \prob_b (\inv{\p{x}}(W))
    =
    \sum_{\mathclap{w \in \valOf(\inv{\p{x}}(W))}}
      \sem{\dist}(\vec{v})(w)
    =
    \sum_{w \in W}
      \sem{\dist}(\vec{v})(w)
    =
    \sem{\dist}(\vec{v})(W).
    \qedhere
  \]
\end{proof}
 \begin{lemma}
\label{proof:wp-if-prim}
  \Cref{rule:wp-if-prim} is sound.
\end{lemma}

\begin{proof}
  For any valid resource $a$,
  \begin{align*}
    & \left(\ITE{v}{\WP{\m[i: t_1]}{Q(1)}}{\WP{\m[i: t_2]}{Q(0)}}\right)(a)\\
    {}\iff {} &
    \begin{cases}
      \left(\WP{\m[i: t_1]}{Q(1)}\right)(a) \CASE v \doteq 1 \\
      \left(\WP{\m[i: t_2]}{Q(0)}\right)(a)\OTHERWISE \\
    \end{cases} \\
    {}\iff {}&
  \forall \m{\prob}_0.
    \forall c \st
    (a \raOp c) \raLeq \m{\prob}_0
    \implies
      \begin{cases}
          \exists b \st (b \raOp c) \raLeq \sem{i: \m{t_1}}(\m{\prob}_0)
      \land Q(1)(b)
      \CASE v \doteq 1 \\
          \exists b \st (b \raOp c) \raLeq \sem{i: \m{t_2}}(\m{\prob}_0)
      \land Q(0)(b)
      \OTHERWISE
      \end{cases} \\
    {}\iff {} &
    \forall \m{\prob}_0.
    \forall c \st
    (a \raOp c) \raLeq \m{\prob}_0
    {}\implies {}
    \exists b \st (b \raOp c) \raLeq \sem{i: \ITE{v}{\m{t_1}}{\m{t_2}}}(\m{\prob}_0)
      \land  Q(v \doteq 1)(b) \\
    \implies &  \left(\WP{\m[i: \ITE{v}{\m{t_1}}{\m{t_2}}]}{Q(v \doteq 1)}\right)(a)
  \end{align*}
\end{proof} \begin{lemma}
\label{proof:wp-bind}
  \Cref{rule:wp-bind} is sound.
\end{lemma}

\begin{proof}
  For any resource $a = (\m{\sigmaF}, \m{\mu}, \m{\permap})$,
  $(\sure{\expr\at{i}=v} * \WP{\m*[i: {\Ectxt[v]}]}{Q})(\m{\sigmaF}, \m{\mu}, \m{\permap})$ iff
    there exists $(\m{\sigmaF_1}, \m{\mu_1}, \m{\permap_1}), (\m{\sigmaF_2}, \m{\mu_2}, \m{\permap_2})$ such that
\begin{gather*}
      (\sure{\expr\at{i}=v})(\m{\sigmaF_1}, \m{\mu_1}, \m{\permap_1})\\
      (\WP{\m*[i: {\Ectxt[v]}]}{Q})(\m{\sigmaF_2}, \m{\mu_2}, \m{\permap_2})\\
      (\m{\sigmaF_1}, \m{\mu_1}, \m{\permap_1}) \raOp  (\m{\sigmaF_2}, \m{\mu_2}, \m{\permap_2})
      \raLeq (\m{\sigmaF}, \m{\mu}, \m{\permap})
    \end{gather*}
By the upwards closure, we also have
    \begin{gather*}
      (\sure{\expr\at{i}=v})(\m{\sigmaF}, \m{\mu}, \m{\permap}) \\
      (\WP{\m*[i: {\Ectxt[v]}]}{Q})(\m{\sigmaF}, \m{\mu}, \m{\permap})
    \end{gather*}
The fact that $(\sure{\expr\at{i}=v})(\m{\sigmaF_1}, \m{\mu_1}, \m{\permap_1})$
    implies that $\m{\mu_1}(\inv{(\expr\at{i}=v)} (\True)) = 1$,
    which implies that
    $\sem{\expr}(s) = v$ for all $s \in \psupp(\m{\mu}_1(i))$.


    By~\cref{lemma:context-binding}, we have for any $s \in \Store$,
    \begin{align*}
      \Sem[K]{\Ectxt[\expr]}(s) = \Sem[K]{\Ectxt[\sem{\expr}(s)]}(s),
    \end{align*}
    which implies that for any $\mu_0$ over $\Full{\Store}$
    \begin{align*}
      \sem{\Ectxt[\expr]}(\mu_0)
      &= \DO{
        s <- \m{\mu_0};
        \Sem[K]{\Ectxt[\expr]}(s)
      }\\
      &= \DO{
        s <- \m{\mu_0};
        \Sem[K]{\Ectxt[\sem{\expr}(s)]}(s)
      }\\
      &= \DO{
        s <- \m{\mu_0};
        \Sem[K]{\Ectxt[v]}(s)
      }\\
      &= \sem{\Ectxt[v]}(\mu_0).
    \end{align*}
Define $\mu'_0 = \Sem{\m[i: {\Ectxt[v]}]} \mu_0$.
    Thus, $(\WP{\m*[i: {\Ectxt[v]}]}{Q})(a)$ iff
\begin{align*}
  \forall \mu_0 \st
    \forall c\st
    (\raValid(a \raOp c) \land  a\raOp c \raLeq  a_{\mu_0})
      \implies
        \exists a'\st
        (\raValid(a' \raOp c) \land  a'\raOp c \raLeq  a_{\mu'_0} \land Q(a'))
\end{align*}
iff
\begin{align*}
  \forall \mu_0 \st
    \forall c\st
    (\raValid(a \raOp c) \land  a\raOp c \raLeq  a_{\mu_0})
      \implies
        \exists a'\st
        (\raValid(a' \raOp c) \land  a'\raOp c \raLeq  a_{\mu'_0} \land Q(a'))
\end{align*}
iff $\model{\WP{\m*[i: {\Ectxt[\expr]}]}{Q}}{(a)}$.
\end{proof}
 \begin{lemma}
\label{proof:wp-loop-unf}
  \Cref{rule:wp-loop-unf} is sound.
\end{lemma}

\begin{proof}
  By definition,
  \begin{align*}
    \sem{\Loop{(n+1)}{t}}(\prob)
    &= \bigl(
         \DO{s <- \prob; s' <- \var{loop}_{\term}(n,s); \Sem[K]{t}(s')}
       \bigr) \\
    &= \sem{(\Loop{n}{t})\p;t}(\prob)
  \end{align*}
  thus the rule follows from the argument of \cref{proof:wp-seq}.
\end{proof} \begin{lemma}
\label{proof:wp-loop}
  \Cref{rule:wp-loop} is sound.
\end{lemma}

\begin{proof}
  By induction on~$n$.
  \begin{induction}
    \step[Base case~$n=0$]
      Analogously to \cref{proof:wp-skip}
      since, by definition,
      $\Sem{\Loop{0}{t}}(\mu_0) = \mu_0$.

    \step[Induction step~$n>0$]
      By induction hypothesis
      $P(0) \proves \WP{\m[j: \Loop{(n-1)}{t}]}{P(n-1)}$ holds,
      and we want to show that
      $P(0) \proves \WP{\m[j: \Loop{n}{t}]}{P(n)}$.
      By \cref{proof:wp-loop-unf},
      it suffices to show
      $ P(0) \proves \WP{\m[j: \Loop{(n-1)}{t}]}{\WP {\m[j:t]}{P(n)}} $.
      By applying the induction hypothesis and \cref{proof:wp-cons} we are left
      with proving
      $ P(n-1) \proves \WP {\m[j:t]}{P(n)} $
      which is implied by the premise of the rule with $i=n-1 < n$.
    \qedhere
  \end{induction}
\end{proof}
 
\subsection{Soundness of Derived Rules}
\label{sec:appendix:derived-rules}


In this section we provide derivations for the rules we claim
are derivable in \thelogic.


\subsubsection{Ownership and Distributions}
\begin{lemma}
\label{proof:sure-dirac}
  \Cref{rule:sure-dirac} is sound.
\end{lemma}

\begin{proof}
  \begin{eqexplain}
    \distAs{E\at{i}}{\delta_v}
\whichisequiv*
    \E \m{\sigmaF}, \m{\prob}.
      \Own{(\m{\sigmaF}, \m{\prob})}
      *
      \pure{\m{\prob} \circ \inv{E\at{i}} = \dirac{v}}
\whichisequiv
    \E \m{\sigmaF}, \m{\prob}.
      \Own{(\m{\sigmaF}, \m{\prob})}
      *
      \pure{\m{\prob} \circ \inv{(E\at{i} = v)} = \dirac{\True}}
\whichisequiv
    \sure{E\at{i} = v}
\qedhere
  \end{eqexplain}
\end{proof} \begin{lemma}
\label{proof:sure-eq-inj}
  \Cref{rule:sure-eq-inj} is sound.
\end{lemma}

\begin{proof}
\begin{eqexplain}
  \sure{\aexpr\at{i} = v}
  *
  \sure{\aexpr\at{i} = v'}
\whichproves*
  \distAs{\aexpr\at{i}}{\dirac{v}}
  *
  \distAs{\aexpr\at{i}}{\dirac{v'}}
  \byrule{sure-dirac}
\whichproves
  \distAs{\aexpr\at{i}}{\dirac{v}}
  \land
  \distAs{\aexpr\at{i}}{\dirac{v'}}
\whichproves
  \pure{\dirac{v}=\dirac{v'}}
  \byrule{dist-inj}
\whichproves
  \pure{v=v'}
\qedhere
\end{eqexplain}
\end{proof} \begin{lemma}
\label{proof:sure-sub}
  \Cref{rule:sure-sub} is sound.
\end{lemma}

\begin{proof}
  \begin{eqexplain}
\distAs{\aexpr_1\at{i}}{\prob}
    *
    \sure{(\aexpr_2 = f(\aexpr_1))\at{i}}
  \whichproves*
    \CC\prob v.
      \sure{\aexpr_1\at{i} = v}
      *
      \sure{(\aexpr_2 = f(\aexpr_1))\at{i}}
  \byrules{c-unit-r,c-frame}
\whichproves
    \CC\prob v.
      \sure{\aexpr_1\at{i} = v \land \aexpr_2\at{i} = f(\aexpr_1\at{i})}
  \byrule{sure-merge}
\whichproves
    \CC\prob v. \sure{\aexpr_2\at{i} = f(v)}
  \byrule{c-cons}
\whichproves
    \CC\prob v. \CC{\dirac{f(v)}} \pr{v}.\sure{\aexpr_2\at{i} = \pr{v}}
  \byrule{c-unit-l}
\whichproves
    \CC{\pr{\prob}} \pr{v}.\sure{\aexpr_2\at{i} = \pr{v}}
  \byrules{c-assoc,c-sure-proj}
  \end{eqexplain}
  where $\pr{\prob} = \bind(\prob, \fun x. \dirac{f(x)}) = \prob \circ \inv{f}$.
  By \ref{rule:c-unit-r} we thus get
  $\distAs{\aexpr_2\at{i}}{\prob \circ \inv{f}}$.
\end{proof} \begin{lemma}
\label{proof:dist-fun}
  \Cref{rule:dist-fun} is sound.
\end{lemma}

\begin{proof}
  Assume $E\from \Store \to A$ and $ f \from A \to B $, then:
  \begin{eqexplain}
    \distAs{\aexpr\at{i}}{\prob}
\whichproves*
      \CC\prob v. \sure{(\aexpr = v)\at{i}}
    \byrules{c-unit-r}
\whichproves
      \CC\prob v. \sure{(f\circ\aexpr)\at{i} = f(v)}
    \byrules{c-cons}
\whichproves
      \CC\prob v. \CC{\dirac{f(v)}} \pr{v}.\sure{(f\circ\aexpr)\at{i} = \pr{v}}
    \byrule{c-unit-l}
\whichproves
      \CC{\pr{\prob}} \pr{v}.\sure{(f\circ\aexpr)\at{i} = \pr{v}}
    \byrules{c-assoc,c-sure-proj}
  \end{eqexplain}
  where $\pr{\prob} = \bind(\prob, \fun x. \dirac{f(x)}) = \prob \circ \inv{f}$.
  By \ref{rule:c-unit-r} we thus get
  $\distAs{(f\circ\aexpr)\at{i}}{\prob \circ \inv{f}}$.
\end{proof} \begin{lemma}
\label{proof:dirac-dup}
  \Cref{rule:dirac-dup} is sound.
\end{lemma}

\begin{proof}
  \begin{eqexplain}
    \distAs{E\at{i}}{\dirac{v}}
    \whichproves*
    \sure{E\at{i} = v}
    \byrule{sure-dirac}
    \whichproves
    \sure{E\at{i} = v} \ast \sure{E\at{i} = v}
    \byrule{sure-merge}
    \whichproves
    \distAs{E\at{i}}{\dirac{v}} * \distAs{E\at{i}}{\dirac{v}}
    \byrule{sure-dirac}
  \end{eqexplain}
\end{proof} \begin{lemma}
\label{proof:dist-supp}
  \Cref{rule:dist-supp} is sound.
\end{lemma}

\begin{proof}
  \begin{eqexplain}
    \distAs{E\at{i}}{\mu}
    \whichproves*
    \CMod{\mu} v. \sure{E\at{i} = v}
    \byrule{c-unit-r}
    \whichproves
    \pure{\mu(\psupp(\mu)) = 1} \ast
    \CMod{\mu} v. \sure{E\at{i} = v}
    \whichproves
    \CMod{\mu} v. \big(\pure{v \in \psupp(\mu)} \ast \sure{E\at{i} = v}\big)
    \byrule{c-pure}
    \whichproves
    \CMod{\mu} v. \big(\sure{E\at{i} = v} \ast \sure{E\at{i} \in \psupp(\mu)} \big)
\whichproves
    \big(\CMod{\mu} v.\sure{E\at{i} = v}\big)
    \ast \sure{E\at{i} \in \psupp(\mu)}
    \byrule{sure-str-convex}
    \whichproves
    \distAs{E\at{i}}{\mu}
    \ast \sure{E\at{i} \in \psupp(\mu)}
    \byrule{c-unit-r}
  \end{eqexplain}
\end{proof} \begin{lemma}
\label{proof:prod-unsplit}
  \Cref{rule:prod-unsplit} is sound.
\end{lemma}

\begin{proof}
  \begin{eqexplain}
    \distAs{\aexpr_1\at{i}}{\prob_1} *
    \distAs{\aexpr_2\at{i}}{\prob_2}
    \whichproves*
    \CC{\prob_1} v_1.
    \CC{\prob_2} v_2.
    \bigl(
      \sure{\aexpr_1\at{i} = v_1} *
      \sure{\aexpr_2\at{i} = v_2}
    \bigr)
    \byrules{c-unit-r,c-frame}
\whichproves
    \CC{\prob_1} v_1.
    \CC{\prob_2} v_2.
      \sure{(\aexpr_1, \aexpr_2)\at{i} = (v_1, v_2)}
    \byrules{sure-merge}
\whichproves
    \CC{\prob_1 \pprod \prob_2} (v_1,v_2).
      \sure{(\aexpr_1, \aexpr_2)\at{i} = (v_1, v_2)}
    \byrules{c-assoc}
\whichproves
    \distAs{(\aexpr_1\at{i}, \aexpr_2\at{i})}{\prob_1 \otimes \prob_2}
    \byrule{c-unit-r}
  \end{eqexplain}
\end{proof}
 
\subsubsection{\Supercond}
\begin{lemma}
\label{proof:c-fuse}
  \Cref{rule:c-fuse} is sound.
\end{lemma}

\begin{proof}
  Recall that
  $
    \prob \fuse \krnl
    \is
    \fun(v,w). \prob(v)\krnl(v)(w).
  $
  which can be reformulated as
  $
    \prob \fuse \krnl =
    \bind(\prob,\fun v.(\bind(\krnl(v), \fun w.\return(v,w)))).
  $

  The $(\proves)$ direction is an instance of \ref{rule:c-assoc}.

  The $(\provedby)$ direction follows from \ref{rule:c-unassoc}:
  \begin{eqexplain}
    \CC{\prob \fuse \krnl} (v',w'). K(v',w')
\whichproves*
    \CC \prob v.
      \CC{\bind(\krnl(v), \fun w.\dirac{(v,w)})} (v',w'). K(v',w')
    \byrule{c-unassoc}
\whichproves
    \CC \prob v.
      \CC{\krnl(v)} w.
        \CC {\dirac{(v,w)}} (v',w'). K(v',w')
    \byrule{c-unassoc}
\whichproves
    \CC{\prob} v.
    \CC{\krnl(v)} w.
      K(v,w)
    \byrule{c-unit-l}
  \end{eqexplain}
\end{proof}
 \begin{lemma}
\label{proof:c-swap}
  \Cref{rule:c-swap} is sound.
\end{lemma}

\begin{proof}
  \begin{eqexplain}
    \CC{\prob_1} v_1.
      \CC{\prob_2} v_2.
        K(v_1, v_2)
\whichproves*
      \CC{\prob_1 \pprod \prob_2}
        (v_1,v_2).
          K(v_1, v_2)
    \byrule{c-fuse}
\whichproves
    \CC{\prob_2} v_2.
      \CC{\prob_1} v_1.
          K(v_1, v_2)
    \byrule{c-fuse}
  \end{eqexplain}
  Where
  \[
    \prob_1 \pprod \prob_2
    =
    \prob_1 \fuse (\fun \wtv.\prob_2)
=
    \prob_2 \fuse (\fun \wtv.\prob_1)
\]
  justifies the applications of \ref{rule:c-fuse}.
\end{proof}
 \begin{lemma}
\label{proof:sure-convex}
  \Cref{rule:sure-convex} is sound.
\end{lemma}

\begin{proof}
  By \ref{rule:sure-str-convex} with $K = \True$.
\end{proof}
 \begin{lemma}
\label{proof:dist-convex}
  \Cref{rule:dist-convex} is sound.
\end{lemma}

\begin{proof}
  \begin{eqexplain}
    \CMod{\prob} v. \distAs{E\at{i}}{\mu'}
    \whichproves*
    \CMod{\mu} v.  \CMod{\mu'} w. \sure{E\at{i} = w}
    \byrule{c-unit-r}
    \whichproves
    \CMod{\mu'} w. \CMod{\mu} v.  \sure{E\at{i} = w}
    \byrule{c-swap}
    \whichproves
    \CMod{\mu'} w.  \sure{E\at{i} = w}
    \byrule{sure-convex}
    \whichproves
    \distAs{E\at{i}}{\mu'}
    \byrule{c-unit-r}
  \end{eqexplain}
\end{proof}
 \begin{lemma}
\label{proof:c-proj}
  The following rule is sound:
  \begin{proofrule}
  \infer{
    \forall (v,\wtv)\in\psupp(\prob).
    \forall \prob'.
    \CC{\prob'} w. P(v) \proves P(v)
  }{
    \CC \prob (v,w). P(v) \lequiv
    \CC {\prob\circ\inv{\proj}} v. P(v)
  }
  \end{proofrule}
\end{lemma}
\begin{proof}
  Assume that for all $(v,\wtv)\in\psupp(\prob)$,
  $\forall \prob'. \CC{\prob'} w. P(v) \proves P(v)$
  (\ie $P(v)$ is convex).
  By \cref{lm:fuse-split} there is some~$\krnl$ such that
  $ \prob=(\prob\circ\inv{\proj})\fuse\krnl $.
  Then:
  \begin{eqexplain}
    \CC \prob (v,w). P(v)
    \whichisequiv*
    \CC {\prob\circ\inv{\proj}} v.
    \CC {\krnl(v)} w. P(v)
    \byrule{c-fuse}
    \whichisequiv
    \CC {\prob\circ\inv{\proj}} v. P(v)
  \end{eqexplain}
  The last step is justified by the convexity assumption in the~$(\proves)$
  direction,
  and by \ref{rule:c-true} and \ref{rule:c-frame} in the~$(\provedby)$ direction.
\end{proof}



\begin{lemma}
\label{proof:c-sure-proj}
  \Cref{rule:c-sure-proj} is sound.
\end{lemma}
\begin{proof}
  By \cref{proof:c-proj} and \cref{proof:sure-convex}.
\end{proof}
 \begin{lemma}
\label{proof:c-sure-proj-many}
  \Cref{rule:c-sure-proj-many} is sound.
\end{lemma}

\begin{proof}
  Let $
    X_i \is \set{ \p{x} | \ip{x}{i} \in X }
  $ for every $i\in I$.
  Then:
  \begin{eqexplain}
    \CC\prob (\m{v}, w).
      \sure{\ip{x}{i}=\m{v}(\ip{x}{i})}_{\ip{x}{i}\in X}
\whichisequiv*
    \CC{\prob} (\m{v}, w).
      \LAnd_{i\in I}
        \sure{\LAnd_{\p{x}\in X_i} \ip{x}{i}=\m{v}(\ip{x}{i})}
\whichisequiv
    \LAnd_{i\in I}
      \CC{\prob} (\m{v}, w).
        \sure{\LAnd_{\p{x}\in X_i} \ip{x}{i}=\m{v}(\ip{x}{i})}
    \byrule{c-and}
\whichisequiv
    \LAnd_{i\in I}
      \CC{\prob\circ\inv{\proj_1}} \m{v}.
        \sure{\LAnd_{\p{x}\in X_i} \ip{x}{i}=\m{v}(\ip{x}{i})}
    \byrule{c-sure-proj}
\whichisequiv
    \CC{\prob\circ\inv{\proj_1}} \m{v}.
      \LAnd_{i\in I}
        \sure{\LAnd_{\p{x}\in X_i} \ip{x}{i}=\m{v}(\ip{x}{i})}
    \byrule{c-and}
\whichisequiv
    \CC{\prob\circ\inv{\proj_1}} \m{v}.
      \sure{\ip{x}{i}=\m{v}(\ip{x}{i})}_{\ip{x}{i}\in X}
  \end{eqexplain}
  Note that the (iterated) applications of \ref{rule:c-and}
  satisfy the side condition
  because the inner assertions are by construction on disjoint indices.
  The backward direction of \ref{rule:c-and} holds by
  the standard laws of conjunction.
\end{proof}
 \begin{lemma}
\label{proof:c-extract}
  \Cref{rule:c-extract} is sound.
\end{lemma}

\begin{proof}
  \begin{eqexplain}
    \CC{\prob_1} v_1. \bigl(
      \sure{\aexpr_1\at{i} = v_1} *
      \distAs{\aexpr_2\at{i}}{\prob_2}
    \bigr)
\whichproves*
      \CC{\prob_1} v_1.
        \bigl(
          \sure{\aexpr_1\at{i}=v_1} *
          \CC{\prob_2} v_2. \sure{\aexpr_2\at{i}=v_2}
        \bigr)
      \byrule{c-unit-r}
    \whichproves
      \CC{\prob_1} v_1.
      \CC{\prob_2} v_2.
        \bigl(
          \sure{\aexpr_1\at{i}=v_1} *
          \sure{\aexpr_2\at{i}=v_2}
        \bigr)
      \byrules{c-frame}
    \whichproves
      \CC{\prob_1} v_1.
      \CC{\prob_2} v_2.
        \sure{\aexpr_1\at{i}=v_1 \land \aexpr_2\at{i}=v_2}
      \byrules{sure-merge}
    \whichproves
      \CC{\prob_1 \pprod \prob_2} (v_1,v_2).
        \sure{(\aexpr_1\at{i},\aexpr_2\at{i})=(v_1,v_2)}
      \byrules{c-assoc}
    \whichproves
      \distAs{(\aexpr_1\at{i},\aexpr_2\at{i})}{(\prob_1 \pprod \prob_2)}
      \byrule{c-unit-r}
    \whichproves
      \distAs{\aexpr_1\at{i}}{\prob_1} *
      \distAs{\aexpr_2\at{i}}{\prob_2}
      \byrule{prod-split}
  \end{eqexplain}
\end{proof}
 \begin{lemma}
\label{proof:c-dist-proj}
  \Cref{rule:c-dist-proj} is sound.
\end{lemma}

\begin{proof}
  By \cref{proof:c-proj} and \cref{proof:dist-convex}.
\end{proof}
 
\subsubsection{Relational Lifting}
\begin{lemma}
\label{proof:rl-cons}
  \Cref{rule:rl-cons} is sound.
\end{lemma}

\begin{proof}
  \begin{eqexplain}
    \cpl{R_1}
\whichis*
      \E \prob.
        \pure{\prob(R_1) = 1} *
        \CC\prob \m{v}.
          \sure{\ip{x}{i} = \m{v}(\ip{x}{i})}_{\ip{x}{i}\in X}
\whichproves
      \E \prob.
        \pure{\prob(R_2) = 1} *
        \CC\prob \m{v}.
          \sure{\ip{x}{i} = \m{v}(\ip{x}{i})}_{\ip{x}{i}\in X}
    \by{$R_1 \subseteq R_2$}
\whichis \cpl{R_2}
  \qedhere
  \end{eqexplain}
\end{proof} \begin{lemma}
\label{proof:rl-unary}
  \Cref{rule:rl-unary} is sound.
\end{lemma}

\begin{proof}
  \begin{eqexplain}
    \cpl{R}
\whichis*
    \E \prob.
      \pure{\prob(R) = 1} *
      \CC\prob \m{v}.
        \sure{\ip{x}{i} = \m{v}(\p{x}{i})}_{\p{x}{i}\in X}
\whichproves
    \E \prob.
      \CC\prob \m{v}.
        \pure{\m{v} \in R}
        * \sure{\ip{x}{i} = \m{v}(\p{x}{i})}_{\p{x}{i}\in X}
    \byrule{c-pure}
\whichproves
    \E \prob.
      \CC\prob \m{v}.
        \sure{R(\p{x}_1\at{i}, \dots , \p{x}_n\at{i})}
\whichproves
    \E \prob.
      \sure{R(\p{x}_1\at{i}, \dots, \p{x}_n\at{i})}
    \byrule{sure-convex}
\whichproves
    \sure{R(\p{x}_1\at{i}, \dots, \p{x}_n\at{i})}
  \qedhere
  \end{eqexplain}
\end{proof}

%
 \begin{lemma}
\label{proof:rl-eq-dist}
  \Cref{rule:rl-eq-dist} is sound.
\end{lemma}

\begin{proof}
  \begin{eqexplain}
    \cpl{x\at{i} = y\at{j}}
\proves{}&
    \E \prob'.
     \CC{\prob'} (v_1,v_2).\bigl(
      \sure{\Ip{x}{i} = v_1} \land
      \sure{\Ip{y}{j} = v_2} \land
      \pure{v_1=v_2}
    \bigr)
\whichproves
    \E \prob'.
     \CC{\prob'} (v_1, v_2).\bigl(
      \sure{\Ip{x}{i} = v_1} \land
      \sure{\Ip{y}{j} = v_1 }
    \bigr)
    \byrule{c-cons}
\whichproves
    \E \prob'.
     \CC{\prob'\circ\inv{\proj_1}} v_1.\bigl(
      \sure{\Ip{x}{i} = v_1} \land
      \sure{\Ip{y}{j} = v_1 }
    \bigr)
    \byrule{c-sure-proj}
\whichproves
    \E \prob.
     \CC{\prob} v_1.\bigl(
      \sure{\Ip{x}{i} = v_1} \land
      \sure{\Ip{y}{j} = v_1 }
    \bigr)
    \by{$ \prob = \prob'\circ\inv{\proj_1} $}
\whichproves
    \E \prob.
      \bigl(
      \CC\prob v_1.
       \sure{\Ip{x}{i} = v_1}
      \bigr)
      \land
      \bigl(
      \CC\prob v_1.
        \sure{\Ip{y}{j} = v_1}
      \bigr)
\whichproves
    \E \prob.
    \distAs{\Ip{x}{i}}{\mu}
    \land
    \distAs{\Ip{y}{j}}{\mu}
    \byrule{c-unit-r}
\whichproves
    \E \prob.
    \distAs{\Ip{x}{i}}{\mu}
    \ast
    \distAs{\Ip{y}{j}}{\mu}
    \byrule{and-to-star}
  \end{eqexplain}
\end{proof} \begin{lemma}
  \Cref{rule:rl-convex} is sound.
\end{lemma}

\begin{proof}
 \begin{eqexplain}
   \CMod{\prob} a \st \cpl{R}
\whichis*
\CMod{\prob} a \st
     \E \prob'.
       \pure{\prob'(R) = 1} *
       \bigl(
         \CC{\prob'} \m{v}.
           \sure{\ip{x}{i} = \m{v}(\ip{x}{i})}_{\ip{x}{i}\in X}
       \bigr)
\whichproves
     \E \krnl.
     \CMod{\prob} a \st
       \bigl(
         \CC{\krnl(a)} \m{v}.
           \sure{\ip{x}{i} = \m{v}(\ip{x}{i})}_{\ip{x}{i}\in X}
           * \pure{R(\m{v})}
       \bigr)
   \byrules{c-pure,c-skolem}
\whichproves
     \E \hat{\prob}.
     \CMod{\hat{\prob}} (a,\m{v}) \st
       \sure{\ip{x}{i} = \m{v}(\ip{x}{i})}_{\ip{x}{i}\in X}
       * \pure{R(\m{v})}
   \byrule{c-fuse}
\whichproves
     \E \hat{\prob}.
     \pure{\hat{\prob} \circ \inv{\proj_2}(R) = 1} *
     \CMod{\hat{\prob}} (a,\m{v}) \st
       \sure{\ip{x}{i} = \m{v}(\ip{x}{i})}_{\ip{x}{i}\in X}
   \byrule{c-pure}
\whichproves
     \E \hat{\prob}.
     \pure{\hat{\prob} \circ \inv{\proj_2}(R) = 1} *
     \CMod{\hat{\prob} \circ \inv{\proj_2}} \m{v} \st
       \sure{\ip{x}{i} = \m{v}(\ip{x}{i})}_{\ip{x}{i}\in X}
   \byrule{c-sure-proj-many}
\whichproves
     \E \hat{\prob}'.
     \pure{\hat{\prob}'(R) = 1} *
     \CMod{\hat{\prob}'} \m{v} \st
       \sure{\ip{x}{i} = \m{v}(\ip{x}{i})}_{\ip{x}{i}\in X}
\whichis \cpl{R}
 \end{eqexplain}
 In the derivation we use
 $ \hat{\prob} = \prob \fuse \krnl $,
 and
 $\hat{\prob}' = \hat{\prob} \circ \inv{\proj_2}$.
\end{proof}
 \begin{lemma}
\label{proof:rl-merge}
  \Cref{rule:rl-merge} is sound.
\end{lemma}

\begin{proof}
  Let $R_1\in \Val^{X_1}$ and $R_2\in \Val^{X_2}$
  and let
  $ X = X_1 \inters X_2 $,
  $ Y_1 = X_1 \setminus X $, and
  $ Y_2 = X_2 \setminus X $, so that
  $ X_1 \union X_2 = Y_1 \dunion X \dunion Y_2 $.

  By definition, $\cpl{R_1} * \cpl{R_2}$ entails that for some
  $ \prob_1,\prob_2 $ with $ \prob_1(R_1)=1 $ and $ \prob_1(R_2)=1 $:
  \begin{eqexplain}
&
    \CC{\prob_1} \m{v}_1.
      (\sure{\ip{x}{i} = \m{v}_1(\ip{x}{i})}_{\ip{x}{i}\in X_1}) *
\CC{\prob_2} \m{v}_2.
      (\sure{\ip{x}{i} = \m{v}_2(\ip{x}{i})}_{\ip{x}{i}\in X_2})
\whichproves
  \CC{\prob_1} (\m{w}_1,\m{v}_1). (
    \sure{\ip{y}{i} = \m{w}_1(\ip{y}{i})}_{\ip{y}{i}\in Y_1}
    \land
    \sure{\ip{x}{i} = \m{v}_1(\ip{x}{i})}_{\ip{x}{i}\in X}
    * \pure{(\m{w}_1\m{v}_1) \in R_1}
  ) *
  {} \\ &
  \CC{\prob_2} (\m{w}_2,\m{v}_2). (
    \sure{\ip{y}{i} = \m{w}_2(\ip{y}{i})}_{\ip{y}{i}\in Y_2}
    \land
    \sure{\ip{x}{i} = \m{v}_2(\ip{x}{i})}_{\ip{x}{i}\in X}
    * \pure{(\m{w}_2\m{v}_2) \in R_2}
  )
  \byrule{c-pure}
\whichproves
  \CC{\prob_1} (\m{w}_1,\m{v}_1).
    \CC{\prob_2} (\m{w}_2,\m{v}_2).
    \begin{pmatrix*}[l]
    \sure{\ip{y}{i} = \m{w}_1(\ip{y}{i})}_{\ip{y}{i}\in Y_1}
          \land
          \sure{\ip{x}{i} = \m{v}_1(\ip{x}{i})}_{\ip{x}{i}\in X} * {}
    \\
    \sure{\ip{y}{i} = \m{w}_2(\ip{y}{i})}_{\ip{y}{i}\in Y_2}
          \land
          \sure{\ip{x}{i} = \m{v}_2(\ip{x}{i})}_{\ip{x}{i}\in X} * {}
    \\
      \pure{(\m{w}_1\m{v}_1) \in R_1}
    * \pure{(\m{w}_2\m{v}_2) \in R_2}
    \end{pmatrix*}
  \byrule{c-frame}
\whichproves
  \CC{\prob_1} (\m{w}_1,\m{v}_1).
    \CC{\prob_2} (\m{w}_2,\m{v}_2).
    \begin{pmatrix*}[l]
    \sure{\ip{y}{i} = \m{w}_1(\ip{y}{i})}_{\ip{y}{i}\in Y_1}
          \land
          \sure{\ip{x}{i} = \m{v}_1(\ip{x}{i})}_{\ip{x}{i}\in X} * {}
    \\
    \sure{\ip{y}{i} = \m{w}_2(\ip{y}{i})}_{\ip{y}{i}\in Y_2}
          \land
          \sure{\ip{x}{i} = \m{v}_2(\ip{x}{i})}_{\ip{x}{i}\in X} * {}
    \\
      \pure{(\m{w}_1\m{v}_1) \in R_1}
    * \pure{(\m{w}_2\m{v}_2) \in R_2}
    * \pure{\m{v}_1=\m{v}_2}
    \end{pmatrix*}
  \byrule{sure-eq-inj}
\whichproves
  \CC{\prob_1} (\m{w}_1,\m{v}_1).
  \CC{\prob_2} (\m{w}_2,\m{v}_2).
    \begin{pmatrix*}[l]
    \sure{\ip{y}{i} = \m{w}_1(\ip{y}{i})}_{\ip{y}{i}\in Y_1} \land {}
    \\
    \sure{\ip{x}{i} = \m{v}_1(\ip{x}{i})}_{\ip{x}{i}\in X} \land {}
    \\
    \sure{\ip{y}{i} = \m{w}_2(\ip{y}{i})}_{\ip{y}{i}\in Y_2} * {}
    \\
    \pure{(\m{w}_1\m{v}_1) \in R_1 \land (\m{w}_2\m{v}_1) \in R_2}
    \end{pmatrix*}
  \byrule{c-cons}
\whichproves
  \CC{\prob_1} (\m{w}_1,\m{v}_1).
  \CC{\prob_2 \circ \inv{\pi_1}} (\m{w}_2).
    \begin{pmatrix*}[l]
    \sure{\ip{y}{i} = \m{w}_1(\ip{y}{i})}_{\ip{y}{i}\in Y_1} \land {}
    \\
    \sure{\ip{x}{i} = \m{v}_1(\ip{x}{i})}_{\ip{x}{i}\in X} \land {}
    \\
    \sure{\ip{y}{i} = \m{w}_2(\ip{y}{i})}_{\ip{y}{i}\in Y_2} * {}
    \\
    \pure{(\m{w}_1\m{v}_1) \in R_1 \land (\m{w}_2\m{v}_1) \in R_2}
    \end{pmatrix*}
  \byrule{c-sure-proj}
  \end{eqexplain}



  Thus by letting $
  \prob = \prob_1 \pprod (\prob_2 \circ \inv{\pi_1})
  =\bind(\prob_1, \krnl_2)
  $ where
  \[
    \krnl_2 = \fun (\m{w}_1\m{v}_1).(
      \bind(\prob_2,\fun (\m{w}_2,\m{v}_2).
                        \return (\m{w}_1\m{w}_2\m{v}_1))
    )
  \]
  we obtain:
  \begin{eqexplain}
  &
  \CC{\prob_1} (\m{w}_1',\m{v}_1').
    \CC{\krnl_2(\m{w}_1',\m{v}_1')} (\m{w}_1,\m{w}_2,\m{w}).
    \begin{pmatrix*}[l]
    \sure{\ip{y}{i} = \m{w}_1(\ip{y}{i})}_{\ip{y}{i}\in Y_1} \land {}
    \\
    \sure{\ip{x}{i} = \m{w}(\ip{x}{i})}_{\ip{x}{i}\in X} \land {}
    \\
    \sure{\ip{y}{i} = \m{w}_2(\ip{y}{i})}_{\ip{y}{i}\in Y_2} * {}
    \\
    \pure{(\m{w}_1\m{w}) \in R_1 \land (\m{w}_2\m{w}) \in R_2}
    \end{pmatrix*}
\whichproves
  \CC{\prob} (\m{w}_1,\m{w}_2,\m{w}).
    \begin{pmatrix*}[l]
    \sure{\ip{y}{i} = \m{w}_1(\ip{y}{i})}_{\ip{y}{i}\in Y_1} \land {}
    \\
    \sure{\ip{x}{i} = \m{w}(\ip{x}{i})}_{\ip{x}{i}\in X} \land {}
    \\
    \sure{\ip{y}{i} = \m{w}_2(\ip{y}{i})}_{\ip{y}{i}\in Y_2}
    \\
    \pure{(\m{w}_1\m{w}) \in R_1 \land (\m{w}_2\m{w}) \in R_2}
    \end{pmatrix*}
  \byrules{c-assoc,c-sure-proj}
\whichproves
  \CC{\prob} \m{v}.
    \sure{\ip{x}{i} = \m{v}(\ip{x}{i})}_{\ip{x}{i}\in (X_1\union X_2)}
    * \pure{(\restr{\m{v}}{X_1}) \in R_1 \land (\restr{\m{v}}{X_2}) \in R_2}
  \end{eqexplain}
  The result gives us $ \cpl{R_1 \land R_2} $
  by \ref{rule:c-pure} and \cref{def:rel-lift}.
\end{proof} \begin{lemma}
\label{proof:rl-sure-merge}
  \Cref{rule:rl-sure-merge} is sound.
\end{lemma}

\begin{proof}
  \begin{eqexplain}
    \cpl{R} * \sure{\ip{x}{i} = \expr\at{i}}
  \whichproves*
    \E\prob.
      \CC\prob \m{v}. \bigl(
        \sure{\ip{y}{i} = \m{v}(\ip{y}{i})}_{\ip{y}{i}\in X}
        * \pure{R(\m{v})}
      \bigr)
      * \sure{\ip{x}{i} = \expr\at{i}}
  \bydef
\whichproves
    \E\prob.
      \CC\prob \m{v}. \bigl(
        \sure{\ip{y}{i} = \m{v}(\ip{y}{i})}_{\ip{y}{i}\in X}
        * \pure{R(\m{v})}
        * \sure{\ip{x}{i} = \expr\at{i}}
      \bigr)
  \byrule{c-frame}
\whichproves
    \E\prob.
      \CC\prob \m{v}. \bigl(
        \sure{\ip{y}{i} = \m{v}(\ip{y}{i})}_{\ip{y}{i}\in X}
        * \pure{R(\m{v})}
        * \sure{\ip{x}{i} = \sem{\expr\at{i}}(\m{v})}
      \bigr)
  \by{$\pvar(\expr\at{i}) \subs X$}
\whichproves
    \E\prob.
      \CC\prob \m{v}. \bigl(
        \sure{\ip{y}{i} = \m{v}(\ip{y}{i})}_{\ip{y}{i}\in X}
        * \pure{R(\m{v})}
        * \CC{\dirac{\sem{\expr\at{i}}(\m{v})}} w.\sure{\ip{x}{i} = w}
      \bigr)
  \byrule{c-unit-l}
\whichproves
    \E\prob.
      \CC\prob \m{v}.
      \CC{\dirac{\sem{\expr\at{i}}(\m{v})}} w.
      \bigl(
        \sure{\ip{y}{i} = \m{v}(\ip{y}{i})}_{\ip{y}{i}\in X}
        * \pure{R(\m{v})}
        * \sure{\ip{x}{i} = w}
      \bigr)
  \byrule{c-frame}
\whichproves
    \E\prob'.
      \CC{\prob'} \m{v}'.
      \begin{grp}
        \sure{\ip{y}{i} = \m{v}'(\ip{y}{i})}_{\ip{y}{i}\in X}
        * \sure{\ip{x}{i} = \m{v}'(\ip{x}{i})} \\ {}
        * \pure{R(\m{v}') \land \sem{\expr\at{i}}(\m{v}') = \m{v}'(\ip{x}{i})}
      \end{grp}
  \byrules{c-pure,c-assoc}
\whichproves
    \cpl{R \land \ip{x}{i} = \expr\at{i}}
  \end{eqexplain}
  where we let
  $
    \prob' \is
      \left(
      \DO{
        \m{v} <- \prob;
        \return(\m{v}\upd{\ip{x}{i} -> \sem{\expr\at{i}}(\m{v})})
      }
      \right).
  $
\end{proof} \begin{lemma}
\label{proof:coupling}
  \Cref{rule:coupling} is sound.
\end{lemma}

\begin{proof}
  Assuming
  $\prob \circ \inv{\proj_1} = \prob_1$,
  $\prob \circ \inv{\proj_2} = \prob_2$, and
  $\prob(R) = 1$, we have:
  \begin{eqexplain}
    \distAs{\p{x}_1\at{\I1}}{\prob_1} *
    \distAs{\p{x}_2\at{\I2}}{\prob_2}
\whichproves*
    \CC{\prob_1} v. \sure{x_1\at{1} = v} *
    \CC{\prob_2} w. \sure{x_2\at{2} = w}
    \byrule{c-unit-r}
\whichproves
    \CC{\prob} (v, w). \sure{x_1\at{1} = v} *
    \CC{\prob} (v, w). \sure{x_2\at{2} = w}
    \byrule{c-sure-proj}
\whichproves
    \CC{\prob} (v, w). \sure{x_1\at{1} = v} \land
    \CC{\prob} (v, w). \sure{x_2\at{2} = w}
    \byrule{and-to-star}
\whichproves
    \CC{\prob} (v, w).
      (\sure{x_1\at{1} = v} \land
      \sure{x_2\at{2} = w})
    \byrule{c-and}
\whichproves
    \cpl{R(x_1\at{1}, x_2\at{2})}
    \by{$\prob(R) = 1$}
  \end{eqexplain}
\end{proof}
 
\subsubsection{Weakest Precondition}
\begin{lemma}
\label{proof:wp-loop-0}
  \Cref{rule:wp-loop-0} is sound.
\end{lemma}

\begin{proof}
  Special case of \ref{rule:wp-loop} with $n=0$,
  which makes the premises trivial.
\end{proof} \begin{lemma}
\label{proof:wp-loop-lockstep}
  \Cref{rule:wp-loop-lockstep} is sound.
\end{lemma}

\begin{proof}
  We derive the following rule:
  \[
  \infer*{
    \forall k < n\st
      P(k) \proves \WP {\m[i: t, j: t']}{P(k+1)}
  }{
    P(0) \proves
    \WP {\m[i: (\Loop{n}{t}), j: (\Loop{n}{t'})]} {P(n)}
  }
  \]
  (for $n\in \Nat$ and $i \ne j$)
  from the standard \ref{rule:wp-loop},
  as follows.
  Let
  \[
    P'(k) \is
    \WP {\m[j: \Loop{k}{t'}]} {P(k)}
  \]
  Note that
  $P(0) \proves P'(0)$
  by \ref{rule:wp-loop-0}.
  Then we can apply the \ref{rule:wp-loop} using $P'$ as a loop invariant
  \begin{derivation}
    \infer*[Right=\ref{rule:wp-nest}]{
    \infer*{
    \infer*[Right=\ref{rule:wp-loop}]{
    \infer*{
    \infer*[Right=\ref{rule:wp-nest}]{
    \infer*[Right=\ref{rule:wp-loop-unf}]{
    \infer*[Right=\ref{rule:wp-cons}]{
    \infer*[Right=\ref{rule:wp-nest}]{
    \infer*{}{
      \forall k \leq n \st
      P(k)
      \proves
      \WP {\m[i: t, j: t']} {P(k+1)}
    }}{
      \forall k \leq n \st
      P(k)
      \proves
        \WP {\m[j: t']} {
          \WP {\m[i: t]} {P(k+1)}
      }
    }}{
      \forall k \leq n \st
      \WP {\m[j: \Loop{k}{t'}]} {P(k)}
      \proves
      \WP {\m[j: \Loop{k}{t'}]}[\big]{
        \WP {\m[j: t']} {
          \WP {\m[i: t]} {P(k+1)}
        }
      }
    }}{
      \forall k \leq n \st
      \WP {\m[j: \Loop{k}{t'}]} {P(k)}
      \proves
      \WP {\m[j: \Loop{(k+1)}{t'}]}[\big]{
        \WP {\m[i: t]} {P(k+1)}
      }
    }}{
      \forall k \leq n \st
      \WP {\m[j: \Loop{k}{t'}]} {P(k)}
      \proves
      \WP {\m[i: t]}[\big]{
        \WP {\m[j: \Loop{(k+1)}{t'}]} {P(k+1)}
      }
    }}{
      \forall k \leq n \st
        P(k)' \proves
          \WP {\m[i: t]} {P'(k+1)}
    }}{
      P'(0) \proves
      \WP {\m[i: (\Loop{n}{t})]}{
        P'(n)
      }
    }}{
      P(0) \proves
      \WP {\m[i: (\Loop{n}{t})]}{
        \WP{\m[j: (\Loop{n}{t'})]} {P(n)}
      }
    }}{
      P(0) \proves
      \WP {\m[i: (\Loop{n}{t}), j: (\Loop{n}{t'})]} {P(n)}
    }
  \end{derivation}
  From bottom to top,
    we focus on component $i$ using \ref{rule:wp-nest};
    then we use $P(0) \proves P'(0)$ and transitivity of entailment
    to rewrite the goal using the invariant $P'$;
    we then use \ref{rule:wp-loop} and unfold the invariant;
    using \ref{rule:wp-nest} twice we can swap the two components so that
    component~$j$ is the topmost WP in the assumption and conclusion;
    using \ref{rule:wp-loop-unf} we break off the first $k$ iterations at~$j$;
    finally, using \ref{rule:wp-cons} we can eliminate the topmost
    WP on both sides of the entailments.

  It is straightforward to adapt the argument for any number of components
  looping the same number of times.
\end{proof} \begin{lemma}
\label{proof:wp-rl-assign}
  \Cref{rule:wp-rl-assign} is sound.
\end{lemma}

\begin{proof}
  Define
    $\m{\permap}_R \is (\m{\permap} \setminus \ip{x}{i})/2$ and
    $ \m{\permap}_{\p{x}} \is \m{\permap} - \m{\permap}_R $;
  note that by $\m{\permap}(\ip{x}{i})=1$
  we have $\m{\permap}_{\p{x}}(\ip{x}{i})=1$.
  We first show that the following hold:
  \begin{align}
    \cpl{R}\withp{\m{\permap}}
    &\proves
    \cpl{R}\withp{\m{\permap}_R} * (\m{\permap}_{\p{x}})
    \\
    \cpl{R}\withp{\m{\permap}_R}
      * \sure{\p{x}\at{i} = \expr\at{i}}\withp{\m{\permap}_{\p{x}}}
    &\proves
    \cpl{R \land \p{x}\at{i} = \expr\at{i}}\withp{\m{\permap}}
  \end{align}
  The first entailment holds because $\cpl{R}$ is permission-scaling-invariant
  (see \cref{sec:appendix:permissions})
  and by the assumption that $\p{x}\notin \pvar(R)$.
  The second entailment holds by \ref{rule:rl-sure-merge}.

  We can then derive:
  \[
  \begin{derivation}
    \infer*[Right=\ref{rule:wp-cons}]{
    \infer*[Right=\ref{rule:wp-frame}]{
    \infer*[Right=\ref{rule:wp-assign}]{ }{
      (\m{\permap}_{\p{x}})
      \proves
      \WP {\m[i: \code{x:=}\expr]}[\big] {
          \sure{\p{x}\at{i} = \expr\at{i}}\withp{\m{\permap}_{\p{x}}}
      }
    }}{
      \cpl{R}\withp{\m{\permap}_R}
      * (\m{\permap}_{\p{x}})
      \proves
      \WP {\m[i: \code{x:=}\expr]}[\big] {
        \cpl{R}\withp{\m{\permap}_R}
          * \sure{\p{x}\at{i} = \expr\at{i}}\withp{\m{\permap}_{\p{x}}}
      }
    }}{
      \cpl{R}\withp{\m{\permap}}
      \proves
      \WP {\m[i: \code{x:=}\expr]}[\big] {
        \cpl{R \land \p{x}\at{i} = \expr\at{i}}\withp{\m{\permap}}
      }
    }
  \end{derivation}
  \qedhere
  \]
\end{proof} \begin{lemma}
\label{proof:wp-if-unary}
  \Cref{rule:wp-if-unary} is sound.
\end{lemma}

\begin{proof}
  From the premises, we derive:
  \[
  \begin{derivation}
  \infer*[right=\ref{rule:c-wp-swap}]{
  \infer*[Right={\ref{rule:c-unit-r},\ref{rule:c-frame}}]{
  \infer*[Right=\ref{rule:c-cons}]{
  \infer*[Right=\ref{rule:wp-bind}]{
  \infer*[Right=\ref{rule:wp-if-prim}]{
  \infer*{
    P * \sure{\Ip{e}{1}=1} \gproves \WP {\m[\I1:t_1]} {Q(1)}
    \\
    P * \sure{\Ip{e}{1}=0} \gproves \WP {\m[\I1:t_2]} {Q(0)}
  }{\forall b\in\set{0,1}\st
    P * \sure{\Ip{e}{1}=1}
    \gproves
    \ITE{b}{\WP{\m[\I1:t_1]}{Q(1)}}{\WP{\m[\I1:t_2]}{Q(0)}}
  }}{\forall b\in\set{0,1}\st
    P * \sure{\Ip{e}{1}=b}
    \gproves
    \WP {\m[\I1:
        (\code{if $\;b\;$ then $\;t_1\;$ else $\;t_2$})
      ]}{
      Q(b\beq 1)
    }
  }}{\forall b\in\set{0,1}\st
    P * \sure{\Ip{e}{1}=b}
    \gproves
    \WP {\m[\I1:
      (\code{if e then $\;t_1\;$ else $\;t_2$})
    ]}{
      Q(b\beq 1)
    }
  }}{\CC{\beta} b.(P * \sure{\Ip{e}{1}=b})
    \gproves
    \CC{\beta} b.
    \WP {\m[\I1:
      (\code{if e then $\;t_1\;$ else $\;t_2$})
    ]}{
      Q(b\beq 1)
    }
  }}{P * \distAs{\Ip{e}{1}}{\beta}
    \gproves
    \CC{\beta} b.
    \WP {\m[\I1:
      (\code{if e then $\;t_1\;$ else $\;t_2$})
    ]}{
      Q(b\beq 1)
    }
  }}{P * \distAs{\Ip{e}{1}}{{\beta}}
    \gproves
    \WP {\m[\I1:
        (\code{if e then $\;t_1\;$ else $\;t_2$})
      ]}{
      \CC{\beta} b.Q(b\beq 1)
    }
  }
  \end{derivation}
  \qedhere
  \]
\end{proof}

